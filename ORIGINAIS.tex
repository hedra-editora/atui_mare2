\chapter*{}

\thispagestyle{empty}
\vspace{\fill}
\begin{flushright}
\emph{Aos meus colegas voluntários dos Expedicionários\\
da Saúde, que tantas histórias viveram\\
comigo, algumas delas contadas neste livro.}

\bigskip

\emph{E à Luisa Purchio, pela paciência infinita e pelas incontáveis intervenções até chegar ao texto final.}
\end{flushright}

\chapter*{}

\epigraph{\emph{O rio não quer chegar a lugar algum, só quer ser mais profundo.}}
{Guimarães Rosa}

Quase tonta e deixando para trás a salva de palmas abafada pela grande
porta de madeira, Dani saiu da sala de reuniões da diretoria da
faculdade e deu de cara com Theo.

``Você também foi chamado?''

``A nota da minha prova não saiu no mural, pelo visto também vou ganhar
um reconhecimento'', disse ele olhando para a placa com o símbolo da
faculdade no estojo de veludo verde que ela trazia nas mãos. Desde que
terminaram o namoro, cerca de dois anos atrás, eles não haviam tido
muita chance de estar sozinhos. Ele estava todo fortinho e com uma postura
confiante, mas surpreendentemente ela não se sentia atraída por ele,
alguma coisa não combinava.

``Vejo que foi homenageada.''

Dani abriu um sorriso e começou a contar sobre como sua trajetória
acadêmica havia sido louvada pelos eméritos professores há poucos
minutos, mas Theo nem a deixou terminar.

``Como você conseguiu acertar o diagnóstico da prova, Dani?''

``Não acertei. Não dava para acertar.''

``E você foi homenageada mesmo assim?''

Dani era muito sensível e detectou certa dose de desdém na voz de Theo.

``É só um reconhecimento, Theo. Você vai na festa hoje?''

Dani tentou mudar o assunto, mas acabou entrando em um ainda pior.

``Claro, como convidado especial. Vou comandar o som a partir da
meia"-noite.''

Ele se entusiasmou enumerando as festas que havia tocado e para outras
que havia sido convidado, e começou a explicar as dificuldades que
estava tendo para encaixar todos estes eventos em sua concorrida agenda.
Quanto mais Theo falava, mais ela se lembrava por que o relacionamento
entre os dois não havia dado certo.

Com tristeza, entendia também seu afastamento do velho grupo de amigos.
Quanto mais sucesso ele achava que fazia, menos se identificava com
os assuntos da turma, menos glamorosos, mas sem dúvida mais profundos e
consistentes.

No fundo, Theo não era assim, mas a vastidão do mundo que se abriu o
seduziu, e, além disso, ele dava suas escorregadelas usando o mare
nostrum em causa própria com alguma frequência, mesmo com as recomendações
do seu mestre para que não transformasse a navegação em um instrumento
vulgar para tirar vantagens pessoais.

Ele era bom naquilo. Excelente aprendiz, se tornou em pouco tempo um dos melhores
navegadores já vistos, e acessava como ninguém
a enorme fonte de conhecimento da humanidade. Isto mudou sua vida: logo
conquistou a fama e até a conta bancária que sempre sonhara,
consequência dos investimentos que fazia no mercado financeiro após ter
acesso a informações privilegiadas.

Theo foi se afastando cada vez mais do cara que era antes, mas não era
sua mudança que incomodava Dani, nem o fato de não saber direito o que
havia acontecido. Antes fosse. O que a deixava triste, mesmo, era ver um
cara legal se apegar a coisas tão superficiais. Será que ela o havia
avaliado mal inicialmente e ele nunca havia sido mais que um idiota?

``E você, Dani. Vai na festa?''

A pergunta a trouxe de volta para a realidade.

``Não, nunca gostei desses eventos. Aliás, nem você. Quando você
aprendeu a ser \versal{DJ}?''

``Aprendi por aí\ldots{} Você deveria ir na festa, vai ser muito
legal.''

``Prefiro ir com o pessoal no bar.''

Dani começou a explicar o novo trabalho de Mayara e como a amiga morria
de medo de atravessar sozinha a escura sala de manutenção, mas Theo não
prestava atenção em mais nada do que ela falava. Seu celular apitou
mostrando que havia algo errado com seu pai, e Theo abriu o aplicativo
para checar o problema. Gastou uma fortuna com as câmeras e o sistema de
controle, mas assim podia cuidar de tudo à distância. Com apenas um
toque, aumentou a temperatura do ar"-condicionado do quarto que tinha
abaixado demais e acionado o alarme. Sem entender nada e muito irritada,
Dani, achando que ele estava respondendo alguma mensagem ou apenas
checando alguma rede social, parou de falar e o encarou querendo
matá"-lo.

``Senhor Theo, por favor.''

A secretária veio chamá"-lo. Theo despertou em um pulo, alinhou a gravata
e, sem nem ao menos se despedir de Dani, entrou no grande auditório
sendo recebido por não menos que Eurípedes Carvalhosa, o diretor e a
figura mais respeitada da faculdade.

``Theo, você sempre foi um aluno brilhante dessa ilustre casa. E nós já
te homenageamos à altura. Nessa ocasião, no entanto, não são méritos que
justificam essa convocação. Queira sentar"-se, por gentileza.''

Surpreso e sem saber para onde aquilo ia levar, Theo engoliu seco, olhou
para a cadeira separada especialmente para ele, posicionada de frente
aos professores, e se sentou. Acomodados nas bancadas de madeira
entalhada, eles o observavam em silêncio. Ao fundo, o professor
Alcântara franzia a testa mirando o infinito, recostado nas altas
cortinas de veludo verde. Theo não entendeu por que seu mestre, que abriu
a porta para esta sua nova vida, com quem se aventurou tantas vezes no mare
nostrum, estava tão distante e não o preparou para aquele momento.

``Existe uma forte suspeita de fraude na sua prova, Theo. É de se saber
que condutas como essa são claramente repudiáveis pelo Código de Honra
de nossa instituição.''

Os outros professores começaram a murmurar e se levantaram com a cópia
da avaliação de Theo na mão.

``Como você acertou esse diagnóstico?''

``O caso é complicadíssimo. Tal dedução não é possível, não com as
informações fornecidas.''

``Perguntou"-se o diagnóstico para que fosse dito que era impossível a
apuração desta informação.''

``Só eu tive acesso a esse caso, ninguém mais poderia saber a
resposta.''

``O rapaz só pode ter tido informações privilegiadas. Como as
conseguiu?''

``Acalmem"-se, ilustríssimos professores. Sentem"-se, por gentileza. Theo,
nenhum outro aluno acertou o diagnóstico do paciente, mas esse era o
ponto menos importante da questão. Observe a resposta da Daniela, como
ela descreve minuciosamente os sintomas do paciente, as hipóteses
diagnósticas e intervenções. Esse é nosso papel, somos condutores de
descobertas, não deuses oniscientes. Você me entende?''

Theo grudou na cadeira e não conseguiu pronunciar uma palavra.
Lembrou"-se do dia da prova final, quando sentiu"-se oco ao se deparar
com a principal pergunta da avaliação. Ele não havia estudado nada,
mas como iria arrumar tempo para estudar com tanta coisa para fazer?
A tentação foi forte demais e, mais uma vez, não resistiu. Fechou os
olhos e em poucos segundos estava na mente do professor responsável pela
prova, lendo o prontuário que ele estudou para criá"-la.

``Talvez você não tenha conseguido assimilar o verdadeiro significado de
ser médico, Theo. Sentimos orgulho de acertar diagnósticos e realizar
procedimentos, mas esses são apenas detalhes deslumbrantes da profissão.
Que prioridade você teria dado ao paciente desta prova? Que perguntas
teria feito àquele que entrou no consultório esperando falar de suas
dores e ser ouvido? Você acha que seria capaz de fazer seu paciente se
sentir acolhido e protegido? Na medicina, assim como na vida, caminhar
junto tem muito mais valor que apenas mostrar onde se deve chegar.''

Theo ficou vermelho e os velhos professores começaram a falar novamente.

``Quem trapaceia numa prova não pode ser chamado de filho dessa casa.''

``Por que não faz o sexto ano novamente?''

``Você não vai se formar.''

Alcântara finalmente reagiu, se levantou e pediu a palavra, para
surpresa de Theo.

``Senhores, este é de fato um episódio lamentável. Conheço, no entanto,
a retidão desse aluno. Não creio que privá"-lo de ser médico repare seus
pequenos desvios de conduta. Terá também pouco efeito submetê"-lo a um
ano estudando o que já estudou. Afinal, como podemos esperar um
resultado diferente se não mudarmos os coeficientes? Proponho algo
diferente.''

Theo se recostou ainda mais na cadeira e estranhou o tom de voz quase irônico de
seu mestre, receoso em saber onde esta conversa iria chegar.

``Vamos colocá"-lo em um laboratório vivo e dar a ele a oportunidade de
vivenciar a medicina na sua forma plena. Um lugar em que o cuidar seja
aplicado na totalidade da sua palavra. Conheço uma equipe de médicos que
atua em uma tribo indígena em plena floresta, onde ele não pode se
distrair com a vastidão de informações desse mundo -- neste momento
encarou Theo. -- Vamos mandá"-lo para a Amazônia.''

\asterisc

Theo chegou em casa chutando os móveis. Deu uma checada rápida em seu
pai, agora bem mais confortável com toda parafernália tecnológica que
dominava metade do novo apartamento.

Ainda estava irritado demais, e a pequena escultura que ganhou de seu
mestre, colocada no centro da prateleira de seu quarto, parecia zombar
dele. Como Alcântara teve coragem de fazer uma proposta como aquela e
nem se dar ao trabalho de procurá"-lo depois da reunião? Theo pegou a escultura e
colocou dentro de um baú de coisas velhas, decidido que, nem arrastado,
seria levado para o meio do mato.

\asterisc

Theo embarcou em São Paulo, fez uma conexão em Brasília, passou uma
noite em Manaus e só depois embarcou em um voo regional para pousar em
um pequeno aeroporto que surgia num clarão em meio à imensidão verde da
floresta, junto a um alargamento do rio onde as águas escuras dos
afluentes do rio Negro se tornam mais rápidas, tormentosas e cheias de
espuma. Finalmente São Gabriel da Cachoeira, a capital indígena do
Brasil.

Como foi acabar em uma situação como esta? Ainda estava tentando
entender aquele aeroporto tão simples quando foi surpreendido por um
funcionário do departamento federal que cuida da saúde indígena, que o
esperava junto com uma moça de cabelos castanhos escuros enrolados.

``Prazer, eu sou o Hernany, sou o coordenador desta área. Essa é a
Fernanda, dentista, que vai ser sua parceira.''

Fernanda nem tirou o fone de ouvido que usava, apenas acenou e, alguns
segundos depois, estava desligada da conversa novamente. Eles embarcaram
em uma caminhonete e em pouco tempo estavam em uma estrada esburacada.

``Faz tempo que você trabalha aqui, Hernany?''

``Já trabalhei muito em área indígena, mas cheguei aqui há pouco tempo,
eu nasci em Belém. E você, por que resolveu atender nessas bandas?''

``Eu vim para pegar um pouco mais de experiência.''

``Tá certo. Mas você veio porque quis ou te mandaram?''

``Eu vim porque eu quis'', disse, após gaguejar.

``Hm\ldots{} isso é bom. Eu trouxe a carga para vocês com os insumos
básicos, tem gasolina, comida, remédios, insumos médicos, vacina, tudo
que vocês precisam para ficar no mato.''

Hernany tinha experiência suficiente para saber que aquele escovinha da
cidade tinha caído de paraquedas ali e não tinha a menor ideia do que o
esperava.

``Vocês vão ficar alojados no polo base, que é como um posto de saúde,
tudo muito simples. Vão atender uma comunidade dentro de uma área de
reserva, mas fica tranquilo que a Fernanda é bem experiente, faz isso há
anos e vai poder te ajudar.''

``Mas e se tiver alguma emergência? Não dá para fazer tudo em posto de
saúde!''

``Lá a gente faz os procedimentos básicos, né? Mas quando é muito
urgente, a gente tenta remover o paciente pra cidade. Nem sempre dá,
depende da logística, mas é o que temos.''

``E isso acontece muito?''

``A gente faz o que pode para cuidar das pessoas'', disse Hernany,
claramente querendo acabar com a conversa.

Theo começou a ficar receoso, mas sua preocupação não durou muito tempo,
já que eles logo chegaram em um cais na beira do Rio Negro e
preocupações bem mais urgentes ocuparam sua mente. Como levar sua enorme
mala por uma estreita prancha de madeira de vários metros apoiada entre
as balsas? Se sentiu um perfeito idiota enquanto tentava vencer o
obstáculo e não cair no rio. Estava totalmente fora de contexto,
começando pela bagagem inapropriada e que parecia pesar centenas de
quilos. Precisou da ajuda do Capelão, o prático do barco que iria
levá"-los até o destino, que embarcou as bagagens e os insumos médicos
com uma facilidade irritante.

Quando o barco começou a rasgar as águas do extenso rio Negro, Theo
sentiu um frio na barriga e, conforme pegou velocidade, teve a sensação
de que estava decolando da superfície da água.

``É por isso que vocês chamam esse barquinho de voadeira?''

``É, o casco quase não encosta na água, assim diminui o atrito e a gente
vai mais rápido.''

O barulho do motor quase abafou a resposta de
Fernanda para Theo, que foi agarrando as mãos com firmeza em qualquer
fresta que estivesse ao seu alcance.

``Fica tranquilo e curte a natureza. Você está animado para chegar na
aldeia?''

``Muito. Faz esse barquinho dar meia volta e me levar para casa!'',
gritou, tentando fazer uma gracinha.

``Você só pode estar brincando. Eu estou fora há um mês e já não vejo a
hora de voltar para cuidar dos parentes.''

``Você é índia?''

``Não, mas fui praticamente adotada por uma família.''

O barulho do vento e o ronco do motor estavam impedindo uma
conversa mais elaborada, e Theo voltou sua atenção ao ambiente. Aos
poucos ele foi finalmente relaxando, sentindo o vento bater em seu
rosto. A altura da voadeira em relação à água lhe deu a sensação de que
tinha asas, como as araras que, aos pares, sobrevoavam o barco. Ele
foi apreciando a beleza da natureza e ficou extasiado conforme ia
percebendo a força do rio e a vastidão da floresta ao seu redor. Theo
viu ao longe a chuva, mas ela, curiosamente, não passou por eles, mas
deixou em seu rastro um belo arco íris. Toda essa beleza o foi
deixando bem mais tranquilo.

Ele achou que iria contemplar as margens durante todo o trajeto, mas
logo percebeu que estava enganado porque a paisagem era muito repetida,
como a vista da janela de um avião que encanta os passageiros novatos,
mas rapidamente se torna repetitiva e chata. Theo olhou para o lado e
não entendeu sua companheira de viagem se divertindo com a ideia de
voltar para floresta, sorrindo e olhando para tudo como se estivesse
vendo um interessante filme no confortável sofá de sua sala. Gente
doida.

``Encosta aqui um minutinho, Capelão.''

O prático atendeu ao pedido de
Fernanda e desligou o motor, parando em um dos bancos de areia que volta
e meia despontavam no meio do rio, baixo por causa do período da
estiagem. Com a parada, Theo percebeu o calor e a umidade do ambiente da
floresta tropical: foi só o vento do barco cessar que o suor começou a
brotar e mosquitos de todos os tamanhos começaram a lhe incomodar, como
se estivessem ali, parados naquele ponto da floresta, apenas esperando
alguém chegar para zunzunar.

``Olha só, que linda.''

Assim que a voadeira encalhou na margem, Fernanda tirou o chinelo e saiu
do barco molhando suas pernas brancas e a pontinha do vestido florido
que ficava acima do joelho. Subiu a encosta que demarcava o nível do rio
durante a cheia e andou até a árvore mais imponente e colossal, que
certamente ficava com sua base dentro do rio no período de alagamento.
Fernanda caminhou vagarosamente em volta dela, até que parou, olhou para
o céu, fechou os olhos e ficou um tempo em silêncio. Com o facão que
havia levado, fez uma leve marca em seu tronco revestido de musgos
verdes, a reverenciou lentamente e voltou para o barco.

``Eu sempre peço para os espíritos da floresta permitirem meu retorno
para a mata.''

Theo tentou esconder o riso e concluiu que Fernanda era realmente
maluca. Claro, quem mais viria para este fim de mundo espontaneamente?

Sem grandes surpresas, navegaram por horas nas curvas do grande rio,
algumas vezes sob o sol escaldante e outras debaixo de chuva. Finalmente
eles chegaram diante de uma pequena cachoeira e o prático encostou na
margem do rio.

``Podem descer.''

``Como é que é?''

``Vamos descer, Theo. Não dá para o barco subir essa cachoeira com o
nosso peso, ainda mais agora que o rio está tão raso. Se a hélice bater
em uma pedra e quebrar, ficamos presos aqui. O Capelão vai subir sozinho
com o barco vazio, nós vamos por esta trilha e encontramos com ele lá em
cima.''

Por mais que Theo tentasse, não conseguia subir o barranco com a mala de
rodinha, própria para rampas de aeroportos. Ele ia ridiculamente
vencendo uma parte do barranco para depois escorregar para baixo
novamente. Fernanda olhou pensativa para a grande mala de Theo, tentando
arranjar uma solução para aquele problema. Os insumos médicos tinham que
ir, mas\ldots{}

``Já sei! Temos uma mochila vazia sobrando, separa só o que você precisa
e coloca nela. O resto deixamos aqui e pegamos na volta.''

Theo achou que aquilo era uma piada de mau gosto. Ele precisava de tudo
o que estava levando, afinal ficaria meses isolado em uma tribo no meio
do nada. Além do quê, ele não deixaria suas coisas ali no meio do nada.

``Vamos, Theo! Afinal, o que você trouxe que pesa tanto? Escolhe só o
essencial, não dá pra enrolar mais, ou você quer percorrer o rio no
escuro? Não podemos chegar depois do anoitecer!''

Ele separou algumas peças de roupa e objetos de uso pessoal e não
conseguiu deixar de pegar as guloseimas e as barras de chocolate que
estava levando, mesmo derretidas sob aquele sol quente.

"Não se preocupe, ninguém vai tocar nas suas coisas. Quando voltarmos,
tudo vai estar no exato lugar onde você deixou."

Incomodado, mas sem opção, Theo cobriu a sua mala com uma lona de
plástico e deixou em um canto mais protegido da prainha de areia.

Subiram a ribanceira pela lateral da praia e se encontraram com Capelão do outro lado das pedras.
Exausto e muito incomodado, Theo se jogou no barco. Ele não queria conversa, mas
como se sentir sozinho em um barco de seis metros e com mais duas
pessoas?

``Quanto falta para chegar?''

``Só mais meio dia de viagem. Olha, os botos vão viajar com a gente de
novo!''

Theo olhou para a água e não conseguiu ver nada. Tentou se recostar na
sua nova e ridícula mochila enquanto esperava o tempo passar. O barulho
do motor foi deixando"-o sonolento e o balanço da embarcação na água o
acalentava como numa cantiga de ninar. Logo aprofundou o pensamento e já
não pensava mais na viagem, mas em como sua vida tinha lhe guiado até
ali. Apesar de todos os poderes que a navegação lhe conferiu, ele se
sentia pequeno e frágil, com um vazio que aumentava à medida que
percebia um destino caprichoso, que não o deixava comandar a própria
vida. Estes pensamentos, ao mesmo tempo que eram assustadores, faziam
com que se sentisse revigorado como há muito tempo não se sentia.
Quantas surpresas a vida teria deixado no caminho à sua frente?
Surpreendido por uma faísca de dúvida, ele sentiu medo.

Foi surpreendido com a água que espirrou em seu rosto depois que a
voadeira fez uma curva repentina, se secou e deitou a cabeça no banco
tentando se manter acordado. Suas pálpebras pesavam cada vez mais e ele
teve a ideia de navegar um pouco no mare nostrum. O processo de
concentração para mergulhar em seu mar interior era bem simples agora;
sem esforço algum, fechou os olhos e se concentrou em sua pulsação. Foi
tomado pelo som repetitivo de sístoles e diástoles, válvulas se abrindo
e fechando sob uma penumbra, até que se viu no seu lugar preferido, seu
igarapé, para onde sempre ia quando queria repousar. Ficou ali
observando aquela paisagem que lhe dava paz, com uma floresta e um
espelho d'água perfeitos, até que foi procurar por sua amiga Cristine.

``Você gosta mesmo daqui, não é?''

Theo sempre a encontrava sentada na ponte da Baía de Sydney,
contemplando a paisagem.

``Muito. Não sei como demorei tanto tempo para encontrar esse lugar.''

Logo após ter apresentado o mare nostrum para Theo e ajudá"-lo a
descobrir que Alcântara era seu mestre, há pouco mais de dois anos,
Cristine resolveu viajar pelo mundo. Estava cansada de ver a vida
pelos sonhos dos outros e resolveu ter suas próprias experiências.

``Quando tentamos controlar tudo, perdemos a chance do inusitado, do
inesperado. Eu adoro as surpresas da realidade!'', costumava dizer.

Em suas viagens, passou pela Austrália e gostou tanto do lugar que
acabou ficando por lá. Desde então, Theo não a encontrava mais na vida
real, mas eles frequentemente se conectavam no mundo virtual.

``Como estão as coisas na Amazônia, Theo?''

Ele começou a reclamar da viagem, mas um chiado alto, como de um rádio
que não sintoniza em nada, interrompeu a conversa e a imagem começou a
ficar turva.

``Cristine, está ruim!''

Ela não conseguia ouvi"-lo, até que tudo foi ficando silencioso novamente.

``Nossa, que horrível. Espero que não aconteça de novo.''

``O Alcântara me alertou sobre isso. A Amazônia é muito isolada, fica
difícil se conectar.''

``Que bom, quem sabe assim você descansa um pouco. Como está indo a
viagem?''

``Você tá louca? É tudo horrível, Cristine. Estamos viajando há horas e
ainda temos quase um dia inteiro de rio e mato pela frente. O Alcântara
só podia estar louco de me mandar para um lugar como esse.''

Ela deu uma gargalhada e Theo se irritou.

``Ué, você não quis trapacear na prova, Theo? Deveria ter pelo menos
colado direito. Para que se arriscar desse jeito?''

``Pode parar, Cristine, já ouvi tantos sermões que\ldots{}''

Antes de Theo terminar sua frase, o chiado novamente tomou conta do
ambiente, dessa vez mais alto, e a imagem foi ficando mais turva que da
primeira vez. Quando ele menos esperava, se viu de volta ao barco com o
capelão e a Fernanda, que agora dormia estranhamente, com um dos olhos
meio aberto.

\asterisc

``Sua navegação pode não funcionar bem na Amazônia, Theo.''

O monte Roraima estava com uma névoa branca tão espessa que os impedia
de ver as árvores e os inúmeros rios que cortavam a paisagem abaixo, e
que garantiam um espetáculo à parte nos dias claros. Theo se cansou
daquela cena e fez um movimento com a mão direita, limpando as nuvens,
franqueando para ambos a sua vista preferida. Ele conhecera aquele
planalto em uma reportagem, e ficou encantado com seus quase três mil
metros de altura e a estonteante paisagem em seu entorno. Desde então,
aquele se tornou o lugar oficial de seu treinamento com Alcântara.

``Mestre, não se preocupe, eu já sou um navegador experiente. Posso
superar qualquer dificuldade de navegação que aparecer, até mesmo na
Amazônia.''

``Não estamos falando da sua capacidade, Theo, mas de como as coisas
funcionam no mare nostrum. Para navegar e ter acesso a essa grande
biblioteca de conhecimento da humanidade, precisamos de uma conexão entre
as mentes. Cada mente é como um servidor que vai nos interligando,
sucessivamente, uns aos outros. Como a Amazônia é muito esparsamente
povoada, a conexão ocorre em bolsões, por isso você vai ter dificuldades
de navegar de maneira ampla e talvez se restrinja apenas a um grupo
geograficamente delimitado.''

``Mas como é possível sobreviver sem o mare nostrum? Sabemos que ele é
importantíssimo para a sanidade mental das pessoas.''

``Fica tranquilo, você vai entender uma porção de coisas assim que
estiver verdadeiramente conectado com aquele mundo. Você está muito
dependente do mare nostrum, Theo, um choque de realidade vai te fazer
muito bem.''

Theo estava com uma expressão preocupada e Alcântara achou melhor
adiantar algumas informações.

``Preste atenção. Existem outras formas de navegar na Amazônia,
mas elas são um pouco diferentes das que você conhece.''

``Como assim?''

``Existem os pequenos oásis de informações.''

``Como as gaiolas de Faraday?''

``Também, mas não é disso que eu estou falando. Veja bem, Theo, as
gaiolas de Faraday são represas no mare nostrum, blindagens
eletrostáticas construídas artificialmente que fazem com que as pessoas
que estão dentro delas consigam se isolar e conectar suas mentes apenas
entre si, para treinar, aprender e ensinar, sem risco aos alunos. Como
um tubo de ensaio, certo?''

``Certo.''

``Isso vai acontecer com você na Amazônia também, mas sem precisar
construir uma gaiola de Faraday, porque você vai estar em um local com
uma blindagem natural: as longas distâncias e a baixa densidade
populacional. Portanto, na Amazônia você só vai conseguir se conectar às
mentes próximas. Você vai estar em um tipo de `lago', isolado do mar
principal.''

``Não vejo como isso pode ser útil para mim, mestre. As sociedades na
Amazônia são arcaicas, o senhor sabe disso. O que eu posso aprender se
puder me conectar apenas a este tipo de consciência?''

Alcântara se lembrou da primeira vez em que viajou para a região e deu
um leve sorriso.

``Isso é o que vamos ver, Theo.''

O garoto não deu atenção ao que Alcântara falou e o professor se
perguntou se já estava na hora de dar ao jovem mais informações sobre o
tipo de conexão que ele iria encontrar, até que percebeu que não devia
facilitar as coisas para ele e resolveu continuar o treinamento padrão.
Theo assimilava tudo tão rápido que seu mestre sabia que um pouco de
sofrimento não lhe seria mau.

\asterisc

``Eu não ligo de trabalhar lá, não, Zé. Tô tão quebrado que faço
qualquer coisa, desde que me paguem bem. Lá eles pagam, então eu tô
feliz. Mas que é esquisito, isso é\ldots{}''

Fred virou o copo de café enquanto seu ouvinte se servia de uma garrafa
de cerveja gelada na mesa torta, espremida em um canto do botequim.

``Tem certeza que você não quer beber uma cerveja, Fred? Descansa um
pouco, homem, hoje é sexta. Já trabalhou a semana inteira, precisa
relaxar.''

``Não posso, estou com turno dobrado. Não estou te falando que arrumei
pra noite esse bico de faxineiro? Se eu beber, vou capotar. Cara, eu tô
muito cansado, mas tá valendo a pena. Eu vou acertar minhas dívidas e
daqui um tempo vou até pagar mais pensão pras meninas, você vai ver'' --
Fred deu um bocejo longo e virou três colheres de açúcar no segundo copo
americano de café que chegou.

``E como que é trabalhar numa \versal{UTI}, irmão? Eu não ia aguentar ver todos
os dias a morte tão de perto.''

``Tem morte nenhuma lá, é bem tranquilo, ninguém me incomoda. Eu chego,
limpo tudo, fico fazendo uma hora e vou embora. Às vezes até consigo
dormir pelos cantos, ninguém percebe.''

``Toma cuidado, Fred, você pode ser descoberto e perder o emprego.''

``Da primeira vez eu até tive medo, mas agora já acostumei. Não tem
perigo, não. Lá é bem tranquilo, arrumei um armário que é perfeito para
tirar uma soneca sem ser visto.''

``Pô, quem dera eu tivesse um trampo assim.''

``É, eu tive sorte. Mas lá é um pouco estranho, sabe? Para entrar
passamos por um lugar, com uma porta de cada lado, como se fosse uma
passagem com um ímã, sabe? Eles avisam pra gente tirar tudo que é metal,
senão sai voando. Perguntaram várias vezes se tem ferro no corpo ou
coisas assim, chega a dar uma angústia na gente. Aí lá dentro é uma \versal{UTI},
mas não é um hospital, e o estranho é que tem um monte de equipamento só
pra dois pacientes. Devem ter sido pessoas importantes, mesmo.''

``Devem ter sido? Você fala como se já estivessem mortos.''

``É como se estivessem, ninguém lá está esperando pela melhora deles, eu
percebi isso.''

``Então por que ficar na \versal{UTI}? Coisa estranha'', o homem virou mais um
copo de cerveja.

``Você nem imagina. Deve ser por isso que eles pagam tão bem os
funcionários.''

\asterisc

Era desesperador. Tudo continuava exatamente igual, o barulho do motor
do barco, a paisagem que se sucedia a cada curva do rio e até as pessoas
que, acostumadas com a viagem de voadeira, permaneciam imóveis e
quietas. Fernanda dormia desde que a viagem começou, enquanto Capelão
parecia uma estátua, concentrado nas pedras e nos canais do rio. Theo
não conseguia ficar ali curtindo aquele barco nem por mais um minuto.
Tentou então, mais uma vez, voltar ao mare nostrum. Fechou os olhos e foi
tentar se encontrar com Cristine.

``Que bom que você conseguiu voltar.''

``Você sabia que isso poderia acontecer?''

``Claro. Não tenho a força das suas ondas cerebrais, Theo, mas conheço o
mare nostrum há muito mais tempo que você. Eu também tive dificuldade
para me conectar quando estive na Noruega, perto do Polo Norte.''

Theo estava profundamente irritado, se sentindo injustiçado, e Cristine,
percebendo seu mau humor, começou a mudar os cenários para tentar
animá"-lo um pouco. Lembrando da sua viagem à Noruega, ela tentou mostrar
a Theo as belezas de Svalbard, a ilha do urso polar. Sua expressão
emburrada, no entanto, continuava a mesma.

``Theo, o que está acontecendo com você? Nunca te vi assim.''

``Você também, Cristine? Já chega o Alcântara fazendo perguntas e
tentando se meter em tudo o que eu faço.''

``Só estamos tentando te ajudar. E ele fez muito bem de te mandar para a
Amazônia. Por que não navegamos um pouco nas suas memórias? Podemos
tentar\ldots{}''

Theo ficou louco só de pensar em Cristine fuxicando sua mente e
imediatamente materializou uma redoma de vidro, ficando aliviado quando
a viu gesticulando com as mãos do outro lado da vitrine, não ouvindo
nada do que ela falava; com todas as informações do mundo a sua
disposição, ele não precisava dela. Começou a se aprofundar no mare
nostrum buscando algo interessante para aguentar a viagem, mas ouviu
novamente um chiado alto, cada vez mais intenso, e então perdeu a conexão
e se viu novamente no barco, com Fernanda e Capelão.

``Eu preciso navegar, o que vou ficar fazendo aqui?'', ele se levantou
irritado.

O prático o olhou e deu uma risada, achando"-o completamente louco. Theo
tentou uma, duas, três vezes, mas sempre que tentava avançar no oceano
das mentes humanas era levado de volta ao barco, onde apenas contava com
as nada inusitadas presenças de Fernanda e Capelão.

``Você não cansa disso aqui?''

``Não'', respondeu o prático, voltando a se concentrar na direção da
embarcação.

Capelão era monossilábico. Não conseguindo desenvolver uma longa
conversa com ele, Theo tentou arranjar outras formas de se distrair na
voadeira. Olhou para o rosto de Fernanda sonhando e se deu conta de que
ali poderia navegar como seu mestre lhe havia explicado: os três estavam
isolados do restante do mundo, mas, próximos uns dos outros, poderiam
conectar entre si.

Primeiro espiaria os sonhos de Fernanda, sempre tão alegre e risonha.
Queria descobrir o que a deixava daquele jeito. Fechou os olhos, se
concentrou e logo direcionou sua mente para a dentista. Se viu em um
barco ensolarado, onde um leve som de flauta tocava ao fundo. Estivera
naquele mesmo local instantes atrás, mas o cenário que Fernanda
enxergava era completamente diferente do dele. Para seu espanto, um
adolescente todo engomadinho admirava seu reflexo na água do rio,
como aquelas representações de Narciso apaixonado pela própria
imagem. Theo percebeu que o garoto era ele e aquilo foi como um tapa em
sua cara. A dentista o via como alguém infantil, e ainda por cima mimado
e vaidoso.

Aborrecido, logo saiu dali e foi para a mente de Capelão. A simplicidade
dos pensamentos do prático o surpreenderam, era como se ele fosse a
personificação da natureza. Sua vida com sua família e amigos
interessava bem pouco a Theo, e por mais que tentasse entrar em uma cena
diferente, só conseguia ver aqueles mesmos elementos: floresta, natureza
e água, muita água, como se fossem curvas do mesmo rio.

``Você não cansa de viver como se todos os dias fossem iguais?''

``O que mais eu quero ver nessa vida? Tudo que eu preciso está aqui, bem
perto de mim.''

Theo era imaturo demais para entender o que o prático quis dizer com
aquela frase. Entediado com aquilo tudo, foi embora, e, mesmo sabendo que
não conseguiria navegar em sonhos mais profundos, teimou e avançou
novamente para aprofundar seu transe. Não foi com surpresa que ouviu
mais uma vez um chiado cada vez mais alto, insuportável. Voltou ao
barco, condenado a viver a realidade.

\asterisc

O armário de vassouras no corredor da \versal{UTI} não era exatamente o local
mais apropriado para se dormir, mas Fred precisava descansar sem ser
visto e se sentia seguro ali. Programou o despertador em seu celular
para vibrar em uma hora, fez uma trouxinha com o velho casaco e
colocou sob o pescoço. Andava tão cansado por conta dos turnos
nos dois empregos que não era preciso muito conforto para que
conseguisse cair no sono. Fechou os olhos e adormeceu em poucos
segundos.

Curiosamente, sempre sonhava com o mesmo lugar quando dormia ali: uma
grande casa de praia, toda branca e envidraçada, onde nunca teve coragem
de entrar, mas, por alguma razão que desconhecia, tinha certeza de que
havia alguém ali dentro. Dessa vez, no entanto, Fred estava disposto a
enfrentar seu medo e, com muito cuidado, caminhou até a grande
sala de estar daquela bela casa. Achou o local muito chique, diferente
de tudo o que já havia visto, e se impressionou por conseguir ver a praia
e o mar de todos os cantos da sala. Ele ficou observando as
ondas batendo na areia quando se deu conta de que, ao fundo, bem
baixinho, um leve som de piano tocava. Ele nunca ouvira uma música tão
bonita, e resolveu descobrir de onde ela vinha. Caminhou pela casa em
busca do som até que encontrou uma biblioteca, com a porta entreaberta.
Deitado sobre um divã de couro, viu um homem com os olhos fechados,
respirando fundo, como se estivesse relaxando ao som da música.

Fred resolveu ficar ali, parado, porque estava gostando da música, até
que o homem deitado no divã lentamente abriu os olhos e viu o faxineiro
parado na porta do escritório, usando seu uniforme de trabalho. Levantou
em um impulso.

``Finalmente resolveram mandar alguém. Você é da companhia de limpeza,
não é? Estamos precisando muito de sua ajuda, essa casa está sempre
cheia de pó.''

``Sim, eu sou. Mas senhor, que música é esta?''

Fred procurou um
controle remoto, um aparelho de rádio ou qualquer equipamento que
estivesse controlando o som, mas tudo o que viu foram caixas embutidas
no teto.

``Eu não sei.''

Marc ficou envergonhado por não saber o que estava tocando em
sua própria casa, e Fred estranhou aquele comportamento.

``Venha, eu vou te mostrar os cômodos.''

Ele levou Fred diretamente para um quarto de bebê, e eles pararam em
frente a uma porta na qual um par de sapatinhos azuis estava pendurado.

``Esse é o quarto mais importante, preciso que fique muito limpo. É do
meu filho.''

``Que massa, doutor. Eu também tenho duas filhas. É muito bom, eles
crescem rápido, né?''

Marc ficou mais uma vez sem graça com a resposta que ia dar.

``Ele ainda não nasceu, mas estamos esperando por ele.''

``Entendi. Bom, mas o senhor deve estar felizão de ser um moleque, né?
Aposto que vai torcer pro time do senhor.''

Marc se mostrava confuso e Fred ficou se perguntando se havia falado
alguma coisa de errado.

``É\ldots{} na verdade nós ainda não sabemos o sexo do bebê.''

``Bom\ldots{}'', Fred falou coçando a cabeça e olhando para a decoração do
quarto. ``Hoje em dia não existe mais isto, menina também pode gostar de
azul, né?''

Marc estava muito incomodado com a conversa e achou melhor encerrar
abrindo a porta do quarto com cuidado e mostrando um quarto
cuidadosamente decorado, iluminado pelo céu nublado. Entrando no cômodo, Fred se
aproximou de um criado mudo ao lado do berço, e viu sobre o móvel, entre
miniaturas de heróis e vários brinquedos com motivos masculinos, um
vasinho com um grão de feijão que parecia ter sido recentemente
plantado.

``Posso ver?''

``Tenha cuidado.''

``Foi você que plantou?''

``Fui. Tentei fazer vários desse quando era criança, mas eles nunca
chegavam a brotar. Esse não vai falhar.''

``Parece que dessa vez você acertou. Já tem um brotinho saindo do
grão.''

Fred se lembrou de quando havia ajudado suas filhas a fazerem o trabalho
com o feijão, a pedido da escola. Teve de pedir para os vizinhos algodão
emprestado, já que o dinheiro mal dava para comprar o grão que era
colocado na mesa. Naquela época ele ainda tinha contato com as filhas,
mas depois do divórcio a mãe as levou para longe e ele sentia muita
falta das meninas. Só que agora, ao ver aquele feijãozinho brotando na
casa de Marc, ficou feliz por ter um trabalho novo, conseguir juntar
algum dinheiro para visitá"-las e até poder levar um presente.

``Você consegue limpar o quarto em meia hora? Seria melhor você terminar
antes da minha esposa chegar, ela está caminhando na praia e deve voltar
a qualquer momento.''

``Consigo sim, senhor.''

Fred achou estranho aquele pedido, o quarto estava muito limpo, sem uma
partícula de pó. Achando que o trabalho poderia lhe render um bom
dinheiro, ele não fez nenhuma pergunta e atendeu ao pedido de Marc.
Quando terminou, foi procurá"-lo na sala, e o encontrou atirando dardos em
um grande alvo de madeira pendurado na parede.

``O quarto está pronto.''

``Ótimo, semana que vem você continua. Já pode ir agora.''

Fred ficou parado meio sem jeito, como se estivesse esperando por alguma
coisa, e Marc não entendeu o que ele queria.

``Obrigado, te espero semana que vem.''

Ele continuava ali, em pé, pensando em como falaria sobre o pagamento,
mas Marc ficou olhando"-o ainda sem entender. Então Fred resolveu
explicar.

``É, sabe o que é, doutor? A gente não combinou nada, né, então pode me
pagar qualquer valor.''

``Pagar? Eu não tenho dinheiro.''

``Como assim o senhor não tem dinheiro?''

``Não sei, eu não tenho. Eu não sei onde está.''

Marc ficou confuso com aquela pergunta e, descontrolado,
começou atirar os dardos furiosamente contra a parede.

``Eu não sei, eu não tenho controle nenhum do que se passa por aqui.''

Fred ficou sinceramente assustado com a reação e resolveu sair dali
antes que as coisas se agravassem.

Passou disfarçadamente pela porta e, logo que tirou os pés da casa,
estava na praia. Saiu correndo, mas não sem antes olhar para trás,
admirar a casa e pensar como o dono de um lugar tão bonito poderia ser
tão confuso e infeliz, já que ele, só de se imaginar morando com suas
filhas em um lugar como aquele, se sentia realizado. Caminhou com pressa
pela praia, até que chegou longe o suficiente para se sentir seguro.
Então respirou fundo, sentiu uma leve brisa soprando. Fazia tanto tempo
que ele não tirava férias que tinha até se esquecido da sensação de
pisar na areia fria.

Foi se afastando da casa, até que sua visão foi totalmente coberta por
uma névoa branca. Ele não enxergava nada e, totalmente perdido, tentava
caminhar, mas não sentia nada sob os pés.

Fred então ouviu um celular vibrando, cada vez mais forte, e a sensação
sobre o seu corpo mudou. Abriu os olhos e se deu conta de que estava no
apertado armário, ao lado de rodos, vassouras e panos de chão. Colocou
as mãos no bolso para desligar o celular e percebeu que já estava na
hora de ir. Saiu do armário, pegou seu kit de trabalho e foi para a \versal{UTI}
terminar o serviço daquele dia. Ao entrar na grande sala iluminada pelas
frias lâmpadas brancas, viu um grupo de médicos trabalhando em cima do
homem internado, que parecia agitado, um deles se perguntando o que o
teria deixado tão nervoso. Desde que começara a trabalhar ali, Fred
nunca sentiu qualquer empatia por aqueles pacientes, mas, dessa vez, ao
olhar para o rosto do homem, sentiu muita pena, como se o conhecesse.
Mesmo sob o stress daquele momento, não pode evitar de reconhecer
uma leve música que tocava no ambiente, um bonito som de piano que saia
do mp3 player de um dos enfermeiros.

\asterisc

%\begin{center}
%\emph{Theo chega na aldeia}
%\end{center}

``Chegamos.''

O barco foi fazendo um longo arco e desacelerando até que Capelão
encostou no pequeno píer de madeira. Nesta época do ano, durante o
período da seca, havia uma grande ribanceira de terra a ser vencida
antes de chegar a uma pequena trilha. Theo estava exausto da viagem e
odiou aquele lugar quente, onde o mato ia roçando em suas pernas
enquanto caminhava. Já estava quase de noite e ele estava esperando
encontrar uma vila com pelo menos um comércio e uma praça, mas, chegando
ao local onde os índios moravam, viu uma área descampada com apenas
diversas palhoças, uma ao lado da outra, dispostas em um grande círculo.
No centro, uma enorme casa de palha se destacava. Impressionado com a
complexidade da construção, Theo examinava"-a com atenção, imaginando
como aqueles índios teriam construído aquela estrutura sem os avançados
conhecimentos de engenharia dos homens brancos. Fernanda percebeu seu
interesse e começou a falar.

``Incrível, não é? Essa é a casa do guerreiro.''

``Para que serve?''

``É como um centro comunitário, eles usam para rituais e também para
tomarem decisões da aldeia.''

Pararam um momento contemplando a construção.

"É feita inteira sem um prego ou arame, tudo com marcenaria artesanal."

Theo fez menção de entrar para conhecer o interior mas os indígenas ao
redor começaram a falar alto e gesticular.

Fernanda explicou:

``Você não pode entrar, só quando for convidado.''

Incomodado, mas sem perder a compostura, Theo deu ombros e continuou o
caminho. Logo se dirigiram até uma outra trilha, mais larga, caminhando
até um local que parecia outra ala da comunidade, menos tradicional, com
pequenas casas de alvenaria, telhados de metal e um padrão quadrado na
disposição das construções.

Ao lado de um campinho de futebol, Theo avistou uma das poucas
construções grandes deste espaço. A simplicidade do local contrastava
com a pompa de uma placa de bronze, que marcava a inauguração da
construção e levava o nome ``Centro Médico de Atendimento Indígena Dr.
Ricardo Affonso Ferreira''. Aquela estrutura tinha sido claramente
concebida por algum arquiteto que nunca nem mesmo visitara o local, tão
inadequada que era.

Era como se a floresta estivesse tomando de volta o espaço ocupado pelo
prédio, das rachaduras em cada parede, em que brotavam pequenas plantas -- como
um jardim vertical, em voga nas grandes
cidades --, até formigueiros e ninhos dos mais diversos insetos em cada viga do
telhado. E pó, muito pó por todos os lados.

``Aqui é o polo base, Theo, nossa casa pelos próximos meses.''

Theo ainda estava boquiaberto imaginando como viveria ali quando eles
entraram em um cômodo que supostamente seria o quarto. Ficou
procurando a cama, até que Fernanda pegou uma rede em um baú
que estava fechado no canto do cômodo e entregou a ele.

``Essa é sua. Pode pendurar em qualquer lugar.''

Theo olhou para o pano e não acreditou que teria de dormir ali.

``Onde fica o banheiro?''

``Ali atrás, é só dar a volta.''

Ele separou suas coisas em cima do baú tentando organizar as ideias.
Queria tomar banho para afastar a sensação de sufocamento que o calor
estava causando. Não teve dificuldade de encontrar o pequeno banheiro de
alvenaria e azulejos. Organizou suas coisas no nicho reservado para o
banho e pressentiu o frescor da água fria caindo em seu rosto. Porém,
após tentar abrir o registro, percebeu que aquele cubículo não via água há
séculos, pelo menos se levasse em conta a quantidade de teias de aranha
que havia.

Achou estranho a naturalidade com que Fernanda abriu a porta e entrou no
banheiro, não dando a menor confiança para sua quase nudez; a cueca era
a única coisa que o separava dos selvagens ali fora.

``Para o chuveiro funcionar é preciso que a bomba esteja ligada'', explicou,
quando percebeu a intenção de Theo.

``Quando ligam?''

``Como assim ligam? Você pode ligar a hora que quiser, mas não acho que
temos gasolina suficiente para isso. Em geral tomamos banho no rio.''

Theo desistiu e voltou para o quarto, exausto e ainda sujo de suor. Ele
só queria descansar e relaxar, mas tudo o que encontrou foi o cômodo
abafado, com uma rede que ele não fazia a menor ideia de como pendurar.

Vasculhou o quarto à procura de ganchos, mas não encontrou. Olhando com
atenção, percebeu alguns pedaços de corda que serviriam para atar a rede
às vigas do teto. Calculou a posição para que a rede ficasse perto da
janela, se achando muito esperto por ocupar o lugar mais ventilado.

Não seria difícil instalar sua rede. Depois de várias tentativas, e
completamente molhado de suor, conseguiu passar as cordas por sobre uma
tora que sustentava o telhado, mas os pontos ficaram muito próximos, e
depois de amarrada a rede ficou muito fechada. Refez todo o processo e
acabou colocando os pontos em uma posição tal que a corda era curta
demais, e a rede ficou muito alta.

Estava a beira do desespero quando chegou Fernanda, toda suja de terra.

``O que aconteceu, Fernanda?''

``Matando a saudade dos meus cachorros.''

``Tá certo\ldots{} Você pode me ajudar com essa rede?''

Fernanda percebeu a dificuldade de Theo.

``É difícil mesmo, mas com o tempo você pega o jeito.''

Com muita facilidade, Fernanda deu um nó na ponta da corda, a girou
no ar e lançou"-a em direção da viga de madeira, com muito mais precisão
do que ele, antes de colocar o nó. Nesta hora, Theo percebeu
que ela não depilava o sovaco. Ficou tão focado nisto que não
percebeu os outros macetes que a dentista usou para instalar a rede
de forma super eficiente e sem esforço. Quando o médico percebeu, a rede
estava pronta, na altura e na posição perfeita para Theo dormir e na
posição mais ventilada

``Coloquei virada para cá para poder receber o vento de lado, é mais
gostoso. Agora o bebê já pode tirar sua naninha'', disse, rindo, e se
retirou.

Apesar de estar cedo, o dia tinha sido demais para ele e, mesmo sem
banho, deu um sorriso amarelo e se deitou. Diferente do que ele supunha,
a rede de algodão era fresca e confortável. Imitou a posição dos
índios quando usavam a rede e se sentiu muito bem.

Apesar de sentir os olhos pesados e sua mente estar implorando por uma
noite de sono, Theo ainda se focou para uma última navegada. Ele tinha
de tentar, afinal essa prática já havia se tornado um ritual para ele
antes de dormir, quase um vício, como as pessoas que
não conseguem dormir sem antes pegar seu celular e dar uma olhadinha em
alguma rede social para não se sentir só. Além de companhia, o mare
nostrum lhe oferecia a oportunidade de passar seu dia a limpo e rever os
``melhores momentos''.

Fechou os olhos e se concentrou nos seus batimentos cardíacos, respirou
fundo, relaxou e mergulhou em sua pulsação até se ver em seu igarapé.
Então logo tentou avançar, estava ansioso para encontrar seu mestre,
convencê"-lo a desistir da ideia da Amazônia e deixá"-lo voltar para São
Paulo, para sua vida na faculdade, para sua casa.

Ele começou a ouvir mais chiados e, desistindo de procurá"-lo, se
concentrou em achar alguém que conhecesse a vida na Amazônia e lhe
ensinasse desde o idioma dos indígenas até as mínimas coisas que se
precisaria para sobreviver naquele mundo, como colocar uma rede, por
exemplo. Tentou se lembrar qual personagem da história poderia ajudá"-lo
e quais memórias o ensinariam a se virar na selva, como dormir em uma
rede e até mesmo construir um sistema que lhe proporcionasse um banho
quente. Então se lembrou dos livros que lera dos irmãos Villas Bôas,
contando suas expedições pelo Parque Indígena do Xingu.

Fechou os olhos, imaginou as cenas que lera nos livros, mas quanto mais
se esforçava para se afastar de seu igarapé e avançar na navegação, mais
chiados ouvia. Theo ia sendo levado por imagens desconexas, sem chegar
em sonho algum. Se viu pequeno, brincando com seu pai, e em meio a
ruídos foi parar na cena em que Alcântara o havia presenteado com o
Muiraquitã, o pequeno sapo esculpido que fora de sua mãe. Perdido, Theo
tentava navegar e controlar os sonhos nos quais entrava, mas em vão.
Tudo o que ele conseguia fazer era acessar sua própria memória.

Sem perceber, sentiu saudades de coisas simples como o conforto da
presença do pai. ``Se vira!'', é o que ele diria se pudesse falar com
seu filho.

Essa lembrança fez Theo perceber que estava sendo mimado, fraco, e que
não poderia continuar daquela forma. Ele queria dar orgulho para seu pai
e sabia que, se ele estivesse ali, ficaria decepcionado ao ver o filho
reagindo daquela forma naquela situação.

Theo então percebeu que, dessa vez, teria de se virar sozinho, sem o
mare nostrum.

\asterisc

%\emph{Marenostrum II -- Marc conta sobre a visita de Fred para Teresa}

``Limpa bem os pés antes de entrar.''

Teresa atendeu ao pedido de Marc, sacudiu os pés e entrou com cuidado na
casa.

``Eu fiz uma surpresa para você, Teresa, venha ver.''

Marc pegou em sua mão e subiu as escadas com ela correndo, até que parou
na frente da porta do quarto do bebê, que estava fechada. Ficou parado
por alguns segundos para fazer suspense e então abriu, esperando ansioso
pela reação da esposa. Ela não esboçou nada, se manteve apática,
como era constante nos últimos tempos.

``Que lindo, querido'', disse, sem convencer.

``Mandaram um funcionário para nos ajudar. Eu sabia que esse trabalho
valeria a pena, Teresa. Imagina só nosso filho nascendo cheio de
regalias, um mordomo só para cuidar da nossa casa. Aposto que é só o
começo, daqui a pouco ele vai nos mandar até um motorista.''

Eles entraram no quarto e Marc começou a mostrar o trabalho que havia
sido realizado. Passava os dedos nos móveis, sorrindo, mas ele mesmo se
cansou daquilo e de repente caiu em si, até que sentiu muita raiva. Quem
ele estava querendo enganar? O quarto estava tão limpo quanto sempre
esteve. Porém, mesmo furioso, ele não deu o braço a torcer e levou a
história até o fim.

``Não me olha com essa cara, Teresa. Eu sei que você é uma esposa
dedicada e quer cuidar de tudo sozinha, mas agora eu sou um cara
importante e eu tenho pessoas para me ajudar.''

Marc começou a abrir as cortinas do quarto e mostrar os vidros
imaculados de limpos, tentando com isto encontrar alguma reação em sua
companheira.

``Tudo bem, Teresa, eu já entendi por que você está preocupada. Relaxa,
foram eles que nos mandaram o funcionário. E olha só, com a nossa casa
desse jeito podemos até receber visitas importantes, podemos preparar
jantares para a chefia.''

Teresa continuava sem falar nada e Marc foi ficando mais nervoso.

``Pare de me questionar, Teresa. Eu não posso ficar nessa mesma vida de
sempre, eu preciso crescer e esse foi só um primeiro passo.''

Ele então começou a jogar os objetos do trocador no chão, bagunçando
todo o quarto.

``Está vendo, Teresa? Isso é para você ver que não sou um fracassado,
pra você ver o quanto eu virei uma pessoa importante. Podemos fazer o
que quiser porque eles vão vir arrumar. E você, minha mulher, não
precisa mais limpar o chão nem arrumar nada.''

Ela continuava serena, passiva, observando\ldots{} uma lágrima
escorria do seu rosto.

\asterisc

%\emph{Marenostrum II -- 9 - Senhor D manda Jonas intensificar a sedação do Marc}

``Não me venha com desculpas, Jonas. Eu quero a nossa arma funcionando,
quero ver todos aqueles terroristas mortos e quero agora. Nós não
podemos nos dar ao luxo de deixar pontos fracos.''

``Mas senhor, ele não é uma máquina. E mesmo se fosse, até uma
metralhadora tem de parar por um momento para esfriar.''

Jonas estremeceu quando Senhor \versal{D.} tirou os óculos e olhou para ele. Seu
chefe, que estivera estudando um mapa com rabiscos em cima de sua mesa,
voltou a esbravejar.

``Estou cansado dessas suas justificativas. Nós investimos milhões para
criar essa arma e agora ela está pronta para atirar. Por que não está
atirando? Está na hora de você afrouxar a coleira do nosso cachorro, ele
precisa mostrar que ainda serve para alguma coisa.''

``Mas existem particularidades na cabeça do ser"-humano, senhor, não é
assim tão fácil quanto\ldots{}''

``Tem de ser. A imperfeição humana é um luxo que não podemos nos dar. Eu
quero precisão e eficiência. Nem que para isso você tenha de tirar toda
a humanidade dele.''

``Mas senhor, é justamente a humanidade dele que nos permite
controlá"-lo. Além de se considerar um fracassado, o acidente nos deu um
ponto de apoio para dominar sua vontade: ele se sente culpado por ter
prejudicado a mulher que ama -- grávida, ainda por cima. A culpa que ele
sente é o nosso maior triunfo, não podemos abrir mão dela, é o que nos
mantém no controle.''

``Venha aqui, Jonas''. Senhor \versal{D.} baixou o
tom da voz, se aproximou de seu
assistente e colocou a mão em seu ombro. ``Também estou em uma posição
difícil, eu recebo ordens. Estou sendo muito pressionado, preciso que
você me ajude.''

Senhor \versal{D.} conhecia muito bem os ``pontos de apoio'' para acessar seu
assistente, sabia que sua carência era tamanha que qualquer migalha de
contato humano e consideração eram suficientes para dominar sua vontade.

Curioso Jonas ser tão inteligente e ter tido sucesso onde tantos
falharam ao criar uma forma de controlar a mente de uma pessoa, mas
não perceber que o Sr. D fazia exatamente o mesmo com ele. Sua vontade
fraquejava.

``Tudo bem, senhor. Vou dar um jeito'', suspirou.

Satisfeito, o chefe se afastou e retornou para sua mesa, voltando a
esbravejar.

``Ótimo. Intensifique a atuação dele. Eu quero que
ele crie roteiros e fantasias, que fique com mais raiva dos terroristas.
A raiva é um excelente estimulante, Jonas.''

``E a mulher, senhor, o que faremos com ela?''

``Deixe"-a como está, ela quase não interage mais. É praticamente um
vegetal, continue regando apenas para que não morra.''

``Vou ver o que posso fazer, senhor.''

``Veja também o que não pode. Se é que você gosta do seu emprego.''

\asterisc

%\emph{Theo atende o índio com dor nas costas e navega com o paciente}

Quando o sol subiu pelo horizonte, Theo finalmente se levantou, mas já
estava acordado há muito tempo. Os mosquitos não o haviam deixado dormir
e seu corpo pesava de tão cansado. Ao sair do cômodo, viu uma pequena
movimentação de pessoas e crianças correndo nuas pelo terreiro.

Saiu do banheiro, deu a volta na área da casa e viu uma fila na
porta do polo"-base. Aquele seria um dia como outro qualquer na
comunidade, não fosse pela chegada dos profissionais de saúde.

Como sempre, os índios haviam acordado com a primeira luz do
dia e sentado ao redor da fogueira para espantar o friozinho da manhã,
depois comeram beiju e desceram para o rio para se banhar. Na volta, nem todos
seguiram para seus afazeres, alguns deles foram para o polo base para
serem atendidos pelos doutores.

``Mas por que tão cedo? Eles não dormem?'', Theo perguntou para
Fernanda, que se preparava para o atendimento separando instrumentos e
pecinhas nas caixinhas.

``A vida aqui é como deve ser, Theo. Se rende ao sol'', disse, com um
sorriso esquisito.

Ele estava sonolento demais para entender a poesia na ideia de que a
vida ainda podia ser regida pelo sol, mesmo no século 21. Então foi para
a pequena cozinha e procurou água para beber. Encontrou um filtro de barro
ao lado da pia e, ao colocar a água no copo, percebeu que ela estava
amarelada.

Olhou para os lados em busca de informação, mas não havia ninguém.
Sentindo muita sede e sem outra opção, resolveu beber.

``Mas que porcaria é essa?'', disse cuspindo.

Quando colocou o líquido na boca, sentiu um gosto que desconhecia,
horrível, e se arrependeu de ter bebido sem falar com ninguém.

Então encontrou um pote com fubá e resolveu fazer um mingau, pelo menos
assim ferveria a água. Comeu"-o lentamente e voltou para a área de
atendimento.

Chegando na sua sala, no seu novo consultório, Theo tentou arrumar as
coisas que utilizaria, para atender da forma como gostava. Abriu as
portas dos armários e as gavetas, estendeu os utensílios em cima do
balcão e começou a organizá"-los. Colocou o estetoscópio na primeira
gaveta, ao lado do aparelho de medir a pressão, do termômetro e do
otoscópio. Viu que em um armário de metal e vidro ficavam os materiais
para curativos e procedimentos e se surpreendeu com o quanto
eram ultrapassados, muito mais antigos do que os que utilizava em
São Paulo.

``Bom dia.''

Um homem com o cabelo raspado do lado, um estilo tipicamente indígena
e o rosto pintado entrou no consultório trazendo um paciente. Ele
usava uma camiseta de campanha eleitoral, um shorts vermelho de futebol
e tinha um estetoscópio pendurado no pescoço. Sem perguntar nada, mandou
o paciente sentar e, vendo que as coisas estavam fora do lugar que
estava acostumado, começou a reorganizar tudo.

``Posso ajudar?'', perguntou Theo, tentando ser educado.

``Eu sou o Marcos, agente de saúde indígena. Vamos atender juntos.''

Marcos remexia as coisas e, para evitar brigar com o novo assistente,
Theo voltou sua atenção para o paciente.

``Bom dia. O que te trouxe aqui?''

O homem sorriu, mas não respondeu e assentiu com a cabeça, demonstrando
claramente que não entendia nada do que Theo falava. O agente de saúde
então se aproximou para ajudar com a tradução e o paciente começou a
descrever seus sintomas. Ele pronunciava milhares de palavras, até que
finalmente se calou.

``Ele disse que sente dor nas costas'', traduziu Marcos.

``O que mais?'', perguntou Theo.

``Só isso.''

``Mas ele ficou falando sem parar, tem de ter mais alguma coisa.''

``O que importa é que ele sente dor nas costas.''

Sem uma noite decente de sono, sem entender uma palavra do que seu
paciente falava e ainda com uma pessoa em seu consultório se comportando
como se soubesse tudo. Theo respirou fundo e, tentando se controlar,
voltou a atender.

Posicionou o paciente sentado na beira da maca, colocou as mãos em seus
ombros e o apertou, tentando descobrir qual musculatura estava doendo. O
homem tinha pernas e braços magros, mas Theo ficou surpreso com sua
força. Seus músculos eram rígidos como uma pedra e seu corpo sem uma
gota de gordura. A pele mostrava as marcas de anos de vida no mato sem
roupas, botas ou qualquer proteção.

Vagarosamente, examinou sua região lombar e, quando apertou suas costas,
o homem fez uma careta de dor. Com o exame, Theo descobriu que ele
estava com uma contratura muscular, ou seja, uma forte dor muscular
focal, provavelmente devido a um esforço excessivo.

``Há quanto tempo ele sente essas dores, Marcos?''

O agente de saúde fez a tradução e, mais uma vez, o paciente desembestou a falar
uma infinidade de palavras que Theo não entendia.

``Muito tempo'', respondeu Marcos.

``Quando, quando começou?'', Theo já estava levantando a voz.

``Desde que começou a carregar madeira.''

``Madeira? Mas quanto de madeira?''

``É bastante. Ele carrega tudo o que o corpo consegue aguentar.''

``Mas por que carregar tanto? Não tem ninguém para ajudar?''

``Em geral os parentes ajudam, mas ele não tem ninguém porque é brigado
com o irmão da mulher.''

``Tem de diminuir a carga, ou então usar carrinho. Vocês não têm
carrinho?''

``Não dá para usar carrinho na mata, doutor, tem muita planta no chão.''

Marcos não acreditava que estava tendo de explicar para um médico, um
homem estudado, porque um carrinho não tinha utilidade na floresta.

``Diga para ele que, pelo menos por enquanto, ele vai ter de parar de
carregar madeira.''

``Como assim?''

``Ele vai ter de parar de trabalhar por um tempo.''

Quando Marcos explicou para o paciente a recomendação de Theo, ele não
acreditou. Mudar a rotina que aprendera desde que nasceu não era uma
sugestão razoável.

Theo pegou seu receituário, escreveu o nome de um remédio no papel e
entregou"-o a Marcos, que ficou olhando sem entender.

``Volte daqui uma semana. Até lá, faça compressas com gelo e tome esse
remédio quando estiver com dor, no intervalo de oito horas.''

``Gelo, doutor? Não tem gelo aqui!'', Marcos já se mostrava insatisfeito
com o desconhecimento de Theo sobre a comunidade indígena.

``Tudo bem, então pode só tomar o remédio. Mas a dor vai diminuir pra
valer quando diminuir a carga. Tem de trabalhar menos.''

Marcos fez a tradução para o paciente enquanto Theo pegou uma caixinha
com pílulas e lhe entregou. Decepcionado com a consulta, o homem virou
as costas e foi embora junto com o agente indígena.

Em sua sala, Theo aguardava sozinho pelos próximos pacientes, que
não chegavam. Saiu e espiou lá fora, mas ninguém esperava para
falar com ele.

Frustrado, chateado e sem nada para fazer, ficou pensando em como seria
se pudesse conversar diretamente com o paciente que acabara de sair, e
teve uma ideia: e se encontrasse aquele paciente no mundo dos sonhos? Onde
pudesse entrar em seus pensamentos e, sem a interferência de um
tradutor, conversassem para entender melhor suas queixas.

Fechando os olhos, Theo se concentrou e logo se conectou.

Manteve seu foco no índio com a dor nas costas e, depois de alguns
instantes, aos poucos, foi se formando em seu entorno a cena de uma
floresta. Theo entrou em uma trilha de uma mata fechada e viu o homem
andando com as costas curvas, carregando nos ombros toras pesadas de
madeira.

``Larga isso. Você não pode carregar esse peso todo sozinho.''

O homem olhou para Theo e o reconheceu.

``Doutor, eu não tenho escolha. Eu estou fazendo uma casa para minha
mulher e meus filhos, são muitos, eu não posso parar.''

Ele continuou seu trabalho, e Theo sentiu pena daquele homem alquebrado
pelo trabalho. Então se colocou ao seu lado e começou a ajudá"-lo no
carregamento de madeira.

``Sabe, doutor, nós estamos dormindo em um lugar muito ruim, quando
chove molha tudo lá dentro. Não vejo a hora de terminar nossa casinha.''

O homem parou, colocou as toras no chão e deu um gemido de dor.

``Como você se chama?''

``Rauaní.''

``Como é a dor que você sente, Rauaní? Você consegue explicar ela para
mim?''

``As pernas formigam, doutor, depois de uma meia hora de trabalho. Já as
costas doem o tempo todo, quando eu abaixo, quando eu me levanto, até
mesmo quando eu me viro. Parece que piora à noite, é muito difícil
dormir.''

``O senhor não pode trabalhar tanto assim, muito menos sozinho. Alguém
precisa te ajudar.''

``Não tem ninguém, não. Minha mulher cuida da roça o dia todo e minhas
filhas ajudam ela.''

``E os outros, onde estão?''

``Aqui cada um cuida da sua família. Todos têm muito trabalho, não sobra
tempo para cuidar da família dos outros.''

``E os seus parentes?''

``Tenho um cunhado. Eu ajudei ele a construir a casa dele.''

``Ele não pode te ajudar?''

``A gente brigou, não nos falamos mais.''

``Olha Rauaní, eu entendo a tua situação, eu sei que não é fácil, mas
você vai ter de dar um jeito de parar de construir, antes que seja tarde
demais. Se você continuar nesse ritmo, uma hora o teu corpo vai pedir
socorro.''

``O que o senhor quer dizer com isso?''

``Você pode não conseguir mais andar. Se isso acontecer, quem vai cuidar
da tua família? É melhor você parar agora.''

``Você não entende, doutor. Eu não tenho escolha. Se eu parar, não vamos
ter onde dormir. Cada um faz o que tem que fazer.''

Rauaní voltou a se agachar e, ignorando suas dores, continuou o
trabalho.

Theo ficou ali observando aquele homem que, guiado
pelo instinto de sobrevivência, ignorava suas limitações físicas. Sua
força de vontade o impressionava e, diante da impossibilidade de
protegê"-lo do desgaste físico, Theo se sentiu impotente. Mas teve uma
ideia.

``Já sei. Tem um exercício que pode te ajudar.''

Theo se colocou ao seu lado e orientou"-o a imitá"-lo.

``Quando as suas costas começarem a doer, você afasta as pernas, estica
as mãos para cima e as empurra em direção ao céu, olhando para o alto.
Esse exercício alonga a musculatura e vai aliviar a sua dor.''

Impressionado, Rauaní repetiu tudo o que Theo fazia. Theo ensinou outros
três movimentos, observando"-o até se certificar de que ele havia
aprendido todos.

``Muito bem. Faça isso sempre que estiver com dor, depois de uns 30
minutos de trabalho.''

``Muito obrigado, doutor Theo. Já estou me sentindo um pouco melhor.''

O índio se despediu e foi se afastando com a madeira nas costas. Theo
gostaria de ter feito mais por ele, mas fez o melhor que podia naquele
momento.

Quando estava pronto para ir embora, a imagem da floresta foi
desaparecendo, e Theo foi navegando no mundo dos sonhos, passando por
imagens aleatórias, e se lembrou que precisava acordar e voltar para sua
vida na aldeia.

Quando abriu os olhos, tomou um susto e deu um pulo. Fernanda estava a
menos de cinco centímetros de seu nariz, encarando"-o como se estivesse
estudando um alienígena.

``Que é isso, Fernanda?''

``O que você estava fazendo, Theo?''

``Devo ter pegado no sono.''

``Nossa, você dorme de um jeito estranho. Não parecia que estava
dormindo.''

``Fernanda, onde estão meus pacientes?'', desconversou. ``Por que não
tem fila para mim? Seria melhor ter trazido outro dentista, eu não tenho
utilidade aqui.''

``Relaxa, eles só estão te testando. Querem primeiro ver se o seu
tratamento funciona. Aposto que pelo menos um deles veio ao seu
consultório.''

``Como você sabe?''

``É assim que eles fazem. Quando esse paciente melhorar, a rádio
floresta começa a funcionar e os outros vão começar a aparecer. Aqui
você vai aprender a ser paciente. Eu também tive de aprender.''

Theo ficou preocupado ao ouvir ``quando esse paciente melhorar''. Ele
sabia que o alongamento era apenas uma medida paliativa e não era
suficiente para o homem se curar de suas dores. Pior: nada que Theo
falasse iria fazer com que ele desistisse de construir sua casa. Sem sua
melhora, a sala continuaria vazia.

``Fernanda, eu vou dar uma volta. Não tenho mais nada a fazer aqui
hoje.''

\asterisc

%\emph{11 - Theo conhece a criança com bicho de pé}

Theo estava aborrecido e foi ao quarto pegar uma guloseima em sua
mochila. Percebeu que alguns pacotes de chocolate estavam faltando e
ficou furioso.

``Era só o que me faltava. Porcaria de lugar.''

Buscou uma fechadura na porta do quarto, algo que pudesse garantir que
seus pertences ficariam trancados e protegidos de ladrões, mas tudo o
que encontrou foi um buraco onde há muito tempo parecia ter morado uma
tranca. Para seu alívio, encontrou um armário com cadeado e trancou suas
coisas lá dentro.

Passou pela fila onde alguns indígenas aguardavam serem atendidos pela
dentista e viu que um deles tinha feridas nas pernas. Tentando dar
alguma utilidade para sua estadia na Amazônia, tentou puxar papo e
convencê"-lo a passar em consulta.

``Boa tarde, eu sou o Theo. E o senhor, quem é?''

O homem virou as costas e se afastou dele. Um pouco adiante, Theo viu
uma índia com uma hérnia umbilical visível no abdome, precisando com
certeza de uma consulta médica.

``Oi, tudo bem? Faz tempo que você está assim? Dói? Vamos examinar?''

Ela não respondeu e Theo tocou em seu abdômen para verificar se o
contato lhe causaria dor. A mulher, no entanto, ficou extremamente
incomodada com sua abordagem e imediatamente se retirou. Desistindo de
interagir, ele caminhou pelo terreno e, perto da escola, viu algumas
crianças se divertindo com uma garrafa pet. Talvez tivesse mais sucesso
com elas, mas, quando se aproximou, os pequenos fugiram e se sentaram a
alguns metros distante de onde ele estava. Um deles, no entanto, andava
com dificuldades e ficou para trás dos amigos. Ele tinha cerca de seis
anos e tentava se equilibrar sobre os pezinhos, como se tocá"-los no chão
lhe causasse muita dor. Theo se aproximou devagar.

``Oi. Tudo bem?''

Ele não respondeu e Theo se lembrou que o índio não compreendia sua
língua. Seus dois grandes olhos negros o encaravam assustados e Theo se
agachou para que ficassem na mesma altura. Com gestos e mímicas, pediu
para ver os pés do indiozinho, que se deixou ser pego no colo e colocado
sentado no chão. Theo, então, se sentou e começou a examinar seus pés.

Para sua surpresa, viu inúmeras feridas, muitos pontos pretos
praticamente cobrindo a sola dos pezinhos da criança. Eram bichos de pé,
doença causada pela fêmea da Tunga, uma pulga que quando fecundada
penetra na pele do ser humano ou de outros animais para depositar ovos.
Theo havia estudado essa doença na faculdade, mas nunca imaginou que
encontraria um caso tão grave, já que, em São Paulo, ela só aparece com
uma, no máximo duas bolinhas. O pé do menino, no entanto, estava coberto
de lesões.

Ele estendeu as mãos para o menino para levá"-lo ao seu consultório, mas
o pequeno não quis ir. Theo então pegou"-o no colo, mas o garoto começou
a chorar e ele se lembrou de que tinha um pacote de bala de goma em seu
bolso, o único que havia levado para a Amazônia. Abriu, comeu um e o
garoto logo estendeu as mãos em direção ao pacote. Theo lhe deu uma e
espertamente mostrou para o garoto que ele ganharia mais balinhas se o
acompanhasse até o consultório. A técnica foi infalível para ganhar o
paciente.

Chegando em sua sala, colocou o indiozinho sobre a maca e começou a se
preparar para cuidar de seus pés. A primeira coisa seria lavá"-los, então
ele colocou uma bacia e foi vagarosamente esfregando e trocando a água
marrom que ia saindo. Sabia que o tratamento seria dolorido. Os doces
que tinha no bolso ajudavam a distrair a criança, mas não eram
suficientes durante todo o processo de extração das tungas. Sem ideia
melhor, ele começou a tirar as pulgas dos pés do menino, lavando a
região com soro e, cuidadosamente, com uma pinça, extraindo os bichos,
um por um, até que a dor do tratamento começou a incomodar e o
indiozinho começou a chorar.

``Eu sei, isso não é nada legal. Hoje vamos parar por aqui, ok?''

O menino piscou os dois olhinhos para ele e, mesmo sabendo que o garoto
não o entendia, Theo interpretou como se ele estivesse lhe agradecendo.
Ainda seriam necessárias muitas sessões como aquela para deixar o menino
livre da infestação. Ele então fez um curativo caprichado, enrolando
tiras de gaze e prendendo"-as cuidadosamente com esparadrapo. Pegou"-o no
colo, como se já se conhecessem há muito tempo, e levou"-o de volta ao
terreno perto da escola.

\asterisc

%\begin{center}
%\emph{Jonas e Kátia}
%\end{center}

``Eu tenho minhas dúvidas. Será que a gente consegue manter sã uma mente
humana isolada do mundo, impedida de sonhar? Praticamente sozinha?''

Jonas observava sua imagem nua refletida no espelho no teto enquanto
pensava sobre o plano que ele mesmo sugeriu para o Senhor \versal{D.}, mas que
agora dava indícios de não ser tão perfeito assim. A ideia surgiu
quando leu \emph{Sandman}, um clássico da graphic novel, de Neil Gaiman. A
arte imita a vida ou vice versa? Uma série de dúvidas éticas sempre
voltavam a sua mente quando se permitia relaxar um pouco. Será correto,
em nome de manter segredos de Estado, manipular a mente de um indivíduo
desta forma, fazendo ele acreditar que tem realmente uma vida? E os
direitos dele? Podemos fazer isto com um indivíduo? Matar alguém no
mundo dos sonhos é assassinato?

Suspirou olhando mais uma vez para o espelho.

``O cara não tem a menor consciência de si mesmo e vive uma fantasia que
criamos para ele. A mulher então, dá até pena, já morreu e não avisaram
ela. Claro que para isto tudo dar certo precisávamos do indivíduo
perfeito, ou objeto perfeito: autoestima baixa, sem passado ou futuro,
fácil de controlar e cheio de culpa. O primeiro que a gente conseguiu
não durou muito, pirou rapidinho.''

Ela se ajeita na cama, braço formigando; afinal, quem aguenta ficar
longos minutos com uma cabeça deitada em seu ombro.

``Pirou porque estava sozinho, mas este novo, ele trouxe a mulher. Chega
a ser engraçado, a mulher e o filho que não nasceu, e que aliás não vai
nascer. Ela já perdeu o bebê faz tempo, mas a ideia de uma família
mantém ele vivo.

Quanto tempo será que um ser humano consegue se manter vivo e sadio em
uma situação como essa, Kátia? É Kátia né?

Nós estamos à beira do colapso, mais um passo errado e tudo pode ir por
água abaixo.''

Jonas respirou fundo e avaliou sua imagem; magro, mas barrigudo, baixo e
com esta voz desafinada, era difícil imaginar que alguém poderia se
sentir atraída por ele. Encolheu a barriga e voltou a falar.

``Sabe, Kátia, o Senhor \versal{D.} não quer que eu pegue mais leve. Eu acho que
deveríamos afrouxar um pouco, mas ele não acha. Não quero contrariar o
Senhor \versal{D.}, ele é um homem bom, me trata muito bem, só tem
o gênio difícil e às vezes perde o controle.''

Profissional, a mulher de cabelos vermelhos olhava para Jonas com a
postura interessada. Seu rosto e corpo sinalizavam que estava atenta à
conversa, mas sua mente vagava longe, preocupada em como faria para
completar o que faltava para pagar seu aluguel daquele mês.

Sem família e com muito poucos amigos, Jonas era um homem solitário. Sua
capacidade intelectual superior sempre manteve as pessoas afastadas, ou
talvez tenha sido ao contrário. Não que ele quisesse ser assim, as
coisas simplesmente foram acontecendo até que passaram a fugir
completamente de seu controle e ele tivesse bem pouco a se apegar.

\asterisc

%\emph{Marenostrum II - 13 - Como se forma um nerd}

Nunca imaginou que um passo em falso e cair de mau jeito sobre o próprio
pé seria a solução para seus problemas.

Do alto de seus 12 anos, Jonas já sabia que não gostava de esportes, e
mais ainda, que os esportes não gostavam dele. Passar o tempo da aula de
educação física na enfermaria, livre para ler, e até
pensar em paz, era maravilhoso.

A cada semana, o período desta aula se tornou um tormento maior. Não
conseguia dar 5 passos sem ficar ofegante. Se tentasse chutar uma bola,
tropeçava no próprio pé, como de fato aconteceu. Alguns meninos ainda
eram generosos e tentavam ensiná"-lo, mas somente para desistir após
alguns minutos; era um caso perdido.

Depois deste dia, foi criada uma rotina. Toda semana o menino inventava
aos outros que sentia uma dor diferente: na barriga, na cabeça, nas
costas, nos pés. Jonas fingia a dor, os professores e os colegas fingiam
acreditar, e assim ele fugia da dificuldade que mais lhe atordoava; e
resolvia o problema dos outros também, que não precisavam lidar com sua
presença na quadra.

Com o tempo foi aprendendo a evitar outras situações que lhe traziam
angústia, especialmente momentos onde sua inabilidade social aparecia.
Vagarosamente, foi mudando seu jeito de vestir e seus gostos.

``Eu tenho uma inteligência superior, não preciso ir a esta festa, ou
passar por esta dificuldade''. Um bom desempenho escolar reforçava para
ele a ideia de que não precisava de nada além daquilo que já tinha. Seus
pais, sem perceber o processo como um todo, se orgulhavam da maturidade
do menino e acabavam fortalecendo a sua vaidade. Assim, a profunda
solidão que ele sentia era ocultada por um mundo de ilusões que crescera
junto com ele, até mudar completamente seu jeito de ser.

Ele estava sozinho.

\asterisc

Kátia parou de fazer cafuné em Jonas, mas ele, como um cachorro carente,
pegou a mão dela e colocou"-a de volta em seus cabelos, mexendo a cabeça
para sentir o afago. Mas era tarde demais, ele já tinha voltado para a
realidade.

Ele olhou para seu corpo nu refletido no espelho e, por um instante,
sentiu"-se como um homem medíocre. Incomodado com aquela imagem, logo se
levantou, colocou seus óculos e começou a vestir sua calça de marrom
duvidoso. Olhou"-se no espelho e admirou sua imagem.

Tirou o dinheiro da carteira, jogou com desprezo sobre a mulher nua
deitada na cama e retirou"-se do quarto.

\asterisc

%\emph{Marenostrum II -- 14 - Theo começa a conhecer a vida na aldeia}

Ele passou pela roça onde viu algumas índias arando a terra, colhendo
frutas e verduras e com a pele exposta no sol escaldante. Algumas delas
traziam um bebê nas costas, preso por um pedaço de tecido envolvendo o
corpo da criança e da mãe. Aquela imagem das mulheres grudadas com os
filhos fez com que Theo pensasse em sua mãe e ele sentiu uma leve
tontura. Provavelmente fora o calor que causara aquela vertigem, e
Theo resolveu se refrescar no rio.

Chegando na prainha, colocou os pés na terra e a água escura e
congelante do rio foi tocando seu corpo aos poucos. Resolveu mergulhar
para acabar de uma vez com aquela tortura, até que foi se acostumando
com a temperatura da água e começou a gostar dali.

Theo estava curtindo seu banho quando viu um homem nadando próximo
a ele. Achou que já o tinha visto em algum lugar, foi se aproximando e
resolveu puxar conversa. O homem tinha olhos puxados indígenas, mas sua
pele era mais clara e parecia ter idade avançada.

``Você é o Theo?''

``Como você sabe?''

``Me falaram que chegaria um médico aqui na comunidade e eu vi você
tentando falar com os índios lá na fila.''

``Quem é você?''

``Meu nome é Paulo.''

``Prazer em te conhecer, Paulo.''

Theo já estava saindo da água quando o homem o deteve.

``Espera. Antes de você ir, eu queria te falar uma coisa. Sobre a
conversa que você tentou ter com os índios, percebi que você está tendo
alguma dificuldade.''

``Você viu? Foi ridículo, eles não querem conversar comigo.''

Theo não entendeu por que um homem que não o conhecia estava preocupado
com ele e lhe dando conselhos.

``Sabe, Theo, eu sei que para você a primeira coisa que se pergunta é o
nome, mas para um índio desta região não é uma pergunta que se faça logo
de cara. Para eles é muito ruim alguém saber o seu nome, principalmente
o nome indígena. Talvez você pudesse tentar uma abordagem diferente com
eles.''

``Como assim? Não posso perguntar o nome?''

``Poder, pode, mas eles se sentem desconfiados quando alguém faz isso.''

``Mas por que eles se sentem assim? Alguém aqui está sendo procurado pela
polícia?'', riu.

``Eles acreditam que quem sabe o nome deles pode fazer um feitiço,
causar doença, e até a morte. O nome define a pessoa.''

``Você só pode estar brincando. Com um nome? Não dá para acreditar.''

``Esse mundo é muito diferente do seu, Theo. Mas quem pode te ensinar
muito mais sobre isso é a Kokayjho, a xamã desta comunidade. Você já a
conheceu?''

Já não era a primeira pessoa que falava desta xamã para Theo, que ficou
um pouco desapontado por não ter sido recebido pela liderança local.

Paulo já estava saindo da água para ir embora, mas lembrou de um corte
profundo que havia feito em sua mão e achou melhor mostrá"-lo para Theo.

``Você se incomoda se eu te pedir uma ajuda? Eu estava cortando coco, o
facão escapou e cortei minha mão.''

Paulo mergulhou a mão na água para enxaguar o sangue e mostrar o
machucado para Theo, mas foi surpreendido.

``Não faça isso!'', Theo segurou sua mão. ``Esse sangue pode atrair
piranhas. Tem piranha nesse rio, né?''

``Claro que tem.''

Ao ouvir isso, Theo saiu correndo da água, mas Paulo ficou lá, rindo,
com uma expressão despreocupada.

``Sai daí, você pode ser devorado.''

``Meu filho, pode voltar. Nem tudo que você leu nos livros está certo.''

``De jeito nenhum, eu não quero ser comida de peixe.''

``Está cheio piranha aqui neste rio, mas você pode tomar banho
tranquilamente.''

Ainda desconfiado, Theo respondeu:

``E aquela história daquele peixinho que entra na uretra quando se faz xixi
e que só sai com cirurgia? O tal do Candiru?

``Ele existe, sim, mas também não é desse jeito que os livros mostram. A
chance de isso acontecer é muito pequena, nós tomamos banho no rio todos
os dias e vimos um caso em anos. Todo mundo toma banho aqui. Vai ser
muito difícil você encontrar piranha, candiru, sucuri, poraquê.''

``Poraquê?''

``É um peixe que dá choque, consegue até matar um jacaré. Mas pode ficar
tranquilo, porque acidentes com estes animais são muito raros.''

Theo confiava muito mais nos livros que nos conselhos de um homem que
acabara de conhecer e não quis mais voltar para a água. Como alguém que
mora na mata poderia saber mais que autores e pesquisadores que passaram
anos estudando o ecossistema da Amazônia e suas espécies, e registraram
seus conhecimentos em livros e revistas conceituadas?

Paulo continuou sua rotina, vagarosamente juntando suas coisas, e Theo
aproveitou que estavam sozinhos para saber mais sobre sua vida.

``Você nasceu aqui na aldeia, Paulo?''

``Eu sou filho de Manaus, mas minha família é do Ceará.''

``Então você não é índio.''

``Não.''

``E como você veio parar aqui?''

``Eu vim para servir esses povos que vivem no rio. Me mandaram, então eu
vim. Vim e fiquei. Sou um missionário''

\asterisc

%\emph{Marenostrum II -- Theo medita, encontra o indiozinho sem curativo e o devolve para a anciã}

Paulo foi embora e Theo resolveu ficar na prainha do rio, deitado sobre
a areia quente, pensando no que ele havia lhe falado. Como
iria aprender a viver na floresta se mal conseguia ficar sem o ar
condicionado ligado em sua casa? Pior ainda: como seus conhecimentos
médicos poderiam contribuir com a vida naquela aldeia se ele era incapaz
de resolver uma simples lombalgia?

Preocupado, Theo queria relaxar um pouco, descansar sua mente, e se
sentou em uma rocha rente ao rio em posição sukhasana, com as pernas
cruzadas e as mãos estendidas sobre o joelho. Apesar de não conseguir se
conectar ao mare nostrum, gostava de navegar apenas para ficar em seu
igarapé virtual, onde conseguia alcançar paz e se reconectar consigo, e
não corria o risco de ser devorado ou ter a uretra invadida por peixes.

Fechou os olhos, se concentrou e logo estava em seu refúgio. Ficou por
ali, se divertindo no seu rio particular, dentro de sua mente. Ali ele
conseguia controlar a temperatura da água e ficar o tempo que quisesse
em baixo da água, sem precisar emergir para respirar. No mundo dos
sonhos, Theo se sentia Deus, capaz de controlar até mesmo a passagem do
tempo.

Depois de relaxar e sentir sua mente equilibrada, Theo acordou e voltou
para a aldeia. No caminho, viu seu pequeno paciente com algumas crianças
brincando na frente das casas e correndo com dificuldade pela terra.
Seus pés estavam descalços, úmidos e apenas um pedaço de gaze do
curativo que Theo havia feito resistia em pendurado em volta de um dos
tornozelos.

``O que você fez?''

Theo pegou"-o no colo e, mais uma vez, esqueceu que ele não entendia o
que ele falava.

``Olha só seu pé, está todo sujo de novo.''

Seus grandes olhos pretos o encararam de novo, mas, acostumado com a
presença de Theo, abriu um sorriso deixando seus dentes comidos pela
cárie aparecerem. Algumas índias saíram de suas casas e, vendo um homem
branco com suas crianças, chamaram"-nas e em um instante o terreiro
estava vazio.

Theo olhou ao redor e não viu mais ninguém.

``Onde está a sua mãe? Venha, vamos proteger isso de novo.''

Theo levou o indiozinho para seu consultório e foi para o almoxarifado
procurar um curativo mais forte. A pequena sala de materiais tinha
cheiro de mofo e um forro despencando do teto. As prateleiras só
guardavam materiais simples e o melhor que Theo encontrou para tentar
evitar que seu curativo molhasse foram sacos plásticos.

Ele voltou para a sala e começou a limpar a sujeira dos pés do garoto,
ao mesmo tempo que foi se empolgando e começou a remover com cuidado mais
bichos do pé. Dessa vez Theo não tinha balinhas para
distraí"-lo e ele chorava mais ainda de dor. Com pena do menino, ele
tentava colocar uma anestesia superficial, o que aliviava um pouco a dor.
Fez um novo curativo nos pés da criança e colocou"-os dentro
dos sacos plásticos. E quando colocou o menino no chão, um pacote de
chocolate caiu de seu bolso.

``Então foi você, seu ladrãozinho. Esse chocolate é meu!''

Theo tirou o chocolate da criança e guardou"-o na gaveta.

``Como você encontrou?''

O menino cobriu a boca com as mãozinhas e, com uma cara levada, piscou
olhando para ele.

``Pode ficar, mas só com esse. Mas não quero mais que você entre no meu
quarto e nem mexa nas minhas coisas.''

Ele levou o garoto de volta para o terreiro, mas não havia mais ninguém
fora das casas e ele não sabia onde deixá"-lo. Então viu uma senhora
entrando em uma das casas e, ressabiado, foi espiá"-la na habitação.

``Alô, alô. Tem alguém aí?''

Uma senhora com o rosto pintado, fumando um cachimbo, apareceu.

``Eu queria devolvê"-lo para a família dele. Você sabe onde está a mãe
dessa criança?''

A senhora olhou para os pés do menino e, com uma expressão preocupada,
pegou"-o no colo e entrou na cabana.

``Ele não pode pisar na terra, está quase perdendo os pés'', Theo gritou
do lado de fora.

A anciã virou, olhou para Theo indicando que entendera a recomendação e
desapareceu dentro da casa.

\asterisc

%\emph{Marenostrum II -- 16 - Marc vai para a missão}

``Eu fiz um bolo para você, querido. Do jeito que você gosta.''

Marc estava inquieto e Teresa parecia fazer de tudo para acalmá"-lo.
Sentado na \emph{chaise longue} da sala de estar, batia seus pés no chão
sem parar. Fazia tempo que o alarme não avisava de uma nova missão e ele
se perguntava por que seu chefe estava demorando tanto para convocá"-lo.

``Deixa em cima da mesa, daqui a pouco eu como.''

Ele se levantou e começou a andar de um lado para o outro. Apesar de não
ser mais um recruta, toda missão o deixava tenso, mas não ter nenhuma
missão era pior ainda. Para se acalmar, ele repassava em sua mente
todas as horas de treinamento. Mesmo depois do sucesso das primeiras
missões, passava tantas horas no simulador que até decorou cada etapa.
Avançava com sucesso no jogo de tiro, no labirinto da morte e nos
enigmas da história, e assim se mantinha mais do que pronto para ser
chamado. A tela do escritório, no entanto, ficava dias sem qualquer
movimento.

``O que foi, querido? Por que está tão tenso?''

Marc finalmente parou de andar e colocou as mãos na barriga da esposa.

``Quando nosso filho crescer, Teresa, vou ensiná"-lo a ser um soldado
leal, igual ao pai dele.''

Ele apertou sua barriga na tentativa de fazer o filho se mexer, mas,
como sempre, ele continuava imóvel. Marc não fazia a menor ideia de
quanto tempo fazia que sua mulher estava grávida, mas sonhava em ter
o filho perto dele para ensiná"-lo tudo o que aprendera. De repente,
ouviu um alarme vindo de seu escritório.

``Finalmente.''

Sem se conter de tanta alegria, ele foi correndo para o cômodo e viu na
tela de \versal{LCD} o código da próximo missão: \versal{DREAM1438}.

Pegou seu casaco e desceu para o porão. Teresa foi atrás dele com o
pedaço de bolo e, antes que ele atravessasse a porta que levava para a
sala secreta, o entregou. Marc não estava com vontade, mas comeu só para
agradar a mulher.

``Querido, acho melhor você não ir dessa vez.''

``Você está louca, Teresa? Preciso proteger você e nosso filho dos
inimigos da liberdade. Essa é a minha vida.''

``É que você anda muito nervoso.''

``Fica tranquila. Eu sempre volto.''

\asterisc

%\emph{Marenostrum II -- Theo toma banho de manhã no rio mas ainda está deslocado, encontra indiozinho sozinho na escola}

A hora do banho parecia uma festa. Bastava o sol nascer para que as
famílias fossem para o rio, onde as crianças pulavam na água e algumas
mulheres aproveitavam para lavar as roupas cantando. Depois das
explicações de Paulo sobre os animais das águas, Theo começara a se
acostumar um pouco mais com a ideia de tomar banho ali, e até aceitou
quando Fernanda o chamou para acompanhar os índios pela manhã. Ela já se
comportava como um deles e nadava no meio das mulheres, como se fizesse
parte do clã, mas ele preferia se banhar sozinho, na margem do rio. E
mesmo que a maioria estivesse nua, ele não se sentia confortável de tirar
a bermuda.

Caminhando estabanado de uma pedra para outra, pisou em uma rocha com
musgos e escorregou caindo de bunda. Quando olhou para o lado, viu as
crianças gargalhando do seu jeito atrapalhado mas acabou rindo com elas.

Aqueles banhos coletivos o faziam sentir mais confiante, e, por isso,
resolveu abordar os índios novamente para convencê"-los a passar por sua
consulta médica, desta vez sem perguntar seus nomes. Estava certo de que
eles iriam aceitar mas todos ainda viravam as costas para ele,
rejeitando"-o novamente.

Passou mais um dia estudando, sem nenhum paciente para ver. No final do
dia, cansado, resolveu ir até a beira do rio exercitar sua navegação
para tentar relaxar e lidar com a grande frustração que sentia.

Quando passou pela escola, não acreditou no que viu. O garoto que ele já
tinha tratado duas vezes estava sentado sozinho no terreno, brincando
pelado no chão, todo sujo de terra e com os pés novamente descobertos.
Theo sentiu um cheiro ruim, foi procurar de onde vinha e encontrou um
monte de lixo jogado no chão, perto de onde eles estavam.

``Aqui não é lugar de brincar, ainda mais desprotegido desse jeito!''

Quando pegou"-o no colo para dar"-lhe uma bronca, viu que o problema da
tunga era pior do que ele imaginava. O bumbum do menino também estava
gravemente infestado de bichos e não sabia como ele ainda conseguia se
sentar.

Theo levou"-o novamente ao seu consultório, decidido que, dessa vez, seu
curativo não ia falhar. Tirou um pouco dos bicho"-de"-pé do seu bumbum,
fechando o tratamento com gaze. Tratou novamente seus pés e, para
finalizar o procedimento do dia, deu inúmeras voltas de gaze e prendeu"-a
com fita silver tape.

``Agora, sim. Quero ver você aprontar com um curativo desse!''

Para garantir o sucesso da operação, Theo tirou seu Crocs e colocou nos
pés da criança. Os pés de Theo eram três vezes maior que o do menino,
mas o curativo fazia tanto volume que garantia que o sapato ficasse
preso em seu pé.

Parecendo um palhaço, o garoto se equilibrava de um lado o outro, como
se o sapato fosse seu mais novo brinquedo, e Theo não conseguiu segurar
o riso.

Deixou o menino no pátio com outras crianças e resolveu ficar ali
observando"-o, queria se certificar que ele não estragaria o curativo
dessa vez. O garoto estava comportado, mas outra coisa chamou a atenção
de Theo. Ele observou as outras crianças sentadas na terra e viu que,
apesar de seus pés também estarem desprotegidos, nenhum deles tinha os
pés como o do seu pacientinho.

Theo ficou intrigado com aquilo, mas decidiu pensar nisto depois, já
estava escurecendo e não queria ficar ali naquele horário para evitar os
carapanãs, como eles chamavam os pernilongos que eram muitos por ali,
mas, quando percebeu que as mães das crianças passavam pelo pátio,
faziam um sinal e eram seguidas pelos seus filhos decidiu enfrentar os
pernilongos na expectativa de conhecer a mãe do menino.

O tempo foi passando, as crianças indo para suas casas e o pátio foi
ficando vazio. O menino que Theo tratara ficou lá, sozinho, e Theo não
conseguia ir embora e abandoná"-lo ali. Pegou"-o no colo e levou"-o
novamente à casa da senhora, o único lugar que ele sabia que poderia
deixá"-lo.

\asterisc

%\emph{Marenostrum II -- Theo conhece a história de Picún}

``Oi, tem alguém aqui?''

Theo ficou esperando do lado de fora da casa segurando no colo seu único
paciente.

Só depois de um bom tempo a mesma senhora que ele havia encontrado
anteriormente apareceu. Ela estava igualzinha à ocasião anterior. Mesmo
agora, no lusco fusco do entardecer, pôde notar as marcas profundas no
seu rosto que denunciavam anos de exposição ao sol, mas ainda assim Theo
não saberia dizer a idade daquela índia.

``Você pode me ajudar?''

A senhora esticou os braços, pegou o indiozinho no colo e se virou para
entrar. Vendo aqueles sapatos enormes nos pés do menino, os arrancou e
jogou no chão.

``Não tira, isso é para protegê"-lo.''

A anciã o ignorou novamente, entrou na casa e desapareceu no escuro da
casa sem janelas.

``Você pode me explicar onde está a mãe desse menino?''

Dessa vez Theo não queria ficar sem resposta e entrou na casa mesmo sem
ser convidado. Ele percebeu que ela era feita de uma estrutura de
troncos de madeira, todos cobertos com palha, e algumas frestas deixavam
os últimos raios de sol do dia entrarem. Estava suja, cheia de moscas
sobre papéis de bolacha jogados no chão.

``Escuta, eu sou o médico dele.''

A senhora continuava ignorando a presença de Theo.

``Eu preciso falar com a família dele, ele não pode mais pisar descalço
na terra, nem arrancar esses curativos. Ele estava quase perdendo os pés
e, se continuar assim, é o que vai acontecer.''

A senhora finalmente parou o que estava fazendo e resolveu atender Theo.
Com as mãos, convidou"-o para se sentar sobre um pedaço de pano colorido
que estava esticado no chão. Ele se sentou e o indiozinho foi se
aconchegar em seu colo. O gesto do menino pareceu ter agradado a
senhora.

``Você quer saber onde está a família dessa criança?''

Theo ficou aliviado ao ouvi"-la falando sua língua.

``Eu vou contar.''

Ela deu uma longa pitada no cachimbo.

``Picún também é filho de guerreiro, todo mundo aqui é. Mas Jeunamá, seu
pai, cansou da vida na floresta. Foi para o exército e acabou sendo
pervertido, ficou branco, não quis mais voltar.''

Theo já estava acostumado a ouvir histórias como essa, mas na cidade não
era o exército que fazia os homens largarem as mulheres grávidas, e sim,
na maior parte das vezes, a dependência pelo álcool ou a sede de
liberdade.

``Isso não é motivo para esse garoto ficar sozinho. Onde está a mãe
dele? E ela não pode cuidar sozinha do filho? Isso é muito comum onde eu
vivo.''

Theo não gostava de pronunciar a palavra mãe, pois para ele, que foi
abandonado logo que nasceu, o conceito de mãe era quase teórico. Mas aquela
situação atípica lhe deu um percepção que até ele desconhecia.

``Talvez seja comum na cidade, mas aqui uma mulher ou um homem sozinhos
não conseguem criar os filhos sem o apoio um do outro.''

A senhora então separou um pouco de fumo em uma tigela de barro e
começou a colocá"-lo no cachimbo. E começou a pitar.

``Naiara, a mãe de Picún, já tinha filho de um ano quando soube que
estava grávida. Ficou sozinha, deixada por Jeunamá. Teve medo. Chegou a
fazer a cova para enterrar Picún logo depois que ele nascesse. Todos
diziam que seria mais fácil sem ele.''

Theo já ouvira falar dos infanticídios nas aldeias e teve um arrepio só
de imaginar que ali as histórias que ele lia saíam dos livros e ganhavam
personagens de carne e osso. Então olhou para o garoto encolhido em seu
colo e ficou feliz por ele estar ali.

``E o que a fez desistir?''

``Um padre missionário visitava nossa aldeia naquele tempo. Disse para
ela não fazer isso porque Deus não gostaria.''

``E onde está Naiara agora?''

``Trabalhando na roça com Tutáque, o filho mais velho. Não consegue
cuidar dos dois.''

``Mas por que ela não cuida do Picún também? Por que não consegue?''

``Naiara é mãe solteira. As outras mulheres têm companheiro pra caçar,
pescar, construir a casa, então elas plantam. Mas Naiara não tem, então
ela tenta fazer tudo, muitas vezes não consegue.''

``E ela pode se casar de novo?''

``Ela é bonita e conseguiria um marido, mas todos têm medo de Jeunamá,
ele pode voltar.''

Theo estava começando a entender a dinâmica daquelas pessoas, mas ainda
levaria muito tempo para que pudesse entender completamente seu jeito de
ser. Ele precisava ser prático: será que o garoto teria o que comer
naquela noite? A anciã pareceu adivinhar os seus pensamentos.

``Hoje eu tenho farinha sobrando, mas não é sempre. Vou levar para ela.
Naiara nem sempre dorme em casa, mas hoje ela está lá. Picún vai comigo,
e deixo ele com a mãe.''

A senhora já estava saindo quando Theo viu que seus Crocs continuavam
jogados no chão e ficou preocupado. Do jeito que o garoto era levado,
nem mesmo o grande curativo de gaze que ele fizera e prendera com fita
silver tape seria suficiente para resistir às suas travessuras na terra.

``Espera!'' Theo recolheu os sapatos e os colocou nos pés do menino.
``Não deixa ele tirar esse sapato, ele protege dos bichos.''

A anciã sabia que aquele objeto de borracha não resolveria o problema,
mas foi embora sem falar nada. E Theo percebeu que teria de arranjar uma
solução melhor para salvar os pés daquele menino.

\asterisc

%\emph{Marenostrum II -- Theo tenta se aconselhar com seu mestre}

Nesta época de celular, falar no orelhão era meio ridículo, mas era o
que ele tinha.

``Perdemos mais um, Theo. Mestre Young foi encontrado em sua cama.''

``Não acredito. Precisamos fazer alguma coisa para deter esses ataques.
Mas não posso fazer nada enquanto estiver isolado nessa floresta.''

``Fique tranquilo, Theo, providências já estão sendo tomadas.''

``Eu seria mais útil se estivesse aí.''

Alcântara ficou em silêncio e Theo resolveu falar sobre o que lhe
preocupava.

``Mestre, preciso te pedir um conselho. Conheci uma criança indígena e
seus pés estavam tomados por tungas, quase perdendo os dedos e as solas
completamente arruinadas. Dá pra imaginar crianças perderem os pés
devido a uma infestação de um inseto no século \versal{XXI}? Preciso fazer
alguma coisa. Bicho de pé, mestre!''

``Sei, Theo'', Alcântara se lembrou da primeira vez que visitou a
Amazônia com o pai de Theo e em como as particularidades daquele
universo haviam lhes surpreendido.

``Tem algum jeito de descontaminar essa areia, mestre? A areia em volta
das casas está imunda e contaminada, cheia de pulgas prontas para
infestar quem pisa descalço. Você poderia falar com a faculdade de
biologia. Deve ter alguma pesquisa nessa área, algum pesticida para
jogar no terreno.''

``Vamos ver, Theo.''

``O doutor Beny pode ajudar, ele é dermatologista, já trabalhou em
campo. Você tem o e"-mail dele?''

``Fica tranquilo, Theo, vou falar com ele. Mas duvido que a solução seja
muito complexa. Eles vivem assim há séculos, se não tivesse saída teriam
todos ficado sem os pés!''

Theo estava tão nervoso e fissurado na ideia de descontaminar o terreno
que não considerava nada que seu mestre falava.

``Eu já protegi os pés do menino de todos os jeitos, mestre, ele não
para quieto e estraga todo o curativo. Vamos ter de descontaminar o
terreno, não tem outro jeito.''

``Theo, a Amazônia tem seus próprios problemas e as suas próprias
soluções. Você vai encontrar uma saída. Você já falou com a Xamã?''

``Com quem?''

O telefone ficou mudo, Theo tentou ligar de novo várias vezes, mas a
ligação não completava.

``Não acredito. Alcântara? Alô?''

Ele colocou o telefone no gancho, sentindo"-se completamente desamparado.
Como conseguiria resolver aquele problema sem se conectar ao mare
nostrum e ainda por cima sem conseguir falar com seu mestre? Ele era o
único médico em um raio de centenas de quilômetros, Theo teria de
encontrar uma saída sozinho.

\asterisc

%\emph{Marenostrum II -- Marc volta da missão e conversa com Fred}

``E aí, cara? Achei que não te encontraria mais.''

``Como assim?''

``Eu vim aqui um dia desses, mas você não estava. Foi viajar?''

Marc achou melhor não falar nada para Fred sobre a missão para a qual
Senhor \versal{D.} lhe enviara.

``Tive de ficar um tempo fora. Você encontrou com minha mulher?''

``Não, ela tava lá fora, passeando na praia. Como você não tava, fui
embora logo.''

``Mas nem trabalhou?''

``Eu não gosto de invadir a casa dos outros, não. E tua casa tá sempre
tão limpa, né, doutor? Mas fica tranquilo, não precisa me pagar nada
dessa vez, não.''

Fred se lembrou da reação de Marc ao falar de dinheiro na sua última
visita e achou melhor mudar de assunto.

``E aí, como foi a viagem?''

``Viagem, que viagem?''

``Ué, a viagem que o senhor fez. Foi trabalhar?''

``Ah, certo. Fui, fui sim. Negócios da empresa.''

Marc se lembrou do terrorista e de sua célula de resistência.
Não pôde segurar o sorriso enquanto pensava na forma como tinha
desbaratado o grupo. Foi um bom trabalho: além do chefe, havia derrotado
e eliminado vários aprendizes, não resistiu.

``Sabe, tem um pessoal querendo acabar com tudo. Mas eu não deixei, não.
Estavam escondidos em um pântano, mas fui lá e acabei com a festa,
peguei até o chefe. Agora eles não representam mais perigo, nosso mundo
está salvo.''

``Ah, o senhor parece ser muito importante, esta casa e tudo mais. O que
o senhor faz?''

Marc pensou em como poderia continuar conversando com Fred sem quebrar o
contrato de confidencialidade que havia assinado com o Senhor \versal{D.}

``Eu protejo nosso mundo, nosso jeito de viver. Tem gente querendo
acabar com tudo, alguém precisa proteger o futuro dos nossos filhos. É
bem complicado, sabe? Querem descobrir os segredos.'' Marc gostou de ver
como Fred estava admirado com seu trabalho. ``Pois é, esse é meu
papel'', disse, orgulhoso.

``Pô, que bacana. E o que eu posso fazer pelo senhor hoje, chefe?''

``Eu quero que você limpe bem o quarto do meu filho. Deixe tudo bem
caprichado.''

\asterisc

%\emph{Marenostrum II -- Theo conhece Osmarina}

``Você que é o médico novo?''

Theo estava na cozinha tentando matar a sede com a mesma água amarela e
tomou um susto quando a senhora entrou. Ela era baixinha, magra, com a
pele escura e os olhos levemente puxados, claramente mestiça. Sem
qualquer motivo começou a rir, uma risada enorme.

``Prazer, meu nome é Theo.''

Ele estendeu as mãos para cumprimentá"-la, mas ela logo lhe deu um
abraço.

``Não precisa de tanta formalidade comigo, não.''

A senhora trazia um cesto cheio de ingredientes e Theo ajudou"-a a
descarregá"-lo.

``Muito obrigada, menino. Eu sou a Osmarina. Sabe aquele monte de coisa
gostosa que você comeu esses dias? Então, fui eu quem fiz.''

Ela deu mais uma risada daquelas e Theo achou melhor omitir que estava
achando seus temperos estranhos demais para seu paladar.

``Chega pra lá, menino, que hoje eu começo cedo. O cardápio vai ser
especial.''

Osmarina começou a picar algumas folhas e Theo ficou observando seu
trabalho.

``Pode tirar o olho, viu, porque esse aqui é só para comer de noite. Eu
estou me adiantando porque à noite vou descansar.''

``Você que sempre cozinha aqui, Osmarina?''

``Sou eu, sim. Eu também dou aula, mas sempre sobra um tempinho para
ajudar aqui. E ganhar uns trocados, né?''

``Então você mora na aldeia faz tempo?''

``Ah, faz muito tempo, sim. Eu nasci aqui. Meu pai era de Pernambuco,
mas minha mãe era daqui, era índia. Era linda, Marina sua graça. Do meu
pai era Osmar.''

``Então ficou Osmarina.''

``Isso mesmo, menino esperto.'' E riu com gosto de novo.

Theo achou graça naquele jeitão da moça. Percebeu que ela era mais
pernambucana que índia, e se sentiu à vontade para fazer mais perguntas,
especialmente sobre a infestação das tungas, e pediu ajuda a ela.

``Eu percebi que quase todas as crianças aqui têm bicho"-de"-pé. Sempre
foi assim?''

``Aaah, sempre. Eu tive, meus pais tiveram, meus avós também. Dizem que
só os muito antigos que não tiveram.''

``E vocês não se preocupam com isso?''

``Não tem de se preocupar, não.''

``Eu conheci uma criança que já estava quase sem os pés.''

``Sempre foi assim e todo mundo dá um jeito. Porque não tem outro, né?''

E riu de novo.

``Mas vocês não fazem nada para resolver?''

``É que não tem o que fazer né. Isso começou quando a tribo parou de
mudar de lugar. Antes eles se mudavam o tempo todo, quando gastava a
terra iam embora para outras bandas. Mas aí começou a vir as construções
mais firmes de tijolos, eletricidade e os benefícios do governo e a
comunidade parou de se mudar. Quando uma comunidade fica muito tempo no
mesmo terreno, não tem jeito. O chão fica tudo sujo, e aí vem esse
bicho, baratas, ratos e o cheiro ruim, você já sentiu? As pessoas
estragam o chão.''

``Entendi. Mas as tungas, digo, os bichos"-de"-pé, atacam mais algumas
pessoas do que outras, e ainda não entendi por quê. Tô pensando em como dar um
jeito nisso. Um professor muito famoso, dermatologista, lá de São Paulo
vai\ldots{}''

``Ih, tá faltando a mandioca! Eu vou lá buscar.''

Osmarina deixou Theo falando sozinho. Apesar de incomodado, ele
aproveitou para beliscar os ingredientes que ela estava preparando.
Cheirou um pedacinho das folhas verdes, colocou"-as na boca e se assustou
quando sentiu sua língua adormecer. Não era um grande gourmet, mas sabia
que aquilo só poderia ser jambu.

Empolgado com a nova descoberta, se divertiu comendo as folhas e
sentindo seu efeito na boca.

\asterisc

%\begin{center}
%\emph{Theo quer ir no ritual}
%\end{center}

Theo ficou deitado na rede, pensando em tudo o que a anciã lhe havia
dito, a história sobre Picún, que quase fora morto ao nascer, e a
dificuldade que a mãe tinha em cuidar dele sozinha.

``É de cortar o coração! Tenho que fazer alguma coisa.''

Falou para si mesmo, lembrando"-se do dia em que o conhecera, o menino
mancando, quase sem conseguir caminhar.

Pensou também no que Osmarina lhe havia contado sobre a contaminação do
terreno e decidiu que, enquanto a ajuda do médico de São Paulo não
chegasse, ele mesmo iria dar um jeito de resolver a infestação de tunga
na aldeia. Theo não acreditava que, com toda a ciência que conhecia,
algo tão simples como tungas poderia colocar em risco os pés de uma
criança.

Ele já traçava um plano com as providência que iria tomar quando
Fernanda chegou e o tirou do pensamento.

``Ah, você está aí? Espero que não tenha te acordado.''

``Tudo bem, eu não tava dormindo.''

Fernanda entrou no quarto e Theo viu que ela vestindo apenas shorts e
sutiã, sua pele estava inteiramente riscada com finos traços pretos,
feitos com carvão, formando um grande desenho.

``Tá bonita. Do que é feito?''

``A pintura é com carvão, mas junto com o carvão tem jenipapo. O carvão
apaga no banho e elas usam só para poder enxergar o próprio desenho.
Depois o carvão apaga, mas a pele se queima por causa do jenipapo e fica
marcado. Dura pouco, que nem henna.''

``Legal. E pra quê tudo isso?''

``É que vai ter uma grande festa dos índios. É uma cerimônia, você nem
imagina, é a coisa mais louca do mundo.''

``Opa, então vou me arrumar. Será que alguém me pinta também?''

Theo já estava se levantando quando Fernanda logo o interrompeu.

``Melhor não, Theo. Dessa vez você não foi convidado.''

``Mas por quê? Eu quero participar dessa festa.''

``Sabe o que é, Theo? É uma cerimônia muito importante para os índios,
até a xamã participa. Eles não gostam de estranhos participando.''

Ele fingiu que não ligou, até explicou para si mesmo que acabara de
chegar, mas Theo ficou muito chateado por não ser convidado e muito
curioso em conhecer aquela festa. Queria saber como era o ritual e
principalmente conhecer a xamã.

Depois que Fernanda foi à festa, até pensou em ir espiar, mas como
chegar sem ser visto a uma casa no meio de um terreiro de 200 metros de
raio? Por fim desistiu e ficou sozinho no polo ouvindo os distantes
batuques e cantos que vinham com o vento.

\asterisc

%\emph{Marenostrum II -- Theo sonha com o pai e Alcântara}

Aquele ritmo constante fez com que Theo ficasse sonolento e logo caísse
no sono.

Enquanto dormia, ele viu dois homens e uma mulher conversando ao pé de
uma fogueira, separados dos índios que dançavam e cantavam formando uma
roda. Na dança, eles batiam os pés na terra dura, fazendo com que as
sementes penduradas em volta de seus tornozelos soassem como chocalhos.

Theo estranhou quando percebeu que os homens sentados em volta da
fogueira eram Alcântara e seu pai. Se surpreendeu mais ainda
quando viu que ao lado de seu pai estava uma bela índia que, apesar
de estar vestida para a cerimônia, não dançava junto aos seus; ao
contrário, mantinha uma postura muito íntima e próxima dos dois homens
brancos.

Theo acelerou o passo para se aproximar e ver melhor seu pai, queria
ouvir o que ele conversava com Alcântara e com a índia. Quanto mais
caminhava para perto deles, porém, mais eles pareciam distantes. A bela
índia olhou para seus olhos e fez uma pequena menção de balançar a cabeça
negativamente, e nesta hora tudo começou a ficar turvo e ele voltou.

Num susto, Theo abriu os olhos e se viu
deitado em sua rede. Acordou sem saber o que estava acontecendo, mas
logo se acalmou: tudo não passara de um sonho.
Algumas vezes era difícil separar as fantasias de seu cérebro das
lembranças de outras pessoas que visualizava enquanto navegava. Neste
caso, porém, mesmo estando isolado na Amazônia -- e apesar da sensação
de realidade do que viveu --, Theo sabia que era um sonho, pois não
conseguiria se conectar com
o cérebro de seu mestre ou do seu pai, a muitos e muitos quilômetros
dali. Então, estas não poderiam ser memórias de nenhum deles. A
índia, ele nunca havia visto, pelo menos não que se lembrasse.

\asterisc

%\emph{Marenostrum II -- Theo pergunta da pajé e quer mudar os hábitos dos índios}

Depois de tomar banho no rio com os índios, como Theo já aprendera a
fazer todas as manhãs, ele foi procurar Fernanda no consultório dela.

``Foi legal a festinha ontem?'', perguntou, tentando fingir que não se
importava em não ter participado.

``Não foi nada demais, Theo. Quem sabe na próxima eles te chamam.''

``A xamã estava lá, mesmo?''

``Claro que sim. Não dá para fazer a cerimônia sem ela.''

``E onde ela está agora? Preciso muito falar com ela.''

``Ih, Theo, ela partiu hoje cedo.''

``Como assim partiu? Ela quase nunca está aqui, e quando vem já vai
embora?''

``Theo, fala sério, como eu vou saber? Eu não mando nela.''

``Fernanda, eu bolei um plano, já pensei em tudo. Vou ajudar essa tribo
a não ter mais doença no pé, vou ensinar eles a recolher o cocô dos
animais, a jogar o lixo no lugar apropriado.''

Fernanda ouvia Theo com uma expressão de descrença, mas ele não se
importou e continuou a falar.

``Não tem como falhar, Fernanda. Mas eu queria falar com a chefe daqui
antes de começar.''

``Ela pode demorar pra voltar, Theo. A gente nunca sabe quando ela vem
ou vai.''

``Eu não posso esperar, Fernanda. Eu posso ter algum problema se não
pedir autorização para ela?''

``Não, Theo. As coisas aqui são bem diferentes, todos têm liberdade para
fazer o que quiserem. A pajé é importante para outras coisas, não para
isso.''

``Como assim, que tipo de coisas?''

``Relação com o místico, nada a ver com cocô de animais, Theo.''

Theo queria muito saber compreender melhor o que Fernanda queria dizer
com místico, mas estava tão excitado com a ideia de resolver o problema
das tungas que deixou passar.

``Eu vou começar hoje, Fernanda. Você me ajuda?''

``Theo, eu não quero te desanimar, mas isso não vai dar certo.''

``Não tem como dar errado.''

``Theo, eles vivem assim há muitos anos. É muito difícil mudar os
hábitos de uma comunidade.''

``Fernanda, eu já estudei muito sobre doenças causadas por falta de
saneamento básico, na faculdade. Eu sei exatamente o que precisa ser
feito.''

Fernanda percebeu que nada faria Theo mudar de ideia, e achou melhor não
contrariá"-lo mais.

``Tá bom, Theo, vai lá. Boa sorte.''

\asterisc

%\emph{Marenostrum II - Theo começa a varrer a areia}

``O que ele está fazendo?''

Os índios olhavam para Theo e não entendiam nada. O médico recém"-chegado
varria a areia do terreno que ficava perto da escola, onde as
crianças costumavam brincar depois da aula, tentando retirar toda a
camada superficial da areia em um trabalho quixotesco.

Quando Paulo passou por ali viu uma nuvem de pó, e se aproximou para ver
o que era.

``Paulo, você pode continuar isso aqui pra mim só um minuto? Eu vou ali
pegar um negócio e já venho.''

Theo nem esperou Paulo responder e saiu. Quando voltou, trazia um
carrinho de mão e uma pá.

``Você chegou na hora certa, eu não ia conseguir fazer isso sozinho.
Você segura a pá pra mim enquanto eu empurro a terra?''

``O que você está fazendo, Theo?''

``Presta atenção, Paulo. Sabe essa terra que está na superfície? Ela
está totalmente contaminada. Eu vou tirar ela daqui, vamos deixar tudo
limpo. Assim as crianças não vão mais pegar bicho de pé.''

``Mas Theo, essa terra logo vai estar contaminada de novo.''

``Não vai, não, porque eu vou ensinar os índios a evitar a
contaminação.''

``Theo, isso não vai dar certo.''

Theo não quis dar ouvidos ao que Paulo dizia e continuou seu trabalho
sozinho. Colocava a terra contaminada no carrinho e levava"-a até um
local afastado. Então começou a recolher o lixo que estava em volta do
terreno, encontrando de tudo: embalagens, restos de comida, cocô de
cachorro\ldots{}

Depois de um dia inteiro de trabalho, Theo olhou para o monte de
resíduos que havia feito e ficou espantado com o tanto de terra e de
lixo que havia recolhido. Então, pegou um desinfetante que era usado no
polo"-base e jogou sobre o terreno. Os índios que passavam apenas olhavam
o que ele estava fazendo e iam embora, sem falar nada.

Quanto terminou, Theo estava cansado, mas satisfeito com seu trabalho.
Foi tomar um demorado banho no rio, se lavou bem com sabonete, enojado
com tudo que havia limpado mas feliz por sentir que tinha mudado a
história da comunidade.

\asterisc

%\emph{Marenostrum II -- Fred desabafa com amigo sobre Marc}

Fred tinha recebido seu salário do mês e convidou o amigo para comer um
x"-burguer no bar que eles costumavam ir.

``Tá com uma cara de cansado, irmão. Não vai me dizer que tá trabalhando
mais?''

``Tô nada, Zé, tô trabalhando igual. É que agora já não consigo dormir
como antes no trabalho.''

``Te descobriram? Eu falei para você tomar cuidado.''

``Não é isso, não. É que agora eu tô conversando com o paciente lá na
\versal{UTI}.''

``Mas você não falou que eles estavam malzões, à beira da morte?''

``Estão, sim. Mas eu falo com ele quando tô dormindo, no sonho. É um
negócio muito louco.''

``Cruz credo'', Zé fez o sinal da cruz. ``Minha vó que falava que a
alma da gente saía do corpo quando a gente dormia. Não mexe com essas
coisas não, Fred.''

Fred deu risada do amigo.

``Não tem nada a ver com isso, não, Zé. É um negócio diferente, eu acho
que nossas mentes se conectam.''

Zé ficou olhando para o amigo sem entender. Fred continuou:

``Eu percebi uma coisa naquele prédio: a parede é diferente, no cimento
tá cheio de fio, é como se tivessem prendido a alma do cara lá dentro,
por isso eu só sonho com ele e com aquela casa. Dá muita pena.''

``Pena? Ele que está morando no casarão e vivendo vida de rei, não é?''

``É difícil explicar, é triste, meu. Eles não deixam o cara morrer nem
sonhar, por isso faço companhia pra ele. Eu tive que mentir para o
médico.''

``Médico? Você tá doente?''

``Nada, fui chamado para avaliação de rotina, falei tudo, o que eu como,
quanto eu durmo, mostrei uns exames. Só não falei que eu tô conversando
com o paciente. Esses chefes são meio esquisitos, sabe? Não gostam que
ninguém nem entre naquela \versal{UTI}, fiquei com medo de ser mandado embora.''

``Tá certo, não dá para arriscar, mesmo. Agora, só você para
ter pena desse cara importante assim. Ele que tem pena de você!'', riu.

\asterisc

%\emph{Marenostrum II -- Theo tenta mudar os hábitos dos índios}

``Oi, posso entrar?''

Uma índia com o olhar assustado, carregando um bebê no colo, apareceu
para atender ao estranho homem branco que lhe chamava na porta de sua
casa. Só quando ela viu que ao lado de Theo estava Marcos, o conhecido agente de saúde indígena,
se sentiu mais tranquila para deixá"-los entrar em sua
casa. Não havia sido fácil para Theo convencer o agente a acompanhá"-lo,
mas ele sabia que não seria recebido se estivesse sozinho.

``Iamacunrê, esse é o Theo, o médico que veio da cidade.''

A índia ouvia Marcos e observava Theo de canto, com um olhar de
desconfiança.

``Esses cachorros são seus?''

Iamacunrê não respondeu.

``Cachorro não tem dono, não, doutor'', disse Marcos.

``Isso é muito estranho pra mim. Olha, eu trouxe essas sacolas que é
para ela jogar o lixo dentro e depois fechar.''

``Deixa aí no canto.''

``Não, Marcos, precisamos explicar por que é importante jogar o lixo
dentro na sacola.''

``Tá certo. Iamacunrê, o doutor está pedindo para vocês jogarem os
restos de comida, as embalagens, tudo aqui dentro, e depois deixar
fechado. Assim os bichos não pegam e espalham tudo por aí.''

Iamacunrê olhava para Marcos achando que ele estava falando uma coisa de
outro mundo, e Theo resolveu passar para os próximos passos.

``Bom, Marcos e Iamacunrê, lá onde eu moro, muita gente tem cachorro
dentro de casa, e ninguém pega doença no pé. Sabem por quê? É porque tem
um jeito de ensinar eles a fazer xixi e cocô no lugar certo.''

Os dois olhavam para Theo como se ele fosse um extraterrestre.

``É super simples, eles aprendem rapidinho.''

Theo chamou o cachorro que estava ali para perto dele e se agachou.

``Vamos fingir que esse cachorro acabou de fazer cocô aqui. Você pega o
focinho dele, esfrega no cocô e dá umas palmadas nele. Assim ele entende
que tá errado.''

Theo ia fazendo gestos enquanto explicava.

``Depois de fazer isso, você pega o cocô dele e coloca lá fora, em cima
do papel que eu coloquei perto da porta. Venham ver.''

Os três saíram da casa e viram um pedaco de papel ao lado da porta.

``Aqui vai ser o novo banheiro dele.''

Theo colocou o cocô em cima do papel e fez carinho no cachorro.

``Muito bem, é aqui que você tem de fazer cocô.''

Theo então se levantou e continuou sua explicação.

``Depois de algumas vezes que vocês repetirem isso, o cachorro vai
entender que, quando ele faz cocô ali, ele ganha um carinho. Pode ser
que demore um pouco, mas é só repetir várias vezes que ele vai
aprender.''

Iamacunrê não estava gostando nada daquela performance de Theo, e Marcos
achou melhor ir embora.

``Tá certo, doutor. A Iamacunrê aprendeu, né? Vamos voltar pro polo"-base
agora.''

``Tá bom. Tchau, Iamacunrê. Tchau menininho.''

Todo feliz, Theo tentou brincar com a criança que estava no colo da
índia, mas ela, com uma expressão completamente oposta à de Theo, não
deixou.

``Não esquece do saco de lixo, hein?'', disse, sorridente.

Quando estava voltando para o polo"-base, Theo passou pelo terreno da
escola, onde passara o dia trabalhando, e não acreditou no que viu. Três
crianças peladas brincavam no monte que ele havia feito, em meio ao lixo
e a terra contaminada, entre elas o indiozinho que ele havia tratado.

Ele foi correndo até o monte e tentou tirá"-las no grito.

``Saiam daí, está tudo contaminado!''

As crianças não lhe obedeceram, e ele resolveu pegá"-las no colo e
tirá"-las à força dali.

\asterisc

%\emph{Marenostrum II -- Theo descobre o porquê do brinco}

Depois de tomar outro banho, Theo foi dar uma volta na aldeia para se
acalmar. Não era possível que as pessoas dali preferissem viver daquele
jeito, com tudo contaminado e sujo. Ele não conseguiria fazer aquilo
sozinho, mudar a maneira de pensar deles era mais difícil do que imaginara,
desde convencer o paciente a alongar os músculos e diminuir a carga de
trabalho, até ensinar as pessoas que cocô de cachorro causa doenças.

Passando pelas ocas, viu uma índia sentada no chão com a filha pequena
no colo. Ela havia tirado o brinco de agulha da orelha e, com ele,
cutucava o pé da filha até conseguir tirar um bicho de pé. A pequena
esticava o pezinho como que acostumada com o ritual. Para surpresa de
Theo, após retirar o inseto, ela o colocou na boca, mastigou e engoliu.

Apesar de enojado, Theo continuou observando o minucioso trabalho da
índia, tirando bicho por bicho e comendo, até que ela percebeu que
estava sendo observada.

``O que você está fazendo?''

``Limpando o pé dela. Se não toma tudo.''

Vendo aquela cena, Theo percebeu
que a cultura simples daquelas pessoas já havia encontrado as respostas
que ele procurava, e que tudo era muito mais simples do que imaginava.
Extasiado com a descoberta, saiu correndo para o polo"-base.

\asterisc

%\emph{Marenostrum II - Theo e o colar de agulha}

``O que é isso no seu pescoço?''

Fernanda já estava achando Theo diferente desde que o conhecera na
viagem que fizeram de barco até a aldeia. Ele estava com a pele mais
morena e o olhar mais profundo, mas ela estranhou ainda mais quando o
viu usando um colar de cordão de palha no pescoço, com um pingente de
madeira pontuda, como uma agulha. Não combinava com ele.

``É para limpar o pé do Picún.''

Algumas índias estavam perto e riram quando ouviram a resposta de Theo.

``Por que essas risadas?'', perguntou Fernanda, que tinha mais liberdade
com as índias. Elas responderam, rindo:

``Isso não é coisa de homem.''

``São as mães que tiram o bicho dos filhos.''

Theo não se importou nem um pouco com
aqueles comentários, principalmente quando se lembrou do rostinho feliz
de Picún quando ele limpou seu pé com aquele instrumento. O menino
cresceu vendo seus amigos sendo tratados assim por suas mães, mas jamais
tinha sido cuidado daquela forma. Quando Theo sentou, tirou o
instrumento do seu pescoço e começou a limpar seus pés, Picún se sentiu
acolhido.

Theo foi para a cozinha comer alguma coisa e Fernanda o acompanhou. Eles
se sentaram à mesa de madeira e dividiram um pouco de açaí.

``Achei legal o que você fez com o Picún.''

``Fernanda, eu percebi o quanto eu era arrogante. Eu cheguei aqui
achando que eu ia mudar o jeito deles viverem, agora percebi que eu não
sei de nada. A solução para o problema dos pés dessas crianças já
existe, sempre existiu, e era mais simples que eu imaginava.''

``Os índios têm o jeito deles de resolver as coisas. Sempre funcionou,
durante anos.''

``Sim. Ainda não desisti de ensiná"-los a separar o lixo e cuidar da
sujeira, mas percebi que pode ser com calma, com o apoio das lideranças,
passo a passo.

Eu demorei tanto para perceber isso tudo que o Picún quase perdeu o pé.
A doença dele era social, você viu como está bom o pezinho dele agora?''

Theo ficou um tempo em silêncio, se lembrando de tudo o que tentou fazer
para resolver a infestação de tungas.

``Até a areia eu tentei limpar.''

Os dois riram alto e, sem nem falar nada, Theo percebeu que Fernanda
queria mais açaí e colocou uma colherada de sua porção na cumbuca dela.

``É, Theo, a gente cresce muito quando enxerga os índios com respeito. É
uma troca, a gente aprende com eles e eles aprendem com a gente.''

``Você tem toda a razão.''

Fernanda nunca tinha ouvido Theo concordando com o que ela falava e,
depois daquela conversa, ela passou a olhá"-lo de um jeito diferente.

``Tá uma noite tão linda. Acho que vou nadar no rio. Quer ir?''

Theo não pensou duas vezes e foi com Fernanda dar um mergulho. O
reflexo da lua na água, o som dos sapos e até a temperatura da água,
tudo estava perfeito, como um momento aprisionado no tempo.

No instante seguinte, o equilíbrio se quebra. Fernanda se assustou com
um peixe, uma folha ou qualquer coisa que passou perto de suas pernas --
talvez um susto um pouco excessivo para uma pessoa que convivia com a
natureza a tanto tempo --, e pulou nos braços de Theo. Se olharam
profundamente e em silêncio, por um, dois segundos, finalmente se
beijaram, a princípio de forma tímida, mas logo um beijo demorado e
verdadeiro.

Nada mais incomodava Theo. Ele já não ligava se Fernanda depilava ou não
as axilas, nem se importava com as feras imaginárias que antes o
afugentavam da água. Tudo o que vinha acontecendo com ele, aqueles dias,
a beleza da floresta, a mudança da sua percepção do mundo, tudo
convidava Theo a derrubar suas barreiras interiores e se conectar de
alguma forma àquele lugar e àquela mulher.

Aos poucos, como um agradecimento, Theo relaxou e viveu uma noite como a
muitos anos não vivia, e amanheceu nu na areia sob o céu da amazônia com a
comunidade indo se banhar nas águas do rio.

\asterisc

%\emph{Marenostrum II -- Fred entra no quarto do bebê}

O quarto estava todo bagunçado, com almofadas, cobertores e roupas de
bebê espalhadas pelo chão. Fred não acreditou quando viu tudo daquele
jeito e ficou imaginando quem poderia ter feito aquilo.

Colocou os objetos de volta em seus lugares, arrumando tudo com
capricho. Então, pegou seu esfregão e começou a passá"-lo no chão, mesmo
achando que o cômodo já estava suficiente limpo e que não ficaria melhor
do que já estava. Então, viu caído no chão, embaixo da cama, o vaso onde
Marc havia plantado o grão de feijão. A terra estava espalhada pelo chão
e, quando ele pegou para recolhê"-la e colocá"-la de volta no vaso, viu,
surpreso, que o grão não crescera e continuava exatamente igual à
primeira vez que o vira, como se o tempo não tivesse passado.

Realmente havia algo bem estranho com aquele lugar.

\asterisc

%\emph{Marenostrum II -- Theo tem dor de barriga, sonha com o índio machucado e se prepara para recebe-lo}

Theo estava sentado olhando as
crianças correndo quando começou a sentir uma forte cólica. Teve de
correr para o banheiro e passou muito mal. Sabendo que a única solução
para aquele problema era se hidratar, foi pedir ajuda para Fernanda.
Procurou"-a em vários lugares, e já estava se sentindo tonto devido ao
problema intestinal. Finalmente a encontrou perto da escola.

``Fernanda, preciso de água limpa, não dá pra tomar aquela água amarela,
o que eu faço?''

``Theo, podemos ferver para você, mas a água aqui é boa, vem de um
igarapé branco e é pura, todo mundo toma a mesma.''

Theo queria explicar à Fernanda o conceito da diarreia do viajante, que
acomete aqueles que vêm de fora e preserva os nascidos no local, cujas
bactérias do intestino ``são de casa'', mas não tinha forças para isso.

``Do jeito que estou, preciso de muita água pra me manter hidratado,
talvez até soro.''

``Não se preocupe, a floresta vai entrar no teu corpo para matar a tua
sede. Você precisa ouvir o que a água tem a dizer. Venha, eu vou te
ajudar a entrar em contato com ela.''

Nem depois da noite maravilhosa ele conseguia achar normal as maluquices
hippies de Fernanda.

``Pode deixar, Fernanda, não precisa.''

A barriga de Theo começou a doer de novo e Fernanda percebeu que algo
estava errado.

``O que você está sentindo?''

``A cólica voltou. Tá feio, eu preciso me deitar.''

Theo encheu uma garrafa com água e foi se forçando a beber. Teve de sair
correndo novamente para o banheiro e, quando chegou em seu quarto, estava
tão fraco que deitou na rede e logo adormeceu.

Entre os momentos de delírio e sonho, reconheceu que havia sido
transportado para o meio do mato, e estava na mente de um índio. Este
era um dos preços de ser navegante, algumas vezes, especialmente em
momentos extremos, não era capaz de controlar sua mente.

Com um facão, o índio ia arrancando as folhas de um açaizeiro e as
torcendo com as mãos, até transformá"-las numa corda. Theo não sabia onde
estava, mas continuou observando a cena e, como navegador mais
experiente, tinha certeza que estas cenas estavam acontecendo naquele
momento e não muito longe dali, ``ao vivo''. A nitidez, luz e cores das
imagens mostravam para ele que não se tratava de uma memória e nem de um
sonho.

O índio amarrou as duas pontas da corda com um nó, formando o que
parecia um colar, apertando"-o com força. Sem notar a presença de Theo,
se aproximou de um açaizeiro, olhou para o alto e viu no topo da árvore
um cacho carregado dos pequenos frutos roxos. Ele apoiou o círculo feito
de folhas no chão e colocou seus pés em seu centro. Erguendo os braços,
segurou com as mãos o tronco da árvore e, com um salto, grudou seus pés
no açaizeiro. O colar prendia"-o na planta como se ele tivesse garras de
gato. Dobrando e esticando os joelhos, revezando o peso de seu corpo
entre as pernas e os braços, escalou o açaizeiro em um movimento de
encolhe e estica que lembrava uma lagarta.

No topo da árvore, seu peso fez o tronco inclinar e Theo pensou que ele
fosse cair. Mas, habilidoso, o índio se balançou como se estivesse
brincando, parou na posição vertical, cortou o cacho e desgrudou as mãos
e os pés da árvore e, como um bombeiro em um cano, caiu em pé no chão.

A cena se repetiu inúmeras vezes e a quantidade de cachos de açaí que se
amontoavam sobre a terra ia aumentando. Theo já estava cansado de ver
aquela cena, quando ouviu um forte estalo. O tronco se quebrou e o índio
caiu com tudo, estatelando"-se no chão. Theo se aproximou e viu que o
homem caíra sobre sua perna, deixando a fratura exposta e um pequeno pedaço de
osso para fora da pele. Desesperado, Theo queria ajudá"-lo, mas sabia que no
mare nostrum conseguiria no máximo falar com o homem. Mesmo duvidando que
o homem tomado pela dor conseguiria ouvi"-lo e até mesmo que acreditasse
em uma voz vindo do além, Theo arriscou.

``Vá para o polo"-base que eu cuido de você!''

O homem achou que estava delirando, olhou em volta e não viu nada.

``Confia em mim, eu sou médico, eu posso te ajudar. Eu vou estar te
esperando no polo"-base''.

O índio então respirou fundo e começou a reagir, se arrastando em
direção à aldeia.

Theo ficou tão preocupado com o homem, mas tão feliz com a possibilidade
de ajudá"-lo, que seu coração estava quase saindo pela boca. Ele então
abriu os olhos, esquecendo"-se de sua fraqueza, e foi procurar Fernanda.

``Precisamos nos preparar, Fernanda.'' Ele parou alguns segundos para
respirar. ``Vai chegar um índio todo machucado, com a perna quebrada,
vai precisar de cirurgia, precisamos lavar a fratura.''

``Theo, onde você estava? Do que está falando?''

``Na mata, Fernanda. Foi na mata.''

``Theo, você está delirando.'', respondeu, pensando que a diarreia
poderia estar deixando Theo confuso.

``Fernanda, escuta, eu sei o que estou falando. Um índio se acidentou e
deve chegar aqui já, já.''

``Deve ser a dor de barriga. Toma um pouco desse chá, Theo, vai fazê"-lo
se sentir melhor.''

Fernanda tirou uma garrafa térmica de um armário e entregou para Theo. A
bebida já estava fria e tinha gosto de flores misturadas com algumas
ervas, mas fez com que Theo se sentisse melhor.

Theo tentava convencer Fernanda de que ele tinha visto muito mais que um
sonho, mas ela não acreditava nele. Ele virou as costas e foi para sua
sala, onde começou a preparar as coisas para esperar pelo homem
machucado. A noite já estava caindo, ele não chegava e Theo já começava
a esperar pelo pior.

\asterisc

%\emph{Marenostrum II -- O índio acidentado chega}

Theo estava se sentindo melhor, depois da noite anterior e o dia todo
entre idas e vindas ao banheiro. Finalmente, conseguiu ficar sentado no
pátio, tranquilo, mastigando uma folha de capim que Fernanda lhe dera.
Viu, então, o homem saindo de uma trilha. Ele se arrastava pelo chão e,
quando percebeu que Theo o vira, fechou os olhos e deitou a cabeça no
chão.

``Não acredito! Eu sabia que você ia conseguir.''

Theo saiu correndo ao seu encontro e viu que o homem estava todo sujo e
com uma das pernas muito inchada. Tentou carregá"-lo no colo, mas não
conseguiu e foi procurar ajuda. Finalmente encontrou Fernanda.

``Fernanda, ele chegou.''

``Ele quem, Theo?''

``O índio, Fernanda, ele está todo machucado.''

``É, Theo? Que ótimo, né? Depois eu vou lá conversar com ele.''

Theo percebeu que Fernanda ainda não acreditava nele, então a arrastou
até onde o índio estava. Quando ela o viu, ficou espantada com a
previsão de Theo.

``Como você sabia?''

``Vamos, Fernanda, me ajude a levá"-lo.''

Eles carregaram o homem até o consultório e, no caminho, foram
vistos por alguns indígenas. Curiosos, eles os seguiram, e no final várias pessoas ajudaram a carregar o paciente e entraram no
consultório junto com eles. Theo achou melhor deixá"-los afastados e
tentou tirá"-los do local, mas não conseguia, e teve de continuar com a
sala cheia e muitos espectadores em cima de seu trabalho.

Theo começou a examinar o homem ferido e percebeu que sua face estava
pálida, ele havia perdido muito sangue e gemia muito de dor. Tinha a perna
inchada e o pé gelado, e Theo preocupou"-se ao perceber que o sangue não estava
circulando no pé. Ao medir sua pressão, constatou que
estava extremamente baixa. Colocou um acesso em sua veia e prendeu um
saquinho de soro para hidratá"-lo.

``Fernanda, preciso de ajuda para lavar essa fratura.''

Eles colocaram sua perna em cima de uma bacia e Theo limpou"-a com uma
escova e um sabonete próprio para lavar as mãos antes de cirurgias. Para
isto usou dezenas de litros de soro, em mais de uma hora e meia de
trabalho. O homem gemia muito, apesar dos analgésicos que recebeu
diretamente na veia.

``Espero que adiante'', ele falou preocupado. ``Ele demorou muito para
chegar até aqui, se tivéssemos lavado logo após o acidente teria sido
muito melhor.''

Eles então o colocaram de volta sobre a maca. Depois de ter certeza de
que fizera tudo o que estava ao seu alcance para limpar o ferimento, Theo
tentou reduzir a fratura tracionando o pé do paciente, mas a contratura
muscular e a dor do paciente não permitiram. Então, recorrendo à sua boa
formação de médico, instalou um sistema de roldanas improvisado ---
com uma tração com o ângulo preciso e com cerca de 10\% do peso do indivíduo, utilizando saquinhos de soro, cada um com aproximadamente um quilo. Assim, a gravidade
foi lentamente reposicionando o osso para seu local original até finalmente
se encaixar no lugar certo. Theo, por fim, fixou o pé com uma tala e
fechou a pele que havia se rompido no momento do trauma.

Fernanda, assim como os índios que estavam presentes, ficou extremamente
impressionada com a habilidade de Theo. Ela o respeitava como médico,
mas apenas se deu conta do quanto aquela profissão o exigia quando
vivenciou seu trabalho ao vivo, viu sua precisão manual e sua
resiliência naquela situação de extremo sofrimento do paciente.

Theo logo lhe deu um antibiótico e, concentrado, fez um curativo em
seus pés. Então examinou"-o, ficando aliviado.

``O pé já está esquentando, o sangue está voltando a circular.''

Theo começou a lavar as mãos e a guardar os instrumentos que havia
utilizado, os índios foram embora, mas Fernanda o esperou para sair.
Theo se sentou em uma cadeira ao lado do índio, que estava dormindo.

``Ué, você vai ficar aí?''

``Pode ir, Fernanda, daqui a pouco eu vou.''

``Theo, você também não está muito bem. Até pouco tempo atrás estava com
diarreia''. Theo estava tão preocupado com o índio que até se esquecera
de seu próprio corpo.

``Teu chá deve ter ajudado.''

``Claro que ajudou. Mas agora seria bom você descansar.''

``Eu vou ficar aqui, Fernanda. Não vou deixar ele sozinho.''

Pela primeira vez desde que chegaram, Fernanda não achou Theo um garoto
tão mimado e admirou o seu gesto.

``Fernanda, você sabe quem é ele?''

``Não sei, Theo. Mas ele é de uma comunidade vizinha, aqui ele não
mora.''

``Fernanda, eu lavei a fratura e reduzi, fechei a lesão na pele e demos
antibiótico, mas isso deveria ter sido feito poucas horas após o
trauma, quanto antes melhor. Ele chegou aqui quase dezesseis horas depois, com
a lesão toda suja de terra, e se infeccionar vamos precisar transferir
para um hospital, talvez até amputar a perna. Precisamos chamar a
família dele, Fernanda. Ele certamente ficaria feliz de vê"-los aqui, e
teremos dias difíceis pela frente.''

``Vamos deixar isso para amanhã.''

Fernanda então se retirou e, depois de um tempo, voltou com uma garrafa
de água e algumas raízes cozidas para Theo comer. Ele já estava dormindo
sobre o balcão, ela colocou tudo ao seu lado e foi embora.

\asterisc

%\emph{Marenostrum II -- o índio acidentado piora}

``Tá vindo, tá vindo.''

Theo acordou com um susto. Não sabia que horas eram, apenas que estava
no meio da madrugada. Seu paciente suava frio e delirava, gritava
palavras que não faziam sentido nenhum para Theo. Ele colocou as mãos em
sua testa e sentiu que ela estava queimando.

``Preciso te tirar daqui o quanto antes.''

Theo medicou"-o para baixar a febre, fez compressas de água fria e
colocou"-as sobre a testa do homem. Percebendo que a febre cedera um
pouco, foi até o quarto para acordar Fernanda.

``O que foi, Theo?''

``Ele piorou. Preciso que você me ajude a transferi"-lo, Fernanda, ele
precisa ir para um hospital.''

``Mas a essa hora, Theo? Ninguém navega de madrugada, não vamos
conseguir nada agora. Vamos esperar amanhecer.''

Theo voltou para o consultório decepcionado e continuou cuidando de seu
paciente. Deu antibiótico a ele, continuou fazendo compressas de água
fria e refez o curativo de sua perna, que já estava sujo
de uma secreção com cheiro muito forte.

Quando amanheceu, Fernanda foi ver como o índio estava e, chegando no
consultório, Theo estava dormindo ao lado da maca do paciente, que
também dormia. Mais uma vez, ela ficou surpresa com a atitude de Theo e,
não querendo acordá"-lo, saiu sozinha para conseguir um barco.

Quando voltou, Theo acordara e, visivelmente cansado e abatido, esperava
ansioso por notícias.

``E aí, Fernanda? Conseguiu o barco?''

``Theo, não vai dar para levá"-lo.''

``Como assim não vai dar?''

``Não tem barco, Theo.''

Theo não acreditou no que ouvira. Sabia que se não levasse o índio logo
para um hospital, ele não resistiria.

``Como assim não tem barco? A gente arranja, Fernanda, a gente dá um
jeito. Mas precisamos levar ele para um hospital, e tem de ser agora.''

``Não dá, Theo. O Capelão não está aqui, ele teve de ir para a cidade
para fazer outros serviços. O único motor 50 daqui é dele, só sobraram
rabetas, levaria 5 dias pra chegar lá embaixo. Pedi ajuda pelo rádio,
mas acho que não vai rolar tão cedo\ldots{}''

Theo colocou as mãos nos ombros de Fernanda e encarou seus olhos de um
jeito que ela sentiu medo.

``Fernanda, tem de rolar. Ou ele vai morrer.''

Sem saber o que dizer, Fernanda ficou em silêncio. Olhou para o homem e
viu que ele estava delirando, e então saiu para tentar arranjar um
transporte para eles.

Theo, nervoso, andava de um lado para o outro enquanto esperava,
preocupado com o paciente que ia ficando cada vez mais pálido. Repôs a
bolsa de soro e massageou o pé machucado para tentar estimular sua
circulação de sangue.

Diversos índios passavam na sala
para ver o paciente. Alguns entravam, olhavam e saiam, com uma
curiosidade meio infantil.

Theo continuava ali, ao lado dele e sem
arredar o pé, tomando as providências necessárias para mantê"-lo vivo
enquanto não chegava o barco que não estava vindo.

\asterisc

%\emph{Marenostrum II - Fernanda retorna com notícias sobre o barco}

``Finalmente, Fernanda. Por que você demorou tanto?''

Theo estava novamente trocando o curativo da perna do paciente quando
Fernanda chegou. Estava com um cheiro insuportável, inchada,
escura e com baixa circulação de sangue. Theo trocava os curativos para
evitar encarar a realidade: a perna estava cada vez pior, o sangue
não estava fluindo e ele sabia que era uma questão de tempo para que
aquilo saísse do controle, causando a perda da perna e a morte do
paciente.

``Estou há quase uma hora no rádio, Theo.''

Pela expressão de Fernanda, Theo se preparou para as notícias que
viriam.

``E aí?''

``Consegui um barco, deve chegar no final do dia.''

``Só no final do dia? Fernanda, eu não sei mais quanto tempo ele
consegue aguentar. Ele está séptico, não sei nem se poderia salvá"-lo se
estivesse em um grande hospital.''

``Desculpe não poder ajudar mais, Theo. Aqui na Amazônia as coisas têm
seu ritmo próprio, nem sempre é fácil.''

Theo estava se segurando há tempos, mas não aguentou a pressão e ficou
furioso.

``Fernanda, o que eu estou fazendo aqui? Com um paciente morrendo nas
mãos, no meio da Amazônia em uma aldeia e sem poder fazer nada por
ele!''

Percebendo o quanto Theo estava transtornado, Fernanda colocou as mãos
em seus ombros.

``Você não tem culpa, Theo. Você está dando o seu melhor. Por que você
não sai um pouco, vai tomar um banho no rio, comer alguma coisa? Eu fico
aqui com ele, depois você volta.''

``Não, Fernanda, pode ir você. Eu vou ficar aqui ao lado dele até o fim,
seja lá qual for.''

\asterisc

%\emph{Novo - Marenostrum II - 34,5 - Jonas explica a arma e Senhor \versal{D.} desconfia que tem algo errado}

``Que filho da puta! Este cara é burro demais.''

Jonas se encolhia de medo toda vez que via Senhor \versal{D.} nesse estado de
humor.

``Como ele pôde deixar os outros escaparem? Eu já não falei que quero
todos destruídos? Todos, sem exceção.''

Senhor \versal{D.} olhava o mapa tentando imaginar onde estavam os espiões, mas
sabia que seria mais difícil encontrá"-los agora que estavam prevenidos.

``Jonas!''

O funcionário deu um pulo ao ouvir o grito.

``Você também é um idiota. Não está cumprindo seu papel.''

``Mas senhor, eu fiz tudo o que o devia ser feito.''

``Não interessa, não funcionou. Faça com que funcione.''

Jonas respirou fundo e torceu para o chefe lhe ouvisse.

``Senhor, veja. Eu fiz tudo exatamente como devia ser feito, revi todos
os cálculos com muita precisão. Foi perfeito.''

``Se tivesse sido perfeito, estaria funcionando.''

``É perfeito, senhor, o aparelho identifica as ondas cerebrais dele e
produz artificialmente ondas semelhantes que se adicionam às dele,
amplificando o resultado, desta maneira ele é invencível.''

``Não foi o suficiente, imbecil.''

``O mais legal é que colocamos uma inteligência artificial no aparelho, o
sistema evolui com o tempo, já consegue prever mais de 80\% das ondas
dele antes mesmo que ocorram. Ele está ganhando poder, senhor, veja como
está forte.''

``Amplifique mais''

``Senhor, não podemos. O senhor se lembra do colapso da Tacoma Bridge, a
ponte que caiu por causa do vento?''

Jonas ligou a tela de seu computador, quase saltitante diante de tantos
números e do plano perfeito que ajudara a criar. Então abriu o vídeo em
que a ponte balançava como se fosse feita de elástico.

``Veja, a ponte está tão mole que não parece feita de concreto, mas é.
Aqui, no caso da ponte, por um acaso da natureza o vento entrou na mesma
frequência do material da ponte, do mesmo jeito que estamos fazendo com
o cérebro dele. A ponte foi vibrando com uma amplitude cada vez
maior.''

De repente, a ponte que aparecia no vídeo se quebrou e ficou totalmente
destruída.

``Chega uma hora que o material não aguenta, pois há um limite. É o que
pode acontecer com nosso guerreiro se forçarmos demais a sua mente.''

``Ele não parece estar no limite. Se estivesse, já estaria produzindo
armas muito mais poderosas e não deixaria ninguém escapar. Você tem
certeza que esta máquina está calibrada corretamente?''

``Sim, senhor. A frequência não muda, ele está em um ambiente totalmente
controlado.''

``Faça um teste.''

``Não é necessário, senhor. Já fizemos testes e está tudo sob controle.''

``Eu disse para fazer um teste, idiota. Você não é pago para
retrucar.''

``Desculpe, senhor. Isso vai levar um tempo.''

Mesmo contrariado, o Senhor \versal{D.} pareceu satisfeito com a
explicação.

``Ótimo. Tenho uma reunião agora, mas quando voltar quero um relatório
completo. Vamos ver se você fez tudo tão perfeito quanto pensa.''

\asterisc

%\emph{Marenostrum II -- Fernanda chega para avisar que o barco chegou}

Já estava quase de noite quando Fernanda entrou na sala de Theo trazendo
um prato com mandioca e peixe.

``Olha, Theo, a Osmarina mandou isso para você. Come, você precisa se
alimentar.''

Theo estava se sentindo fraco mas, mesmo sem se alimentar por horas, não
estava com apetite. Com a insistência de Fernanda, no entanto, ele
começou a comer o prato que Osmarina lhe mandara, forçando para a comida
descer.

``E o barco, Fernanda?''

``Nós conseguimos, Theo.''

Theo largou o prato e, sem esconder o entusiasmo, começou a preparar as
coisas para transferir o paciente.

``Espera, Theo. Tem um problema.''

``Como assim? O que foi agora?''

``Não tem gasolina suficiente.''

``Você só pode estar brincando.''

``Nós só temos gasolina para ir, não temos para voltar.''

Theo não
conseguia acreditar no que Fernanda falava para ele.

``Nós não podemos
descer o rio se depois não dá para voltar, entende?''

``Não pode ser, Fernanda, a gente dá um jeito. Na volta a gente vê,
podemos voltar remando. O importante agora é que a gente leve ele para o
hospital o mais rápido possível.''

``Theo, o hospital fica a três dias de voadeira daqui. Não dá para
voltar remando. Eu estou tentando conseguir a gasolina.''

``A vida não espera, Fernanda.''

\asterisc

%\emph{Novo - Marenostrum II - 35,5 - Jonas descobre que a frequência do Marc mudou}

``Não é possível, como pode ser?''

Jonas estava transtornado olhando a onda que aparecia no monitor do
aparelho. Sem saber o que fazer, revirava os papéis cada vez mais nervoso
e, mesmo assim, não conseguia encontrar uma explicação para os novos
resultados que ele aferia. O que poderia estar causando isso? A porta da
sala se abriu.

``Eu sabia.''

Só de olhar a cara de Jonas, Senhor \versal{D.} percebeu que seu assistente havia
encontrado algo errado.

``Eu falei que você tinha que rever os dados! Seu estúpido, você não
consegue fazer nada sozinho, mesmo. Se não fosse por mim, você estaria
enterrado naquele laboratório, você jamais daria certo sozinho.''

``Eu não sei o que aconteceu, senhor, está tudo sob controle. A casa
está totalmente controlada. Escrevemos o roteiro perfeito, usamos o
sentimento de culpa, demos a ele um trabalho, perspectiva, uma
companhia, o desejo de ter um filho. Controlamos tudo, da temperatura do
ar até a textura da areia sob os pés dele.''

``Não pode ser tudo, seu idiota. Você errou, está esquecendo algo.''

Por um momento, Jonas saiu de si. Era pressão demais, e o subalterno
cresceu e se impôs:

``Senhor, não existe nenhum estudo sobre este assunto que eu não tenha
revisado. Ninguém conhece tão bem o funcionamento da mente humana quanto
eu.

Estes dados não podem ser corretos! Ele vinha sob total submissão, qual
parâmetro se alterou?'', perguntou"-se, mais para si mesmo do que para seu
chefe.

Senhor \versal{D.} percebeu Jonas literalmente diminuir. Sua voz se abrandou
novamente e voltou a sua postura taciturna, olhando os livros em sua
estante e repassando em sua mente todas as possibilidades. Porém, não
conseguia encontrar o que poderia ter dado errado.

``Pare de pensar, idiota, e comece a agir. Regule agora mesmo esse
equipamento e amplifique as ondas, ele não pode sonhar por si só. Ele
tem que pensar com a gente e se a gente deixar. Faça"-o matar todos, quanto
antes melhor.''

``Tudo bem, senhor.''

``Você me fez perder tempo e dinheiro, seu estúpido. Esse mês vou
descontar metade do seu salário.''

Jonas olhava"-o assustado, mas não ousou dizer uma palavra.

``E ouça"-me bem, é a última chance que você vai ter para me provar que
vale cada centavo que estou investindo no seu trabalho.''

Saiu da sala e deixou para trás Jonas trêmulo e enxugando os olhos.

\asterisc

%\emph{Marenostrum II -- O índio morre}

Theo estava em sua sala, ao lado do paciente, quando percebeu que ele
começou a ficar com a respiração agônica, como um peixe fora
d'água. Ele sabia que aqueles sintomas indicavam que o corpo do paciente
estava falhando. Aquele homem estava prestes a morrer e não havia mais nada que
Theo pudesse fazer para impedir.

Se aproximou da maca, e sentou ao lado do moribundo. Theo estava triste pela
partida do homem, mas curiosamente se sentia bem, porque sabia que tinha feito
tudo o que podia para salvá"-lo e que, por mais que quisesse e tentasse,
algumas coisas estavam fora de seu controle.

Então Theo se levantou e, percebendo a agonia do paciente, colocou as
mãos nos seus ombros, fechou os olhos e se concentrou. Em poucos
segundos, se viu em uma maravilhosa curva de rio, no remanso das águas
iluminadas pelo pôr do sol. Ao seu lado, o índio que ele vinha cuidando
caminhava ofegante de um lado para o outro, com as mãos na cabeça,
visivelmente preocupado. Até que ele se ajoelhou no chão e caiu em
prantos.

``Por que você está assim?''

``Eu não consigo respirar, doutor, meu corpo não me obedece. Cadê a
minha família? Eu quero ver meu filho.''

``Não precisa se preocupar, está tudo bem. Você só precisa se acalmar.''

``Mas doutor, eu estou me sentindo muito mal, não sei mais o que
fazer.''

``Fica tranquilo, está tudo bem. Encosta nessa pedra, descansa um
pouco.''

O homem levantou e voltou a caminhar. Theo tentava acalmá"-lo de
todas as formas, mas nada que ele falava deixava"-o mais tranquilo, até
que teve uma ideia.

``Como você se chama?''

``Meu nome é Carú.''

``Carú, pense nos momentos mais felizes da sua vida.''

Carú ficou em silêncio, parecia estar se concentrando quando as águas do
rio se transformaram em uma grande oca, iluminada por pequenas
fogueiras. Os dois viram Carú ajoelhado no chão, pronunciando algumas
palavras como se estivesse cantando e rezando ao mesmo tempo.

``Onde nós estamos?''

``São as suas memórias, Carú. E aquele ali é você.''

Como em um filme, a cena continuava. Quando uma índia mais velha trouxe
uma trouxinha com um bebê, ele se levantou, correu em sua direção e,
quando viu aquela pequena criança, começou a chorar de alegria.

``Olha, doutor, esse foi o dia em que meu filho nasceu.''

Rapidamente a cena se modificou e Theo se viu no meio de uma caçada.
Carú e seu filho correndo atrás de uma grande anta, longos cipós
marcavam o caminho, balançando enquanto o animal tentava fugir de seus
perseguidores.

Uma flecha não é páreo para o maior mamífero da América do Sul, mas os
índios colocam longas fitas de cipós em cada flecha, e assim, seguindo
os cipós, os caçadores vão alvejando uma após outra flecha e o
animal acaba sucumbindo aos caçadores, pai e filho juntos.

Carú assistia a cena ao lado de Theo chorando.

``Meu filho agora virou um grande guerreiro, aprendeu tudo o que eu
ensinei.''

``Parece que você fez bastante coisa, Carú.''

Carú estava mais calmo e Theo achou que faria bem para ele saber sobre o
mare nostrum.

``Tudo isso que você viu, Carú, está guardado no coração das pessoas que
te amam. E aqui, onde estamos agora, são as memórias delas, que nunca
vão se apagar.''

``Mas doutor, o que vai ser de mim?''

``Pode ficar tranquilo e relaxar, você vai para um lugar muito melhor
que o que conhece hoje.''

Theo fez uma rede se materializar na frente deles, pendurada em uma
árvore.

``Deita um pouco, Carú. Fecha os olhos e descansa um pouco.''

Caru fez como Theo havia pedido, se deitou na rede e logo adormeceu.

Theo então voltou para seu consultório, abriu os olhos e se viu diante
do paciente. Mediu seus batimentos e constatou que eles haviam parado.

\asterisc

%\emph{Marenostrum II -- 37 - o índio morto é levado para sua aldeia}

Quando Theo abriu os olhos, pensou ter
ouvido o grito de uma índia vindo de longe, um grito de profundo
desespero e dor. Antes mesmo que ele avisasse a tribo sobre a morte do
índio, alguns homens chegaram em seu consultório, junto com Fernanda.

``Precisamos levá"-lo para a aldeia dele, Theo, fica aqui perto. Podemos
usar o barco que conseguimos.''

``Tudo bem, Fernanda, eu posso ajudar vocês a colocá"-lo no barco.''

``Por que você não vai com a gente?''

Theo ficou incomodado com aquela pergunta, ele nunca havia acompanhado o
enterro de um paciente seu e dado condolências para seus familiares.
Pelo contrário, no hospital onde estudou, essa tarefa ficava para os
médicos mais experientes.

``Melhor não, Fernanda. Prefiro ficar aqui.''

``Mas Theo, você esteve ao lado dele até o fim. Não tem por que você não
nos ajudar a levar ele para sua família.''

Theo se lembrou do dia em que encontrou um de seus professores saindo do
hospital.

``Olá, doutor Salim.''

``Oi, Theo. Você vai também?''

``Aonde, professor?''

``Ao velório do Carlos. Vou agora e devo voltar daqui a pouco.''

``O Carlos, nosso paciente?''

Theo havia passado muitas horas cuidando daquele paciente na semana
anterior, sob a orientação do professor Salim.

``Sim, Theo.''

``Mas como o senhor consegue ir ao velório de um paciente que morreu em
suas mãos, doutor? Encarar os familiares dele?''

``Theo, é feliz o médico que pode cumprimentar o familiar do seu
paciente que morreu, porque significa que ele fez seu trabalho
direito.''

Lembrando daquela frase, Theo achou que Fernanda tinha razão e poderia
ajudá"-la a levar o corpo.

O dia estava despertando, eles atravessaram o rio e chegaram na aldeia
vizinha. Diversos índios esperavam por eles, já com os corpos pintados
com uma tinta preta e preparando uma grande fogueira. Theo ficou curioso
sobre os rituais de morte naquela etnia, mas preferiu respeitar o peso
da situação e passou pelos preparativos do funeral sem nem desviar o
olhar.

``Essa é a viúva, Theo.''

Theo reconheceu a mulher que havia visto nas memórias de Carú, e
cumprimentou"-a. Ao seu lado, um adolescente alto se esforçava para
manter uma fisionomia madura apesar de estar clara sua tristeza e
insegurança quanto ao futuro. Theo cumprimentou"-os, e, mesmo sem nenhuma
comunicação verbal, eles pareciam compreender o esforço que o médico
tinha feito pelo seu ente querido. Apesar disso, apenas
retribuíram o cumprimento e saíram.

Sem mais nada o que fazer ali, Theo e Fernanda voltaram para a aldeia e,
ainda no caminho de volta, ouviram os batuques da cerimônia de
despedida. Theo se emocionou, lembrando das últimas 24 horas vividas ao
lado de Carú, do quanto havia tentado salvar sua vida.

Pensou em seu mestre e percebeu a sabedoria de ter sido enviado para lá.
Ele precisava ter vivido aquela experiência para ir mais fundo no
significado de ser médico e de ser navegador, e finalmente se sentir um
indivíduo pleno.

Exausto, há quase dois dias sem dormir, Theo chegou ao polo"-base e foi
direto para sua rede descansar.

No outro dia acordou já com o sol alto. Quando chegou ao refeitório,
notou que o café da manhã estava caprichado, mamão, bananas e água de
coco.

``Fernanda! Hoje é dia de festa?'', perguntou, mesmo sem saber onde
estava a dentista.

Após o café, quando foi para sua sala de atendimento, a surpresa: a
rádio floresta tinha anunciado sua performance dos dias anteriores, e uma
grande fila de pacientes o aguardava, ansiosos por seu médico.

\asterisc

%\emph{Novo - Marenostrum II - 38 - Marc mata um russo}

Marc estava entediado, jogava dardos com a pretensão de passar o tempo,
mas curiosamente o tempo parecia não passar. Cada dia igual ao outro, um
vazio e uma repetição; como chamava mesmo aquele filme, o da marmota, em
que o cara acorda sempre no mesmo dia? Agora ele gritava enquanto
atirava as peças na parede, porque sentia uma grande raiva, uma angústia
desesperada, um impulso de brigar ou matar alguém. Era uma sensação que ele não
sabia de onde vinha, como se o propósito da vida não existisse. O
desconforto era tal que ele quase ansiava por não sentir mais nada, assim
ficaria aliviado.

Por sorte o alarme soou, e a tela de \versal{LCD} sinalizou que estava sendo
chamado para uma missão. Marc deu um urro, mais animal que
humano, e saiu correndo para o porão, sem nem esperar Teresa voltar para
se despedir. Onde andava Teresa, mesmo? Ele entrou na sala secreta,
sentou em sua cadeira, puxou o capacete cuidando para não soltar nenhum
eletrodo, respirou fundo de maneira controlada e firme, seguindo
rigorosamente seu treinamento, passo a passo, até que, enfim, estava
pairando no escuro enquanto recebia as instruções.

``Você enfrentará um grupo de espiões. Cuidado com o mais velho,
possivelmente é o treinador, é o mais experiente e tem muita energia.
Defenda nosso mundo desta ameaça, defenda os segredos do nosso país
destes espiões, piedade é para fracos.''

Em um instante, se viu em meio a montanhas brancas cobertas de neve;
Sempre que era catapultado para uma realidade paralela ficava alguns
minutos enjoado e um pouco tonto. Se encolheu e ficou quieto enquanto a
voz do Senhor \versal{D.} ainda reverberava em sua mente, economizando energia,
pronto para aflorar o instinto animal e cruel de sua personalidade.

Avistou adiante, no meio da neve, uma grande cabana de madeira,
tipicamente russa e, reconhecendo"-a como seu alvo, se aproximou sem
revelar sua presença.

Ao chegar, viu pelas janelas algumas pessoas deitadas no chão meditando,
concentrados. Em uma posição impossível, apoiado em uma perna quase
levitando, havia aquele que reconheceu como o mestre. Sua energia
extrapolava seu corpo e, para Marc, era perfeitamente visível algo como uma
nuvem semi"-sólida pairando ao redor dele.

Focando os olhos em suas próprias mãos, Marc materializou um sabre
militar, que achou adequado para a situação; a escolha das armas
dependia de seu humor e estado de espírito. A arma, poderosa, tinha um
leve brilho azul que refletia no ambiente.

Abriu a porta sem tocar na
maçaneta e entrou no grande salão. Parou e esperou.
O seu treinamento mandava que chegasse destruindo tudo, mas algo de
cavalheiro o impedia de fazer assim, afinal, quem saberia? Logo todos
ali estariam mortos, mesmo, e só ele restaria em pé.

Cada uma das pessoas presentes foi tomando consciência da presença dele
e produzindo sua própria arma, não sem antes ficarem parados admirando a
situação por alguns segundos, como que regidos por um maestro.

Em um instante, tudo se moveu, e o combate começou. Apesar de corajosos,
os jovens aprendizes não eram páreo para Marc. Não bastasse sua
velocidade e força muito superiores, o sabre materializado por ele
explodia em fagulhas azuis as armas dos oponentes, não deixando muita
opção para se defenderem de um ataque tão brutal. Cada um que tentava
medir forças caía e sucumbia diante de tanto poder.

Ante tamanha destruição, o mestre em um movimento de seu braço
interrompeu a luta, se posicionando entre Marc e os seus discípulos que
restaram. A aura de luz ao seu redor ficou ainda mais forte, e, após
materializar um florete, fez uma reverência e partiu para o confronto.
Eles trocaram alguns golpes estudando um ao outro, até que Marc percebeu
que seu sabre não destruiria tão fácil a arma deste adversário. Havia
muita energia ali.

A materialização de objetos no mare nostrum depende de foco, vontade e
amplitude de ondas cerebrais, e, por mais que os mestres tentem transformar
o duelo em uma modalidade cavalherística, alguns consideram que não
passa de um confronto similar às disputas de dois cervos pela fêmea,
chocando seus chifres vigorosamente até que um desiste e abre mão do
prêmio e de manter seu \versal{DNA} para a próxima geração. É simples, cada um
materializa sua arma, e aquela que tiver sido gerada e
mantida com mais amplitude das ondas cerebrais rompe a do outro, que é
ferido. Algumas vezes o derrotado volta a produzir outra arma, mas com
menos foco e vontade, já tendo sido derrotado previamente. Grandes
mestres produzem armas muito sólidas e poderosas. Ferimentos neste
ambiente se refletem no corpo físico e podem até levar a sequelas e à
morte.

Lembrou"-se de uma vez em que ele deixou escapar um alvo por não prever a força
da arma que o mais jovem podia materializar. Foi repreendido pelo Senhor
\versal{D.} por isto, mas desta vez ele estava prevenido e não deixaria nada
errado acontecer.

Apertou o botão camuflado no bracelete ornamentado que usava e, em um
segundo, pôde ouvir um som grave e cíclico como se uma grande massa
orbitasse sua cabeça cada vez mais rapidamente. O brilho de sua arma foi
progressivamente aumentando e até irradiava energia de tão potente que
era.

Ele então incidiu sobre o inimigo, que reagiu com uma manobra de esgrima
perfeita e se posicionou para contra"-atacar, mas foi impedido por uma
fraqueza em suas pernas e uma angústia no peito.

``Quem poderá enfrentar tamanho poder? Ele cortou minha arma como de
fosse uma vareta.''

Juntando suas últimas energias, falou na mente de um aprendiz.

``Vá e espalhe a notícia, uma máquina aumenta a amplitude do poder do
nosso inimigo, isto torna suas armas muito mais poderosas que as nossas.
A materialização é potente demais, nenhum de nós deve enfrentá"-lo.''

O discípulo desapareceu no ar e, antes que os outros pudessem fugir,
Marc acabou com eles, suas armas já destruídas em centelhas azuis.

Ao final, Marc olhou para as dezenas de corpos amontoados sem muita
emoção ou qualquer remorso, até que olhou para o canto da sala e
percebeu que um dos corpos estirados era de um menino bem jovem, uma
criança. Ele se lembrou do filho que estava para nascer e começou a
tremer.

``As ondas estão instáveis, aumenta a sedação.''

Marc sentiu algo gelado em seu braço, como se sua veia estivesse
congelando e algo se espalhasse por todo seu corpo, anestesiando"-o aos
poucos, até cair de joelhos e apagar.

\asterisc

%\emph{Novo - Marenostrum II - 39 - Fred questiona o que está acontecendo}

Fred ficou assustado quando viu Marc estirado sobre o chão da sala da
casa, e correu em sua direção para tentar despertá"-lo, em vão. Havia um rastro
de sangue sobre o mármore travertino. Avaliou rapidamente o amigo e
percebeu que, a despeito da mancha abundante de sangue, não havia
ferimento grave, era apenas um sangramento nasal.

Curiosamente, seu amigo respirava de forma serena e contínua.

Passado o susto, resolveu seguir as marcas de sangue no piso até
chegar na escada que levava ao porão. Sabia que aquela casa era
muito estranha, mas a curiosidade pulsava. Olhou para os lados, desceu
lentamente os degraus, abriu a porta e, quando ia avançar para a
sala escura, foi surpreendido.

``O que você está fazendo?''

Fred tomou um susto e, quando olhou para trás, Teresa estava atrás dele.
Ela mantinha uma expressão distante e apática, apesar de questioná"-lo.

Essa foi a primeira vez que ele pôde vê"-la de perto, já que a mulher
estava sempre caminhando na praia. Fred ficou impressionado ao ver
seu rosto. Ela era magra, até bonita, mas parecia vazia e sem vida.

``Que bom que a senhora chegou. O seu marido está machucado, me ajude a
levá"-lo para o hospital.''

Lágrimas começaram a brotar de seus olhos, mas esta foi a única reação
de Teresa.

``Por que a senhora não responde? Precisamos de ajuda, tem alguém nesse
porão, dá para buscar ajuda por aqui?''

Teresa virou as costas e subiu para o seu quarto, Fred foi ficando
irritado e a segurou.

``A senhora tem de fazer alguma coisa, a senhora está esperando um filho
dele, não está? Aliás, cadê a sua barriga? E esse feijão que não cresce?
Eu preciso saber o que está acontecendo.''

Enquanto Fred falava, Teresa olhava para ele, um olhar triste e sofrido
como de alguém que pedia ajuda.

Angustiado, Fred voltou para a sala, olhou para Marc deitado e
vagarosamente foi indo embora, como se estivesse acordando de um sonho
ruim. Sabia que aquilo não estava certo e que não era natural, e foi
andando pela areia pensando em como poderia ajudar aquelas pessoas tão
tristes.

\asterisc

%\emph{Novo - Marenostrum II - 40 - Theo está modificado e vai caçar com os índios}

A relação entre Theo e a comunidade já não era mais a mesma, ele se
sentia outra pessoa. Pequenas coisas sem importância, como tomar banho
nu no rio, brincar com as crianças ou até entrar nas casas das pessoas e
participar das suas vidas agora já eram parte do seu dia a dia.

Picún estava sempre grudado em seu colo, os pezinhos e o bumbum já não
estavam mais infestados de tungas, assim como de outros índios que iam
sendo atendidos pelo novo amigo que a comunidade ganhou. Cada um ia
tendo sua enfermidade amenizada, e ele mesmo, por sua vez, vagarosamente
ia se sentindo cada vez melhor, à medida em que entendia e aceitava a
sabedoria daquele povo.

Nem tudo estava perfeito na aldeia, o terreno continuava contaminado de
lixo, os cachorros continuavam fazendo cocô onde não deviam e as
crianças continuavam sendo crianças, mas, mesmo assim, Theo estava em
paz, porque entendeu que não estava lá para fazer com que os indígenas
vivessem como ele. O encontro entre as culturas podia ajudar a resolver
alguns problemas deles, mas o oposto também era verdade. Ele estava ali
para dar e receber, fazia o seu melhor e isso bastava.

Naquela manhã, estava voltando do rio para sua sala quando foi
surpreendido por um cachorro que correu em sua direção e começou a fazer
festa em cima dele, logo seguido por um grupo de índios jovens, três
homens pintados que seguravam arcos, flechas e facas em suas mãos.

``Yá, para com isso.''

O cachorro se afastou, mas continuou correndo em círculos e saltitando
sobre os homens, aparentemente entusiasmado com as armas que eles carregavam.

``Theo, vamos caçar?''

O índio estendeu um arco para ele que, honrado, abriu um grande sorriso
e agradeceu. O antigo moleque que chegara ali teria pensado nos perigos
que uma caçada representa, e provavelmente até negado o convite. O Theo
que estava ali, porém, era outro, semelhante aos homens do clã, não só
pela pele morena já resistente ao sol, mas também pela coragem que
parecia crescer com sua barba.

``Vai ser uma caçada longa, um clã vizinho vai vir nos visitar e
precisamos ter carne suficiente para todos.''

``Legal!''

``Pega tua rede.''

``E o que mais?''

``Não precisa de mais nada, já estamos levando todo o resto.''

Correu para pegar o mínimo necessário, avisou Fernanda e Osmarina e
voltou rapidamente.

O grupo, animado, entrou na mata e Theo tentava acompanhá"-los, andando o mais
rápido que conseguia, mas logo ia sendo deixado para trás, eles
pareciam desenhados para se locomover na mata. Mais baixos e atarracados
que eram, se locomoviam tão rápido e tão focados que pareciam não
enxergar mais nada a não ser o objetivo à frente. Muitas vezes Theo se via
atrasado, apenas na companhia de Yá. Preocupado, acelerava o passo e
tentava alcançá"-los novamente, mas a trilha era cheia de obstáculos,
troncos e raízes e ele tinha dificuldade para atravessá"-los. Ofegante,
após vencer alguns arbustos, ele não viu mais seus guias e seu coração
disparou, desesperado por não saber a direção que deveria seguir.

Olhou para um lado, olhou para o outro, e nada. Tentou ouvir o barulho
dos índios, mas eles caminhavam como se vestissem pantufas nos pés e
tudo o que conseguia ouvir eram os cantos dos pássaros e dos inúmeros
bichos que pareciam se esconder, prestes a atacá"-lo. Yá apenas o
encarava parado, esperando ele dar o primeiro passo para o seguir, mas
Theo permanecia imóvel.

Depois de alguns segundos de hesitação, começou a andar de novo em
direção a um mato amassado que parecia ter sido o caminho por onde
seguiram seus companheiros de caçada. Foi prestando muita atenção nos
detalhes, em cada árvore e pedra do percurso, assim poderia voltar caso
precisasse. Após menos de um minuto de caminhada, encontrou uma grande
castanheira caída na floresta, uma árvore enorme que parecia um
dinossauro abatido e que ao cair derrubara diversas outras árvores ao
redor, criando um obstáculo enorme que bloqueava o caminho da floresta.
Theo percebeu que eles não teriam conseguido seguir por ali e resolveu
voltar.

``Vamos, Yá!''

Com dificuldade, Theo conseguiu voltar ao local que imaginava ter sido o
último seguido pelos colegas, mas mesmo assim sabia que estava em uma
enrascada.

Apesar de assustado e ansioso, ele resolveu se acalmar, e se lembrou de
quando começou a se envolver com esse universo e a se preparar para ir
para a floresta amazônica. Se recordou das histórias que ouviu sobre
pessoas que se perderam na mata e acabaram mortas de fome no ``deserto
verde'', mas não quis ser tomado por esse pessimismo e afastou esses
pensamentos.

Temeroso com o que poderia acontecer, teve a ideia de se utilizar da
navegação para conseguir ajuda. Rapidamente se conectou ao mare nostrum,
e logo estava com os colegas caçadores. Eles já haviam notado sua
ausência e estavam discutindo o que fazer para encontrá"-lo.

``Ele é da cidade, não conseguirá chegar aqui sozinho, você sabe que tem
até parente que cresceu aqui e não se localiza no mato.''

``Mas se a gente voltar não vai achar ele, os caminhos se multiplicam, é
impossível.''

``Vamos esperar um pouco, os caminhos afunilam por causa da montanha,
ele vai ter que passar por aqui.''

Apesar de tentar repetidas vezes, Theo não conseguiu se comunicar com
eles. Entretanto, teve a ideia de tentar aprender a se localizar no mato
e seguir uma trilha utilizando"-se das memórias e dos conhecimentos dos
seus colegas de caçada. Poderia dar certo.

Se concentrou em Puru, o mais velho deles e, vasculhando suas memórias,
encontrou lembranças antigas de quando ele estava aprendendo a se virar
na floresta, ensinado por um ancião, muito provavelmente seu avô.

``Só um índio muito experiente consegue se achar na floresta, ninguém
conhece a mata toda. Para andar na floresta pensamos em referências e
onde queremos chegar, o caminho é feito pela soma de pequenos trechos
ligando estas referências, o que se leva uma vida para aprender. Apesar
de parecer rica em água e alimentos, aqueles que não conhecem os
segredos de como sobreviver no mato sofrem horrores quando perdidos nas
entranhas da floresta. É realmente muito difícil caçar e pescar nesse
mato, os animais são raros e muito bem camuflados. Obter frutas ou
vegetais comestíveis também não é fácil, as árvores são muito altas por
conta da competição pela luz do sol, o que leva as frutas para dezenas
de metros de altura sobre a floresta, acima da possibilidade de servir
como refeição. O chão também não ajuda, é coberto de folhas em constante
estado de decomposição e, com a umidade, se torna uma grande matéria
orgânica, um ambiente perfeito para fungos, bactérias e insetos.''

Theo percebeu que não aprenderia a sobreviver na floresta simplesmente
investigando as memórias de Puru, leva"-se uma vida para se adaptar à
força e ao poder da natureza.

Assim que saiu de seu transe, foi tomado por uma rara melancolia e um
vazio, sentou"-se em um tronco olhando para a floresta sem saber o que
fazer.

``É Yá, a gente se deu mal.''

Respirou fundo e enxergou a beleza do ponto que havia parado. De toda a
floresta amazônica, ele acabou se perdendo em um lugar lindo, a luz do
sol filtrada pelas folhas criava um efeito incrível quando se projetava
na umidade que a floresta devolve para o ar. Theo pensou em como era
sortudo por estar ali.

Olhou para o cachorro que teimava em babar na perna dele e se levantou
em um pulo. Finalmente, se sentiu parte daquilo tudo, não era um
estranho na floresta.

``Yá, vamos embora, me leva até eles, vamos!''

O cachorro entendeu o que Theo queria e saiu em disparada. Depois de um
bom trecho correndo pelo mato, Theo viu muito à frente o grupo de
índios, seus colegas de caçada, avançando pela floresta.

``Ei, me espera!''

Seu grito foi tão forte quanto sua angústia. Um deles virou na direção
de Theo e, assim que o viu, gritou para os outros, que olharam para
Theo em festa.

``Pô, cara, vocês foram rápidos demais.''

Retomaram a marcha, mas desta vez um pouco mais devagar, até que
finalmente chegaram onde queriam: um descampado, junto a um tranquilo
igarapé. Tomaram banho de rio, beberam água e descansaram da jornada até
ali.

``Theo, vamos nos dividir. Eles vão entrar na mata com o Yá e você fica
aqui comigo.''

Theo e Jaci se abaixaram entre umas pedras, com a atenção voltada à água
plácida do pequeno riacho. Theo não compreendeu muito bem\ldots{}

``O que estamos fazendo?''

Jaci o olhou feio, pedindo para se calar. Ele continuou ali, imóvel, sem
mexer um dedo por horas que se passaram. Agoniado, teve de virar estátua, a
despeito do impulso de se livrar dos mosquitos que se aproveitavam
deles. Finalmente, ouviram um barulho de algo se mexendo por ali. Theo
virou e viu uma paca se aproximando vagarosamente, sem perceber a presença dos
caçadores. É um animal muito
arisco e assustado, mas, após alguns segundos de indecisão, começou a
beber água. Jaci então rapidamente pegou a flecha, posicionou"-a no arco
e, antes mesmo que o bicho pudesse reagir, o matou.

O índio se levantou e comemorou cantando um grito de guerra.

Jaci ensinou Theo a carnear o animal. Naquele mundo, quase tudo
é consumido, mas tem que ser retirado, lavado e tratado para que não
estrague. Pois, com o calor, os insetos e a umidade, tudo apodrece em
um minuto.

``Quando eles chegarem vamos fazer uma fogueira para muquiar esta carne e
as outras que eles vão trazer do mato?''

Theo não fazia ideia do que era muquiar.

``A gente faz um fogueira e posiciona um suporte de madeira, mas tem que
ser feito de varas verdes para que não se queime. Aí fazemos a fumaça da
fogueira passar pela carne, ele seca e fica saborosa, com um pouco de
gosto do fumo da madeira, assim ela dura mais.''

Este processo de defumar a carne encantou Theo. Como estas
comunidades foram adquirindo tanta sabedoria? Era muito interessante ver
como eles resolviam os diversos problemas e de forma simples.

Jaci rapidamente montou uma pequena estrutura de madeira que conseguia
suportar o peso da paca e ainda sobrava espaço, e em mais um minuto
estava montando sua rede em duas árvores próximas do igarapé. Deitou
para descansar enquanto aguardava a chegada dos outros caçadores, mas, vendo
a dificuldade de Theo em montar sua rede, levantou e o ajudou na
tarefa. Montar a rede nas vigas do telhado já não é fácil, mas entre árvores é
bem pior.

Eles descansaram por um bom tempo, até que foram acordados pelos latidos de
Yá, que veio correndo da mata. Atrás dele, os outros dois caçadores
carregavam nos ombros três varas de madeira envolvendo um jabuti vivo
amarrado com cipós e, na mão de um deles, um tatu.

Eles chegaram, descansaram a caça na terra e cantaram juntos o mesmo
grito de antes.

Jaci juntou mais alguns pedaços de madeira para começar uma fogueira.
Escolheu uma lenha especial que produzia muita fumaça, para que o
processo de muquiar fosse mais rápido. Theo ficou curioso sobre como
tudo iria acontecer, já havia assistido alguns seriados na \versal{TV} e sabia
que levava horas para fazer fogo apenas com pedras. O índio pegou um
isqueiro de sua sacola, pôs fogo em um pedaço de palha que conseguiu em
um coqueiro ali ao lado e logo a fogueira se acendeu.

Eles assaram e comeram uma parte do tatu, mas deixaram a maior parte para ser
defumada com a paca. O jabuti, coitado, estava vivo e seria transportado
assim até em casa.

Depois de comer, ficaram contando histórias de caçadas e aventuras em
volta da fogueira, até que foram dormir.

No dia seguinte, logo após o sol nascer, Theo foi acordado por Puru.

``Hoje você vai entrar na mata comigo.''

Theo tomou um banho no rio para despertar e entrou na mata com Puru e
Yá. O cão era o grande condutor da caçada, corria disposto pela mata e
eles apenas o seguiam. Até que Yá parou, e Theo viu que ele tentava
desesperadamente entrar em um grande buraco no chão, provavelmente a
toca de algum bicho.

``Muito bem, Yá.''

Puru tirou o cachorro da toca e ele mesmo colocou a mão no buraco,
tentando achar o animal que ali habitava. Após várias tentativas,
conseguiu puxar pelo rabo um enorme tatu e eles comemoraram a caçada.

``Boa, esse é dos grandes! Agora vamos voltar, deixamos várias
armadilhas pela mata, vamos ver se pegamos alguma coisa.''

Passaram por três armadilhas, buracos um pouco rasos na terra e,
dentro deles, dois grandes jabutis tentavam em vão escapar da enrascada
em que estavam.

Puru ensinou Theo a amarrar os jabutis vivos na estrutura de madeira, e
eles retornaram para o acampamento carregando quase vinte quilos nas
costas. Theo ficou exausto, a caçada havia levado muitas horas e já era
tarde quando eles retornaram para o acampamento. Comeram a carne
que havia sido assada no dia anterior e se preparam para voltar, felizes
com o que conseguiram caçar.

Já estava escurecendo, e Theo estava exausto, não estava acostumado a
tanto esforço físico. Já era difícil correr para acompanhá"-los, e à noite
ficou ainda mais.

``Não podemos esperar amanhecer? Não estou conseguindo ver direito.''

``Não dá. Está todo mundo esperando a gente, o clã chega amanhã cedo.''

``Mas é lua cheia, está tudo claro.''

Por várias vezes, Theo pediu para que parassem. A volta estava bem mais
difícil, ele estava cansado e ainda tinha sua cota de peso para ajudar a
levar.

Até que, em um certo ponto, eles pararam. Mas Theo não chegou nem a ficar feliz,
porque logo percebeu que havia algo errado.

Os homens assumiram posição de alerta, com as armas em punho em um
círculo, um de costas para o outro. Theo foi logo puxado para o centro.

``O que está acontecendo?''

``Tem uma onça rodeando a gente.''

Theo olhou para os lados e não viu nada.

``Tem certeza? Eu não tô ouvindo nada.''

``A onça não se deixa ouvir. Mas ela está aqui.''

Os homens ficaram em posição de defesa, prontos para repelir qualquer
ataque, e o tempo foi passando. Theo nunca havia vivido algo parecido.
Ficaram na mesma posição durante boa parte da noite, por horas. De
repente, Yá começou a rosnar e latir, e entrou na mata além da visão dos
caçadores. Para desespero de Theo, ouviu"-se então um som seco como uma
rebatida de um taco de beisebol, um ganido do cão seguido de um silêncio
desesperador.

Todos abaixaram a cabeça com uma expressão fúnebre, brancos, e Theo não
acreditava que tinha acabado de perder seu novo amigo. E ainda, o que
seria dele? Não queria virar comida de onça, lembrou de seu pai, de
sua faculdade, de seus amigos, e pensou se ainda conseguiria vê"-los
novamente.

A carne de Yá possivelmente satisfez o jaguar, já que após alguns
minutos os índios foram relaxando, desfizeram a formação de defesa e se
preparam para continuar andando.

``Ela foi embora. Podemos ir agora.''

Eles seguiram o caminho de volta e chegaram na aldeia com o dia já
raiando. Deram a caça para as mulheres e pegaram algumas frutas para
comer, sentando"-se para descansar. Theo estava intrigado com o que viu
na mata e, apesar do cansaço, puxou conversa.

``Por que vocês não fizeram um rodízio de guarda para conseguirem
descansar durante a noite? Não dava para apenas alguns fazerem a
formação de defesa enquanto os outros dormiam?''

``Theo, a gente caça junto desde muito jovens. Desde sempre um protege
as costas e a vida do outro. A gente aprendeu que o problema de cada um
se reflete no problema do outro, e só quando nos unimos conseguimos
enfrentar qualquer coisa, até mesmo uma onça. Em grupo somos muito mais
fortes.''

Theo imediatamente pensou em sua vida e se deu conta de que estava
sozinho. Sem ninguém para proteger suas costas e ninguém que precisasse
da ajuda dele para se proteger, sentiu uma profunda solidão.

\asterisc

%\emph{Novo - Marenostrum II - 41 - Theo é convidado para o ritual}

``Como Theo estaria se saindo?''

A ideia de enviar o rapaz para a Amazônia para descobrir suas
origens não era nova, seu pai já planejava isso há muito. Parece que
Alcântara até hoje ouvia o velho amigo dizendo:

``Quando este menino crescer vamos mandá"-lo fazer um estágio
lá na floresta. Nada como um pouco de natureza e isolamento para moldar
o caráter de alguém\ldots{} difícil é ele querer voltar.''

E ria enquanto expirava a fumaça do cigarro.

Mas as coisas acabaram acontecendo rápido demais. O grande distúrbio
de energia que Theo provoca no Mare Nostrum o transformava em um alvo
perfeito, e vários navegantes vinham morrendo de forma suspeita.

``Mande ele para mim, aqui na comunidade eu vou ensinar o nosso jeito de
navegar. Acompanhei cada passo dele, e agora já está na hora de
nos conhecer.

Tenho certeza de que ficará protegido aqui na floresta.''

\asterisc

Quando Theo entrou na cozinha, Osmarina cantava uma melodia que ele
parecia já conhecer e, animada, colocava em cumbucas a sopa que havia
preparado.

``Hoje também tem jantar especial?''

``Acertou, rapaz esperto.''

``Vai ter aquele ritual de novo?''

``Isso mesmo. E eu não vou poder ficar de conversa porque a xamã está
esperando o jantar. Até mais tarde.''

Desanimado por saber que não poderia participar da festa, Theo foi para
o quarto deitar na rede e esperar a noite passar, quando Fernanda entrou
toda pintada.

``O que você está fazendo?''

``Ué, esperando o dia amanhecer.''

``Levanta, Theo, já está quase na hora.''

``Na hora do quê, Fernanda?''

``Theo, a xamã te convidou pro ritual!''

``É sério? Eu vou poder participar?''

``Claro né, Theo. Eu iria brincar com uma coisa dessas?''

``Ué, mas o que eu fiz pra ela mudar de ideia?''

``Você fez mais do que imagina, o mundo não é branco e preto, Theo.''

Theo não entendeu, mas mesmo assim ficou super empolgado com o convite.
Vestiu seus colares, um deles com uma medalha que tinha sido de seu pai,
e se dirigiu para a casa do guerreiro com Fernanda.

``Onde a gente está indo?''

``Na casa da Naiara.''

``A mãe do Picún?''

``Isso mesmo.''

No caminho, eles passaram pelas casas e Theo viu diversas mulheres
pintando os homens e seus filhos. Picún veio correndo assim que viu Theo
e ele pegou o menino no colo.

``Theo, essa é a Naiara, a mãe do Picún.''

A índia veio na direção de Theo, que estendeu a mão para cumprimentá"-la,
mas ela o abraçou.

``É assim que vocês brancos mostram que se gostam, né?''

``É, sim'', Theo riu.

``Meu menino gosta muito de você, doutor. Obrigada por cuidar dele.''

Theo colocou o menino no chão e ele saiu correndo para brincar com as
outras crianças.

``Theo, a Naiara vai te pintar.''

``Me pintar? Mas pra quê, Fernanda? Eu não sou índio, não quero ser
pintado.''

``Escuta, você foi convidado para a festa deles! Em Roma, como os
romanos.''

``Ok, ok. Tá, tudo bem.''

Naiara pegou a tinta de jenipapo e foi enfeitando toda a pele de Theo,
até cobri"-la completamente de desenhos.

``Fernanda, o que acontece nesse ritual, hein?''

``Eu não posso falar, mas é uma piração. E quer saber? Mesmo se eu
explicasse, você não ia entender. Mas fica tranquilo que logo logo você
vai saber.''

Para finalizar a preparação de Theo, Naiara amarrou em seus tornozelos
um chocalho feito de sementes.

``Tá pronto.''

``Tô bonito, Fernanda?''

``Tá lindo!''

``Tira uma foto?''

Theo calibrou sua câmera, ergueu as costas e posou. Ficou impressionado
quando viu sua imagem no pequeno monitor do equipamento, era um homem
completamente diferente do que conhecia.

``Uau. Tô bonito, mesmo.''

Eles foram para a Casa do Guerreiro. Os indígenas se aproximavam e
batiam em suas costas.

``Que bom que você vai estar conosco hoje.''

O acolhimento dos índios fez Theo se sentir à vontade. Entrou na
grande oca e viu que alguns deles estavam em pé, em roda, de olhos
fechados, e uma mulher estava sentada em um canto, coberta pela sombra,
provavelmente era a xamã.

Ela logo se levantou e Theo pôde vê"-la melhor, com o corpo todo pintado
e um belo cocar de penas sobre os cabelos acinzentados. Sem ser capaz de
dizer sua idade, viu que ainda era bonita, mas o sol e a vida cobram seu
preço e sua face era marcada pelos maus"-tratos da floresta. A pele
enrugada enquadrava os olhos curvos como arcos, fortes e decididos como
dos que comandam. Theo não conseguia parar de admirá"-la,
surpreendendo"-se ao sentir a força desta mulher antes que ela sequer
abrisse a boca.

O ritual começou e os índios começaram a dançar. Theo os observava e
tentava acompanhar seus passos. Eles batiam tacos de madeira adornados
no chão de terra e pisavam firmes, chacoalhando as sementes presas em
seus pés. A xamã pronunciava algumas palavras e todos repetiam a melodia
que ela cantava, acompanhados por algumas índias que tiravam suaves
notas de suas flautas.

Theo logo aprendeu os passos e a música, e em pouco tempo dançava e
cantava como eles. Era incrível como aquele som o acalmava e arrepiava
cada pelo do seu corpo. O ritual se repetiu durante horas, mas ele não
percebia o tempo passar. Era como se cada palavra e cada passo renovasse
suas energias, e ele ia se sentindo cada vez mais forte.

Quando parecia estar em outra dimensão, a música parou. Todos gritaram,
pararam onde estavam e deram as mãos, menos a xamã, que permanecia em pé
no centro deles, como a grande maestrina do espetáculo. Theo estava tão
encantado com tudo que não conseguiu ver bem o gesto que eles fizeram.
De repente, diversas imagens e vozes começaram a sair dela, eram índios
caçando, crianças brincando, animais voando, gerações e mais gerações
que viveram na tribo e pareciam desfilar diante dos olhos deles.

``Theo, feche os olhos. Venha comungar conosco.''

Aquela voz era familiar para Theo. Ele abriu os olhos e, mesmo vendo a
xamã de olhos fechados, sabia que era ela que falava, por meio de
uma conexão de mentes diferente de tudo o que ele já havia visto. Ele
então fechou os olhos e, pela primeira vez desde que aprendeu a navegar,
não teve de se concentrar, mas foi puxado para a conexão, como se
estivesse sendo sugado por uma grande força para uma grande comunhão de
mentes.

Ele sentiu como se a barreira que o separava dos outros e lhe conferia
individualidade tivesse sido derrubada e agora ele não era um, mas fazia
parte de um clã, um grande grupo de pessoas. Diferente de quando
tentaram entrar em sua mente no mare nostrum,
ali não ficou com medo nem vontade de se proteger. Pelo contrário,
porque aquela música e aquelas pessoas faziam com que ele se sentisse
acolhido, como uma criança sendo envolvida pelo colo de sua mãe.

Diversos índios passavam em sua frente, e Theo conseguia ver tudo à
respeito deles, seu passado e seu presente, seus acertos e seus erros,
tudo. Theo sabia que também estava inteiramente exposto, mas não sentiu
vergonha. Viu que cada um tinha suas limitações, afetos, lutas,
arrependimentos, essencialmente parecidos com os seus, ainda que
pertencessem a outro clã.

Entre paisagens e memórias, Theo avistou a xamã em meio a imagens
diversas e se aproximou, encantado por sua beleza e muito ansioso para
conhecê"-la. Quando a alcançou, foi conduzido para outro lugar e se viu
sozinho em meio à mata, um lugar tranquilo onde a luz do sol atravessava
as árvores e desenhava sombras sobre as águas que repousavam na margem
do rio.

Aquele lugar parecia familiar para Theo, e sentiu um forte aperto no
seu peito. Tentou respirar fundo e calmamente, até que o aperto se
transformou em um calor aconchegante que se espalhou pelo seu corpo.
Então viu adiante a xamã mais jovem, em pé em um barco, remando sobre as
águas, seu ventre claramente carregando um bebê. Ela encostou na margem,
desceu e foi se aproximando de Theo, ficando mais velha à medida que
andava e voltando a ser quem ele havia conhecido inicialmente no ritual.

``Theo, meu menino. Você cresceu tão rápido, virou um grande
guerreiro.''

``O que você quer dizer?''

``Venha, Theo, entre no barco comigo, vamos navegar um pouco.''

Ele sentiu seu corpo sendo levado para o barco e foi colocado sentado na
canoa em frente à xamã. A embarcação começou a navegar pelo rio e entrou
em estreitos fios de água margeados por árvores que se fechavam como um
túnel.

``Eu quero te dar um presente, Theo. Eu, Kokayjho, sou sua mãe, e vou te
oferecer minha história com teu pai.''

O barco avançou, a passagem sob as árvores foi ficando cada vez mais
escura até que Theo se viu em terra firme. Tinha sido levado para sua
aldeia em menos de um segundo, mas havia algo errado, o tempo não era o
mesmo, era como se estivesse em um filme.

Theo caminhou em meio às casas e viu um grupo de homens brancos chegando
do rio. No meio deles estava seu pai, jovem, forte, muito diferente do
homem deitado na cama com o qual Theo convivera na maior parte de sua
vida. Escondida atrás da fresta de uma porta, alguém parecia observar a
chegada dos homens. Era uma linda índia, com os mesmos olhos arqueados
que conhecera e cabelos tão lisos e negros quanto as águas do rio Negro.

``Essa sou eu, Theo. Eu me apaixonei pelo seu pai no mesmo instante em
que o vi.''

``Meu pai também veio para cá? A obra que ele fez na floresta era
aqui?''

``Ele veio ajudar o governo na construção da barragem, quando era
jovem.''

Theo então foi levado para outro lugar. Era uma plantação, onde algumas
índias aravam a terra e, no meio delas, estava Kokayjho. Ela trocava
olhares com o pai de Theo, que caminhava disfarçadamente perto do limite
da horta. Logo, porém, uma índia já idosa a chamou.

``Essa é a minha mãe, sua avó. Ela jamais aceitou o meu envolvimento com
um homem branco.''

Theo quis chegar mais perto de seu pai, e se emocionou ao perceber o
quanto seu jeito de passar a mão no cabelo quando estava disfarçando era
parecido com o dele. Ele então foi levado para diversas cenas em que viu
seus pais se admirando, sempre tentando se aproximar um do outro, até
que foi para um dia em que o pai de Theo conseguiu falar com Kokayjho.
Os índios estavam ocupados com seus afazeres e ninguém percebeu o
contato entre os dois.

``Eu vou te visitar no seu sonho, Kokayjho, hoje à noite'', disse.

``Essa foi a primeira vez que conseguimos conversar, Theo.''

Tudo ficou escuro e, de repente, Theo estava em uma cena embaçada.
Passou a mão nos olhos para clarear a visão, mas tudo continuava turvo.
Ele então percebeu que estava em um sonho dentro de um sonho, era o
encontro que seu pai havia marcado com Kokayjho.

Sob a neblina um pouco adiante, ela nadava sozinha no rio, e seu pai foi
chegando por debaixo d'água, emergindo a certa distância para não
assustá"-la. Quando o viu, ela deixou com que se aproximasse, eles se
olharam e ele acariciou seus cabelos.

``Você é a mulher mais linda que eu já vi.''

Ele se aproximou para beijá"-la, mas ela impediu.

``Homem branco, não posso me envolver com você. Eu pertenço à
floresta.''

Kokayjho rapidamente se transformou em um boto cor"-de"-rosa, e desapareceu
sob as águas do rio, seguida pelo pai de Theo que não pensou duas vezes
e se transformou em um boto maior, prateado sob a luz do sol. Nessa
caçada, eles nadavam com a ajuda das correntes, davam piruetas, o boto
macho fazendo de tudo para conquistar a atenção da botinha rosa, em uma
complicada dança de amor. Quando o boto persistente finalmente conseguiu
encurralar a botinha em um remanso, ela se transformou em uma borboleta
colorida e sobrevoou levemente o rio no que parecia um sofisticado
balé.

Sem hesitar, o boto se transformou em uma borboleta e seguiu a
coreografia, até que a viu se aproximando de um enorme grupo de
borboletas, muito semelhantes a ela. Ela tentava se misturar e se
camuflar às outras, no entanto seu bailado era peculiar e o pai de Theo
não a perdeu de vista nem por um instante.

Quando o grupo se dispersou, Kokayjho voou disfarçadamente e pousou
sobre uma fruta madura caída no chão para se alimentar, até que sentiu
ao seu lado a outra borboleta, com as asas tingidas de fortes cores.
Assim que ele tentou se aproximar, Kokayjho se lançou no ar, ele atrás
dela, até que a suave dança das borboletas se transformou em um bater de
asas.

O que começou com um voo suave foi criando força e vigor até que, na
forma de harpias, eles se lançaram furiosos sobre as copas das árvores
da floresta, desviando abruptamente de galhos e obstáculos em uma
velocidade tão estonteante quanto o desejo deles, subindo até as nuvens,
pairando sobre a floresta, onde finalmente entrelaçaram suas garras em
um abraço mortal. Aceitando o risco do amor proibido, aproveitaram
alguns poucos momentos de contato e intimidade enquanto caíam em direção
à floresta para no último segundo se desvencilharem e, em um arco
ascendente, evitarem a queda mortal no duro chão da floresta.

Finalmente estavam gente novamente, deitados na relva, nus e ofegantes,
apenas com suas mãos suavemente dadas em um lugar onde o silêncio
reinava e nem o canto dos pássaros conseguia chegar.

A paisagem perfeita parecia um jardim secreto, vitórias régias
descansavam sobre as calmas águas, acompanhadas de ninfeias brancas,
amarelas, azuis, vermelhas, enfeitando cada canto do rio. O céu foi
escurecendo e, com ele, as pétalas da flores foram se fechando, e eles não
se importavam mais com nada além do outro ao seu lado.

Theo ainda estava tonto com tudo o que havia assistido quando sua mente
foi levada de volta ao barco. Ele e sua mãe foram lentamente saindo do
túnel de árvores até voltarem ao ritual. Theo olhou para o lado e viu
que os índios, vencidos pela exaustão, dormiam deitados sobre o chão de
terra.

O dia começou a amanhecer, eles foram gradativamente acordando e aos
poucos saindo da Casa do Guerreiro em direção ao rio. Theo sentia"-se
leve e, seguindo os mesmos passos dos índios, deu longos mergulhos no
rio, seus braços se movendo com força a favor da correnteza, sentindo a
água acariciar cada parte do seu corpo. Ao emergir, respirou fundo,
sentindo"-se diferente, calmo, como se acabasse de renascer.

Será que realmente teria encontrado seu passado ali, tão distante de si
mesmo?

\asterisc

%\emph{42 - Theo conversa com a mãe}

Theo passou várias horas olhando o minucioso trabalho de tecer balaios.
Praticamente todas as mulheres da comunidade estavam ali, o período de
colheita e tingimento da palha do tucumanzeiro já havia acabado e agora
a última parte do processo para produção das cestas estava sendo feita.

Ele ainda estava tomado pela emoção que sentiu no ritual da noite
anterior, não sabia avaliar se estava vivendo uma fantasia ou se realmente
encontrara sua mãe. Claro, queria fazer muitas perguntas sobre o seu
próprio passado, mas por algum motivo ficou passivo, esperando que ela
se manifestasse.

Finalmente, no final da manhã, Theo recebeu o recado de Kokayjho para ir visitá"-la. Ao vê"-la,
percebeu instintivamente que estava diante de sua mãe de verdade, que toda a experiência vivida no ritual do dia
anterior fora real. Mas como isso poderia ser?

Assim que entrou em sua casa, estranhando a simplicida															de da morada da
chefe espiritual do grupo, ele se manteve em pé, lívido, visivelmente
desconfortável com a situação.

``Senta.''

Theo se acomodou na pontinha de tronco de madeira, e ela lhe entregou um
pequeno embrulho de folhas de bananeira.

``O que é isso?''

``Pé"-de"-moleque.''

Theo sempre detestou amendoim e murchou por sua própria mãe conhecê"-lo
tão pouco. Provavelmente sua cara o traiu:

``O pé"-de"-moleque do índio é feito com mandioca. Você vai gostar.''

Ele abriu o pacote, deu uma mordida e se surpreendeu com o que parecia
um pequeno pão doce com castanhas do Pará. Ela então lhe trouxe um pouco
de água fresca e se sentou ao seu lado.

``Eu te chamei para contar mais da minha história com seu pai. E da tua
também.''

Theo não sabia se queria saber mais sobre a vida dos seus pais como
homem e mulher. A alegoria que presenciou no mundo dos sonhos já foi
difícil de acompanhar, mas não podia resistir à curiosidade.

``O que aconteceu depois dos encontros nos sonhos?''

``Nós éramos muito felizes lá, Theo, e nos apaixonamos profundamente,
mas era pouco, especialmente para ele. Eu era um pouco mais resignada,
morria de medo de sua avó. Seu pai achava que viver o nosso amor aqui,
um amor real, seria muito mais verdadeiro, mesmo que estivéssemos
sujeitos a contrariedades e livres da perfeição do mundo dos sonhos. Ele
costumava dizer que o lugar de comunhão, que ele e você chamam de mare
nostrum, deveria ser apenas uma ferramenta, e não um fim em si mesmo.''

Theo sentiu seu peito apertar ao ouvir uma frase tão bonita vindo de seu
pai.

``E ele tinha razão, Theo. Quando começamos a nos encontrar de verdade,
nosso amor começou a crescer, mesmo que sempre precisássemos nos
esconder dos meus pais. Nós nos encontrávamos sempre no mesmo lugar, que seu
pai chamou de Paranã Tipi. Significa ``rio profundo'' em nheengatu, era nosso refúgio onde ninguém conseguia nos encontrar. Você quer conhecê"-lo?''

``Como assim, onde é?''

``Venha comigo.''

Kokayjho saiu da palhoça e entrou na mata fechada que ficava logo atrás
da casa. Eles caminharam durante um tempo e, após pulos e desvios em
meio às plantas que impediam a passagem, chegaram a uma trilha.

``Estamos quase chegando.''

De repente, a trilha terminou e Theo viu um delicado braço de rio, o sol
formando traços em meio às folhagens das plantas que pareciam ter sido
cuidadosamente selecionadas pela natureza. Era um lugar muito familiar
para ele, mas não conseguia lembrar de onde o conhecia. Do outro lado do
rio, havia algo, um pequeno clarão na margem.

``Venha.''

Kokayjho mergulhou no rio em direção ao local. Theo a seguiu e, ao se
aproximar, viu uma rede muito velha presa entre duas árvores,
praticamente desfeita pelo tempo.

``Era aqui que eu me encontrava com teu pai. Ele descobriu esse lugar,
adorava desbravar e andar pelo mato, e nós passávamos horas juntos, ninguém
conseguia nos encontrar aqui.''

Theo olhou em volta, tentando se lembrar de onde conhecia aquele lugar,
até que foi surpreendido por sua mãe que mergulhou no rio desaparecendo
sob a água por longos segundos. Não conseguia parar de repetir:

``Paranã Tipi''


Finalmente, a viu ressurgir com um punhado de barro verde na mão.
Ela esculpiu um pássaro e colocou para secar sobre uma pedra.

``Daqui a alguns dias ficará com a cor mais escura e duro como uma
argila.''

Theo começou a se sentir estranho com este déjà"-vu: ali era seu igarapé.
Aliás, agora não era mais seu igarapé, era antes de tudo o igarapé
deles. Ele então se deu conta de que aquele lugar, onde seus pais se
encontravam, era para onde sua mente sempre o levava quando queria
repousar.

``Esse sempre foi meu lugar favorito, Theo, era para cá que eu vinha
quando estava grávida de você.''

Finalmente as coisas começaram a fazer algum sentido: por que aquele
lugar que nem conhecia era tão especial, e por que para navegar ele
instintivamente focava nos ruídos cardíacos. Afinal, enquanto estivera no
ventre de sua mãe, os sons do coração dela eram muito fortes para o bebê,
e assim, ouvindo estes sons, ele navegou pela primeiras vezes na vida,
ainda que inconscientemente.

``Foi por isso que o Alcântara me mandou para cá? Ele queria que eu me
encontrasse com você?''

``Theo, ele te mandou para cá para que você se reencontrasse consigo
mesmo. O Alcântara e teu pai são grandes amigos e fizeram promessas do
que fariam se algo acontecesse ao outro.''

Enquanto caminhavam de volta para casa, ela continuou.

``Você tem muita força mental, Theo, e se tornou um grande navegador,
mas todo este poder de uma só vez não estava te fazendo bem. Decisões
erradas em momentos importantes podem interferir na vida de uma pessoa
de forma definitiva. Então nós decidimos que já estava na hora de você
conhecer sua história.''

Theo pensou em responder que não tinha tomado decisões erradas, mas se
lembrou da prova final da faculdade, e de como estava se focando em
relações superficiais. Calou"-se.

Eles retornaram para casa em silêncio e, apesar de ainda muito
emocionado, viu que tinha mais. Ela abriu a tampa de um cesto guardado
ao lado da rede onde dormia.

``Venha ver, Theo.''

Quando ele se aproximou, viu artefatos antigos e que pareciam
cuidadosamente limpos, entre eles arcos, flechas, um binóculo e uma
chupeta. Theo ficou emocionado, mais ainda quando viu um porta"-retrato.
Ao pegá"-lo para ver melhor, não conseguiu conter as lágrimas diante do
retrato de seu pai ao lado de sua mãe, ela com um bebê no colo.

``Essa é você? Em São Paulo?''

``Sim, Theo, e esse é você. Sente"-se um pouco.''

Theo se acomodou na rede e Kokayjho se sentou em um tronco ao seu lado.

``Tudo estava bem na aldeia, eu e teu pai continuávamos nos encontrando
escondido e ninguém desconfiava de nada. Até que você começou a crescer
na minha barriga, dia após dia, e eu não conseguia mais esconder a nossa
história dos meus pais.''

``Mas por que eles eram contra o amor de vocês? Que grande mal ele
poderia causar?''

``Theo, eu vou te mostrar. Feche os olhos e relaxe, vamos navegar
juntos.''

Ele ainda estava nervoso e desconfortável em navegar com sua própria
mãe, tudo era muito novo e estranho. Mas fechou os olhos e foi mais
fácil e simples do que parecia. Logo, estavam juntos, e a mente de
Theo foi levada a uma antiga memória de Kokayjho.

\asterisc

%\emph{42,1 - Flashback Kokayjho e sua mãe}

Theo olhou em volta e percebeu que eles estavam na Casa do Guerreiro.
Era de noite, e os rostos cansados de duas índias estavam iluminados pelo
fogo das velas.

``Essa sou eu, Theo, eu tinha acabado de passar pelo ritual da
menina"-moça e minha mãe me chamou para conversarmos.''

Theo viu uma linda adolescente, o rosto e o corpo ainda imaturos, mas os
olhos já tão imponentes quanto os da mulher que ele conhecia. Ela
segurava um instrumento indígena enquanto olhava séria para sua mãe.

``Kokayjho, minha filha, guarde bem o que eu vou te falar. Tudo o que eu
te ensinei ao longo desses anos tem um motivo, as lições não foram por
acaso. Pessoas como nós temos um papel muito importante, filha, eu não
sou só a voz do nosso povo, mas eu tenho o dever de conectar nosso clã
ao mundo dos sonhos.''

``Como assim?''

``Sem nos conectarmos, pouco a pouco as pessoas de nossa comunidade
iriam perder a razão, as pessoas seriam apenas mentes isoladas, e não é
possível viver assim. Um dia, minha filha, eu não vou estar mais aqui, e
alguém terá de assumir o meu lugar. Você é essa pessoa.''

Kokayjho ficou preocupada, e sua mãe já adivinhava o porquê. A menina era
curiosa e ambiciosa, sempre se interessou pela vida fora da aldeia.

``Eu? Tudo bem, mãe, se você achar que eu devo. Mas por que eu? São
tantas as mulheres fortes na nossa tribo\ldots{}''

``Porque não são todos que nascem com esse dom, Kokayjho, pelo
contrário, ele vem para poucos. E desde pequena esse poder se manifestou
em você, você foi escolhida.''

``Mas e eu? Eu não tenho direito de escolher?''

``Todos nós somos livres e podemos escolher, Kokayjho, mas parece que
nenhuma outra criança do nosso clã tem esse dom, só você. A decisão é
sua, você pode ignorar o seu dom, mas lembre"-se, tudo tem um preço.

``O que você quer dizer?''

``A sobrevivência de todo nosso povo pode depender de você.''

Aquelas palavras assustaram a jovem mulher, e sua mãe pareceu adivinhar
seu medo.

``Não se preocupe, minha filha, eu ainda sou muito jovem e vou viver
muito. Mas pense com calma, e nas suas reflexões lembre"-se de que a nossa
vida tem sempre um propósito. Pode ser difícil viver regido por
obrigações, mas é um caminho recompensado por grandes alegrias.''

A jovem ficou em silêncio, elas se olharam durante um tempo e a cena
desapareceu diante de Theo.

\asterisc

%\emph{42,2 - Voltam para casa de Kokayjho, mais conversa}

Kokayjho estava com uma expressão triste quando eles retornaram para a
casa dela, e Theo percebeu que aquela história ainda mexia muito com ela.

``O que aconteceu quando a tribo descobriu a sua gravidez?''

``Ninguém falou nada, Theo, apenas minha mãe sabia o que o destino tinha
preparado para mim. Mas ela nunca me obrigou a nada, sempre me deixou
muito livre pra tomar as minhas decisões. Dizia que estava moldando uma
líder.''

Kokayjho sorriu ao se lembrar da sabedoria e fortaleza da mãe.

``Eu estava de sete meses quando a construção chegou ao fim. Teu pai
insistiu muito que partíssemos juntos para a cidade. Ele tinha que
voltar para o destino que seus avós tinham definido para ele.''

``E você foi? Eu achei que tinha sido abandonado logo depois que
nasci.''

``Eu fui, Theo, porque não queria afastar você do teu pai, mas logo que
eu vi a cidade pela janela do avião percebi que as coisas poderiam dar
errado. Lá tudo é muito grande, como posso explicar? Aqui tudo que pode
me atingir ou me interessar está perto de mim, mas lá eu me sentia sem
balizamento, sem limites. Era muito difícil fisicamente, parecia que eu
ia sufocar.''

Theo achou engraçado como ele, apesar de não ter crescido no mato, se
sentia exatamente assim em meio aos tantos prédios de São Paulo.

``Eu estava muito infeliz lá, Theo, mas apesar de tudo isso, não
conseguia nem imaginar ficar longe de você, e por isso estava abrindo mão
do meu futuro, do destino que me foi reservado, pra ficar com vocês. Mas
o destino não se dá por vencido tão facilmente, nós temos bem menos
controle sobre a nossa vida que gostaríamos de ter e admitir. Um dia,
quando você tinha seis meses, minha mãe morreu. Foi um choque porque ela
não dava sinais de que estava doente. Você sabe o que significa quando
uma xamã morre sem ter um sucessor?''

Theo pensou um pouco antes de responder.

``Não tenho ideia.''

``Você percebeu que não consegue se conectar com o mundo distante? Com
seus amigos, por exemplo?''

Theo ficou incomodado ao pensar que não tinha amigos para sentir
saudades, mas aguentou firme.

``O Alcântara me explicou que eu poderia ter dificuldades, por ser uma
região pouco povoada.''

``É para isso que eu existo, Theo. Um xamã é uma antena, eu conecto a
nossa comunidade com o seu mare nostrum, os nossos rituais são
justamente para isto.''

``Então sem você\ldots{} o que aconteceria?''

``Theo, dizem que aqueles que ficam isolados do lugar de comunhão acabam
perdendo o juízo. Já soubemos de clãs inteiros que acabaram
desaparecendo.''

Theo estava com a vista turva pelas lágrimas que estavam quase brotando
de seus olhos.

``Então você não tinha escolha?''

``Eu tinha que voltar para cá, do contrário condenaria toda minha
comunidade. Tenho muita pena quando penso que minha mãe não estava em
paz quando morreu. Ela colocava muita esperança em mim, mas não pôde
saber que eu voltaria para cuidar de tudo.''

Seu rosto se transformou, mudou de cor e as lágrimas começaram a brotar
mais fortes.

``Me lembro como se fosse ontem do dia em que dei você de presente para o
seu pai e fui embora. Foi o dia mais triste da minha vida. Me perdoa,
meu filho.''

Quando o abraço entre eles foi perdendo a força, Theo foi colocando a
cabeça no lugar e encontrou uma paz que não conhecia. Ele finalmente
percebeu que não havia sofrido nenhuma injustiça e que, ele próprio, se
estivesse no lugar de sua mãe, teria tomado essa mesma decisão.

``Vocês nunca mais se encontraram?''

Ela não respondeu. E apesar de haver muito mais a contar e a ouvir, eles
se calaram e ficaram em silêncio por um longo tempo, contemplando as
sombras se alongando no chão irregular do terreno em frente à maloca.
Depois daquele dia, o silêncio entre eles não os incomodaria mais.

\asterisc

%\emph{42,3 - amigos de Theo}

Uma japonesa alta entrou no Bar do Xerife e se sentou perto da mesa onde
eles estavam, o que fez com que Mauro, pela primeira vez desde que Theo
havia viajado, pronunciasse o nome dele na mesa com os amigos.

``Por onde será que anda o Theo?''

``Deve estar cortando o estômago de algum índio.''

``Que horror, Carlos!''

Eles riram e Mayara continuou com a cara de nojo que sempre fazia quando
alguém insistia de falar sobre coisas nojentas durante a refeição. Dani
preferiu ficar quieta, mas não segurou o riso quando Carlos começou a
imitar o jeito com que Theo abaixava a cabeça para fazer algum procedimento
médico.

``Até que aquele playboy faz falta. Lembra quando a gente pegou o carro
da tua mãe escondido, Mayara?''

``Como eu ia esquecer?''

``Aquela festa foi muito maneira.''

``Foi demais mesmo, mas foi foda bater o carro da sua mãe.''

``A culpa foi daquele folgado, ele não é dono da rua.''

``É, mas se não fosse pelo Theo, eu tava ferrada. Ele que assumiu a culpa
toda, e ainda disse pra minha mãe que a ideia de pegar o carro tinha
sido dele.''

``Você tem falado com ele, Mauro?'', Dani perguntou, como quem não queria
nada.

``Nunca mais. Ele viajou e nem falou nada. Mas deve estar tudo bem,
sempre está\ldots{}''

\asterisc

%\emph{42,4 - Mais sobre o pai de Theo}

``E meu pai? Me conta do meu pai?''

``Ele foi um homem muito especial, Theo.''

Ao ver o brilho nos olhos de sua mãe e o sorriso que abriu ao falar
dele, Theo percebeu o quanto ela ainda o amava.

``Por que vocês se afastaram? Ele poderia ter ido com você para a
Amazônia! Ou pelo menos me levado para te visitar às vezes\ldots{}''

Kokayjho percebeu que Theo ainda se ressentia pelo que havia acontecido.

``Você nasceu no meio de uma crise sem precedentes no mare nostrum, Theo,
eram tempos muito difíceis. Todo mundo queria navegar, havia dezenas,
centenas de navegadores. Alunos e aprendizes floresciam o tempo todo.''

``Mas isso é maravilhoso!''

``Seria, se não tivéssemos tantos ataques. Perto da época em que você
nasceu, muitas pessoas não iniciadas tomaram conhecimento do que
significa a navegação. Não se sabe muito bem como isso aconteceu, mas o
controle das informações disponíveis era apetitoso demais para ser
desprezado. Afinal, controlar essas informações poderia promover muito
poder e dinheiro, motivos mais que comuns para iniciar uma guerra. E foi
o que ocorreu. As grandes potências criaram seus próprios grupos de
navegantes que identificavam oscilações de ondas cerebrais no mare
nostrum para encontrar inimigos. Atacavam em grupo, e vários
mestres foram sendo mortos sem nenhuma explicação.
Seu pai não tinha escolha, ele não podia
simplesmente virar as costas para aquela situação. Ele precisou navegar
o tempo todo para ajudar as pessoas. Nessa época foram criados os
métodos de proteção no mare nostrum, os mesmos que usamos até hoje:
as gaiolas fechadas usadas em treinamentos, as técnicas de
autocontrole dos navegadores para que suas ondas não tenham
turbulências. O seu pai ajudou a criar todos eles, ele era um grande
líder.''

``Líder? Como assim líder?''

``Seu pai foi um grande mestre, Theo, um dos melhores. Ele veio de uma
linhagem de grandes navegadores e tinha muita energia mental, mas não
era só isso. Ele também não parava de dar aulas, queria que o maior
número possível de pessoas pudesse ter acesso ao mare nostrum.''

Mais uma vez Theo percebeu os olhos de Kokayjho brilharem ao falar de
seu pai.

``No final, os mesmos mercenários criados para limpar o mare
nostrum dos navegantes, proteger as informações e defender os interesses
das grandes potências foram aqueles que mais usavam as informações para
si mesmos e causavam mais transtornos. O tiro saiu pela culatra, Theo,
afinal os chefões foram entendendo o erro que haviam cometido e o
programa de mercenários foi sendo abandonado. Talvez tenham encontrado
outra forma de interferir no mar, porque uma nova onda de mortes tem
acontecido recentemente. Seu pai faz muita falta, Theo. A liderança dele
naquela época foi importantíssima para manter o equilíbrio entre as
forças. Ele foi um grande mestre.''

Theo olhou para a sua própria tatuagem ainda incompleta no braço e
percebeu que as coisas ainda não se encaixavam muito bem.

"Espera um pouco. Se meu pai foi um mestre, ele teria uma tatuagem!"

Kokayjho se surpreendeu com a rapidez do raciocínio de seu filho e
voltou a falar.

"Um dia ele teve\ldots{}"

"Como assim?"

"Ele apagou a tatuagem dele, Theo."

"Mas por que ele faria uma coisa dessas?"

``Ele ficou com muita raiva do mundo dos sonhos. Nesses tempos negros,
muitos bons navegantes foram sendo mortos, um por um, e ele foi ficando
cada vez mais infeliz. Tudo o que ele queria era ficar em paz com você e
tentar fazer com que fôssemos uma família. Mesmo que nos víssemos pouco,
ele tinha planos de me visitar. Mas esses planos nunca se concretizaram,
porque ele teve de passar o tempo todo navegando para proteger o
mare nostrum. Ele ficou com tanta raiva que achou que era tudo culpa do
nosso dom, e começou a chamá"-lo de marca, maldição. Foi então que ele
resolveu apagar a tatuagem com laser.''

Pela primeira vez, Theo conheceu um lado de seu pai que jamais havia
imaginado.

``Mas não pense que ele fez por mal, Theo, ele fez isso por você.''

``Como assim?''

``Ele não queria que você soubesse que ele era um navegador para não te
dar o mesmo fardo. Ele preferiu te dar de presente a liberdade, a
possibilidade de escolher o seu futuro. Você se tornou um navegador
porque você quis, Theo, e não apenas por ser nosso filho.''

Theo se lembrou do dia em que Cristine o procurou no \versal{SVO} e tudo foi
começando a fazer sentido.

``Foi você quem deu cada passo da sua descoberta, você traçou o seu
caminho sozinho.''

Falava isto enquanto esticava a pele e olhava a tatuagem do filho,
nostálgica. Theo ficou em silêncio, e Kokayjho percebeu que estava na hora dele saber
de tudo.

``Se eu soubesse que o dia em que voltei para a Amazônia seria o último em
que nos veríamos, bem\ldots{} Eu já estava contando os dias para
reencontrar seu pai, mas aí veio aquele ataque e nós nunca mais pudemos
nos reencontrar direito\ldots{}''

``Ataque, que ataque?''

``O teu pai também foi atingido no mare nostrum, Theo. Ele estava dando
aula quando um grande grupo de inimigos chegou. Ele mandou todos os alunos irem embora, mas
alguns teimaram em ficar. Seu pai era muito
forte. Mesmo estando em franca minoria, só foi subjugado porque teve que
desviar sua energia para defender seus discípulos.''

``Não é possível! Meu pai teve um \versal{AVC}, eu mesmo vi os exames!''

``Ele teve sim uma lesão no sistema límbico do cérebro, Theo, mas porque
o inimigo o atingiu de propósito no córtex cerebral para que ele nunca
mais pudesse sonhar nem ensinar ninguém. Ele o deixou isolado dentro de
sua própria mente, como se estivesse preso em uma represa que não
consegue se conectar com o mar. Eu suspeito que ele tenha uma pequena
regeneração e consiga se conectar um pouco, como um pássaro de asas
quebradas.''

Theo ficou em silêncio, abaixou a cabeça e chorou muito. Kokayjho se
levantou e balançou a rede por um longo período, até que Theo foi fechando os
olhos e adormeceu.

\asterisc

%\emph{Novo - Marenostrum II - 43 - Marc joga e se abre com Fred. Senhor D percebe}

``Opa, tô chegando\ldots{}''

Fred entrou na grande casa envidraçada animado. Fazia tempo que estava
esperando encontrar seu amigo, queria tirar a má impressão do último
encontro.

``Tem alguém aí?''

Ele não encontrou ninguém e, como se fosse da casa, subiu para procurar
por Marc. A porta da biblioteca estava fechada e, ao fundo, tocava o
mesmo som de piano que ele escutou na primeira vez que entrou ali, uma
das melodias mais bonitas que já havia ouvido em toda sua vida. Entrou
sem bater e o encontrou deitado no divã.

``Oi, doutor! Que bom que o senhor está melhor!''

Marc tomou um susto e ficou furioso. Como alguém ousava interromper seu
ritual de recomposição mental? Estava ali há dias, mas não havia se dado
conta, a música apenas começou tocar e ele, como sempre, se dirigiu ao
divã e ali ficou. Se levantou bravo, pronto para brigar com Fred, mas
foi desarmado pelo sorriso e pela postura relaxada do faxineiro.

``O que deseja?'', perguntou, já de costas para não expor seu sorriso.

``Eu achei que gostaria de uma visita, conversar, passar o tempo,
sabe?''

Ainda forçando uma feição reticente, Marc convidou Fred a se sentar.

Rapidamente este estava esparramado em uma poltrona contando
longamente causos e piadas que viveu, e Marc experimentava sensações de que
já nem se lembrava mais. Uma alegria percorreu toda sua mente e
até seu corpo parecia funcionar melhor.

Ficou surpreso com as emoções daquele dia, há muito tempo seus encontros
com um ser humano se limitavam apenas a Teresa, Senhor \versal{D.} e os inimigos
que matava em poucos segundos em suas missões.

``Doutor, o senhor disse que trabalha para defender segredos. O senhor
sempre trabalhou com isso?''

Marc assinou um contrato com o Senhor \versal{D.}, e sabia que não podia
contar do seu trabalho para ninguém a não ser sua esposa, caso contrário
seria morto. Fred, porém, lhe passava confiança, e ele sentiu que poderia
contar a verdade para o amigo, sabia que ele não lhe faria mal.

``Sabe, Fred, eu sempre sonhei em ser um agente da Central de
Inteligência. Hoje eu não trabalho exatamente lá, mas eu ajudo o nosso
país de outra forma.''

``Que legal, doutor. E o que o senhor faz?''

``Eu trabalho para uma outra agência que ajuda a Central, defendendo
nosso país de terroristas, inimigos que querem descobrir segredos e
acabar com a proteção da nossa terra.''

``Pô, que legal. Eu nem sabia que tinha gente trabalhando nisso. Deve
ser melhor que trabalhar nessa tal de Central de Inteligência, né?''

``Pra falar a verdade, eu queria mesmo era trabalhar na Central. Mas não
deu certo.''

``É mesmo?''

``É, Fred. Eu fiquei muito tempo tentando ser aprovado, sem sucesso, a
minha vida tava uma merda completa. Finalmente, um dia, eu recebi um
telefonema, dizendo que fui chamado para um teste.''

``E aí?''

``E aí que eu tive muito azar\ldots{} o avaliador era um filho da puta que
não foi com a minha cara, e por isso me reprovou.''

Marc ficou transtornado só de se lembrar daquele dia e se levantou com
raiva, deu um chute no sofá, até que voltou a sentar.

``Mas isso acontece, né? Eu mesmo já fiquei uns dois anos procurando
emprego, já tomei muito `não' na cara.''

``Mas não foi só isso, Fred. Eu tava muito transtornado quando saí da
entrevista, muito nervoso. Eu tava com a minha mulher no carro, ela
estava grávida, mas eu não ouvia nada. Até que eu bati o carro, foi um
acidente inevitável, e fomos parar no hospital, eu e ela. Sabe o que é
pior, Fred? Eu sei que eu sou um fracassado, no fundo eu sei que tudo
foi culpa minha, a nossa vida, eu ter perdido a vaga na Central\ldots{}''

``Não pensa assim, não, doutor. Se não o senhor vai
enlouquecer.''

``Mas é a verdade. Depois do acidente eu apaguei, mas quando acordei eu
conheci um cara, o assistente do meu chefe. Ele me ofereceu esse
emprego, esse lugar para morar. Mas é tudo muito estranho aqui, essa
casa, minha mulher tá grávida e meu filho não nasce, algumas coisas aqui
não têm sentido.''

Fred achava que estava começando a entender o que estava acontecendo.
Seus pensamentos o levaram para longe, aqueles dois pacientes sozinhos
na \versal{UTI}\ldots{} será que era isso mesmo que ele estava imaginando,
ou estava ficando louco?

De repente, seus pensamentos foram interrompidos por gritos e uma
máquina que apitava sem parar. Fred abriu os olhos e viu que estava de
volta ao armário de vassouras no corredor que levava à \versal{UTI}.

Assim que saiu do armário, ouviu a discussão na sala da diretoria:

``O que é isso, Jonas? Dê um jeito nessa máquina imediatamente!''

Era a voz do Senhor \versal{D.}, furioso.

``Tem alguma coisa errada, Senhor. O sistema está funcionando, ele só
dispara quando há comportamentos inesperados. A frequência dele saiu do
padrão, as ondas se tornaram orgânicas.''

``Como pode ser, idiota? O que você fez com ele?''

``Não fiz nada, Senhor, ele está em casa. O estímulo e a música são os
mesmos de sempre. Acho que o corpo dele está reagindo, mas
como?''

Fred não sabia exatamente o que estava acontecendo, mas suspeitava que
tinha a ver com ele. Procurou fazer seu trabalho quieto, fingindo limpar
o corredor enquanto tentava ouvir a conversa. Ele não podia ser
descoberto, ainda mais agora que ficou amigo de Marc. Começou a suar só
de imaginar o que aqueles homens poderiam fazer caso suspeitassem que
ele, um pobre faxineiro, andava visitando a ``arma secreta'' deles nos
últimos meses.

\asterisc

%\emph{Novo - Marenostrum II - 44- Fernanda comenta com o Theo que ele está diferente}

Mais um dia terminava, e Theo havia passado o tempo todo atendendo os
índios, desde manhã cedo. Ficava feliz em atender, era
justamento o que lhe dava prazer, ainda mais agora que os índios
confiavam nele e podia ajudar pelo menos um pouco aquelas pessoas.

Estava voltando do banho no rio e foi para seu quarto para deitar um
pouco na rede quando Fernanda entrou para descansar.

``E aí, Theo? Trabalhou muito hoje?''

``Bastante, foi muito legal.''

``Como você tá diferente, Theo. Parece que está em casa, mesmo. Viu,
posso comer teu chocolate? Tá ficando seco, tá velho já.''

``Pega aí.''

Fernanda pegou o doce olhando para Theo, meio desconfiada.

``Nossa, você mudou mesmo.''

Vagarosamente foi e deitou na rede junto com ele, virada do lado oposto,
e ficaram olhando um para o outro.

``To pensando em furar minha orelha, Fernanda. Você fura pra mim?''

Ela olhou para ele e riu.

``Furo. Vamos lá.''

Eles foram para o consultório de Theo e ele separou um fio de sutura
esterilizado. Fernanda achou exagero este cuidado todo para evitar uma
infecção, mas não quis criar caso e furou a orelha de Theo com a agulha.
Seguindo sua orientação, fez um nó com o fio para manter o buraco
aberto.

``Vai virando o fio, cuidado só com o nózinho. Com o tempo o buraco
cicatriza.''

``Você já sabe que brinco vai colocar?''

``O espeto de tirar os bichos de pé do Picun. Vou ficar igual a uma
índia.''

Theo olhou para Fernanda e deu um sorriso divertido. Ela riu e teve a
certeza de que ele não estava só mudado, mas também marcado para sempre
com a cultura daquele lugar.

``Eu estava pensando\ldots{}''

``Fala, Fê.''

``Está muito quente hoje, vamos nadar?''

``Agora?''

``Não, mais tarde\ldots{} à noite.''

Theo sorriu.

\asterisc

%\emph{Novo - Marenostrum II - 45 - Xamã fala para o Theo que o Alcântara morreu}

Theo estava morrendo de sono, mas não queria deixar sua mãe esperando.

``Me chamou?''

``Chamei, Theo. Sente"-se.''

Ela estava recolhida, com uma expressão preocupada.

``Aconteceu alguma coisa, mãe?''

Kokayjho se enterneceu ao ouvi"-lo chamar de mãe, mesmo seu
coração estando apertado com a notícia que havia recebido.

``Theo, sei que não vai ser fácil pra você. Você vai ter de ser forte.
Aconteceu uma coisa muito ruim, o Alcântara foi atacado no mare
nostrum.''

``Mas ele está bem? Deixa eu ligar pra ele!''

Theo já estava saindo quando Kokayjho o impediu.

``Theo, espera. Não dá, ele já se foi\ldots{}''

Ele não acreditou no que acabara de ouvir, como era possível? Tinha
conversado com seu mestre há algumas semanas e estava tudo bem. Se
soubesse que seria a última conversa, teria falado tantas coisas.
Theo se sentou, perplexo, baixou a cabeça e começou a chorar. Sua mãe o
abraçou por trás de seus ombros e ficou em silêncio.

``Eu vou fazer um chá para você.''

Tomou o chá, e o líquido quente trouxe um aconchego. Ela se sentou
em sua frente, até se acalmar.

``O que aconteceu? Você sabe como foi?''

``A notícia já se espalhou pelo mare nostrum. Foi aquele homem terrível,
que tem muito poder, é impossível enfrentá"-lo sozinho.''

``Ele estava vestido de preto, parecendo uma máquina?''

``É ele mesmo, Theo.''

``Ele já tentou me pegar!''

``Recebemos informações, agora ele está mais
forte do que nunca.''

``Não acredito! Quem é esse homem? Eu preciso fazer alguma coisa, alguém
precisa parar essa coisa!''

``É impossível, Theo. Ele é muito forte, ninguém nunca conseguiu. Já
houve outros que quiseram patrulhar o mare nostrum, mas este, este é
especial. Há anos ele vem matando os mestres mais poderosos que já
existiram.''

``Mas como é possível?''

``As batalhas no mare nostrum são razoavelmente simples, tudo é uma
questão de força da onda mental, concentração e foco. Este cara tem uma
força artificial, por isso ninguém pode com ele. Seu pai suspeitava que
um aparelho, como um amplificador de ondas sonoras, poderia ser usado para amplificar ondas
mentais.''

Parou por um segundo:

``Theo, escuta, você também está correndo perigo. É
melhor você não voltar para São Paulo, porque ele vai te achar quando
você navegar. Fique aqui, ele não vai te encontrar na Amazônia, nós
estamos protegidos nessa tribo.''

``Ficar aqui? Para sempre? Não dá! Eu preciso voltar para casa, terminar
minha faculdade. E meu pai? É a minha vida, eu não quero largar tudo.''

``Mas você pode reconstruí"-la aqui, podemos trazer seu pai de volta. Não
vai te faltar nada, Theo, nós podemos oferecer tudo o que você
precisa.''

Por alguns segundos Theo tentou se imaginar largando sua casa, a ideia de trabalhar
em um hospital, a residência médica\ldots{} e seus amigos? Andava distante
deles, é verdade, mas se lembrou de Mauro, Carlos, Mayara, Letícia, até
de Dani, não queria se afastar para sempre. Sabia também que não seria
bom para seu pai deixar o conforto de sua casa e toda a assistência que
recebia. Não, não seria possível, lá era o lugar dele e ele teria que
voltar.

``Não, eu não posso, mãe. Você já viveu fora do seu mundo, você sabe
como é. Eu não posso ficar aqui para sempre, eu não sou daqui. Eu não
sou índio, mãe.''

As palavras de Theo fizeram com que Kokayjho se lembrasse de quando teve
de deixá"-lo para voltar para a aldeia, e ela se emocionou. Enxugou as
lágrimas rapidamente e apertou os lábios, preocupada com o que poderia
acontecer com ele.

``Tudo bem, meu filho. Mas então eu vou ter de te preparar.''

``Como, o que você está pensando?''

``Eu vou te ensinar algumas técnicas para você se proteger. Mas não
hoje, outro dia. Agora vamos para a Casa do Guerreiro, eu vou convocar
todos para o ritual do silêncio em honra do Alcântara.''

\asterisc

%\emph{Novo - Marenostrum II - 46 - cerimônia do silêncio, dom xamã}

A cerimônia do silêncio foi diferente de tudo o que Theo já havia visto.
Os índios fizeram uma roda e a xamã cantou uma breve melodia que foi
repetida em um loop por vários minutos, de tal forma que o som foi
criando forma em sua mente e ocupando seu cérebro, se expandindo até
expulsar outros pensamentos. Acabou, assim, meditando sobre o que havia
significado a vida do seu mestre.

Theo levou um susto quando a xamã os despertou do transe, marcando o
final daquela parte da cerimônia, e lentamente foi formada uma roda com
todos os presentes, que começaram a cantar e dançar freneticamente.
Quando acabou a cerimônia, todos foram tomar banho no rio e Theo os
acompanhou.

Durante os dias que se seguiram a cerimônia, Theo ficou pensando em tudo
o que aconteceu, tentando entender o significado de cada passagem do
ritual indígena.

Ele estava atendendo em seu consultório quando recebeu o recado de que
sua mãe o chamou, ela estava em sua casa o esperando.

``Oi, mãe.''

Ela o abraçou com força.

``Como você está, Theo?''

``Estou melhor, o ritual me fez bem.''

``Sempre fazemos essa cerimônia quando morre alguém do nosso clã ou uma
pessoa muito próxima.''

``Quando você falou `cerimônia do silêncio', não imaginei que terminaria
com uma dança.''

``Esse é um ritual muito antigo, Theo, foi criado pelos nossos
antepassados há muito tempo, não sabemos exatamente quando. Tudo tem um
significado ali, o canto melódico é para lembrarmos das boas ações de
quem se foi, porque o que fazemos de bom é o que deixamos nesse mundo. O
silêncio é o que nos permite filtrar as vozes externas para abrir as
portas da própria consciência e ouvir a voz interior, a que realmente
importa. No ritual, o silêncio faz com que cada um se recorde do
significado individual que aquela pessoa teve na sua vida e como ela
ficará guardada dentro de você.''

``E a dança?''

A alegria daqueles índios e a energia com que eles dançavam diante da
morte estava intrigando Theo.

``Nós dançamos para comemorar a partida, Theo, porque aquele que se foi
agora está bem junto do sagrado e isso é motivo de festa. Ainda que a
gente sinta tristeza e saudades, acreditamos que quem partiu está bem,
em comunhão com o mundo. Então nós dançamos!''

Theo não entendia como, mas aquelas palavras e tudo o que aconteceu
naquele ritual realmente o ajudaram a ficar bem diante da morte de seu
mestre. Ele estava triste, mas era como se a força da dança daqueles
índios tivessem lhe dado esperança.

``Theo, essa é apenas uma das sabedorias do nosso povo, eu vou te
ensinar muito mais. Você se lembra do ritual que você participou, de
conjunção de mentes?''

``Claro, foi uma coisa muito diferente.''

``São poucas as pessoas que conseguem conjugar as mentes como você me
viu fazendo. Essa capacidade garante que nossa comunidade possa
sobreviver conectada com o mundo, comunidades sem um xamã definham e
morrem, as pessoas enlouquecem. Eu nasci com este dom, não é uma
escolha.''

``Como você descobriu esse dom?''

``Eu descobri isso quando tinha 11 anos, estava em casa com minha
família e de repente todos começaram a falar sobre o mesmo assunto. Eu
não entendi por que isso tinha acontecido, mas minha mãe era xamã na
época e logo percebeu que eu tinha derrubado as barreiras da mente e
unido todos em um só pensamento.''

``Eu tenho a impressão que isso já aconteceu comigo algumas vezes! Antes
de me tornar navegador!''

Kokayjho deu um sorriso pelo que acabara de saber.

``Isso é um dom, Theo, um grande presente! Mas custou muito caro para
mim.''

Theo não estava para grandes devaneios hoje, não suportaria uma grande
conversa sobre qualquer coisa porque, inconscientemente, queria evitar a
conversa emocional sobre a separação entre eles e acabou se focando em
algo prático.

``E eu tenho esse dom?''

``Parece que sim, Theo, mas não acredito que você tenha compreendido
plenamente o que isso significa. É uma coisa muito poderosa, tenha muito
cuidado e não brinque com isso, esse dom não está em você para te dar
vantagens pessoais. Eu sei muito bem como foi que o Alcântara conseguiu
te mandar para cá.''

Ele fingiu que não entendeu.

``E quando for necessário, o que eu devo fazer?''

``Cada um tem a sua própria maneira de derrubar as paredes das mentes,
Theo, é uma senha muito pessoal. Você vai ter de descobrir a sua.''

\asterisc

%\emph{47 - Ritual de despedida fabio ok}

O tempo de Theo na comunidade estava quase chegando ao fim e ele tentava
aproveitar o máximo para conviver com os índios e principalmente com sua
mãe.

Todos os dias, depois de atender seus pacientes, ele se encontrava com
Kokayjho em sua casa, ondes conversavam durante horas sobre a sabedoria
indígena, as histórias da Amazônia e os ensinamentos milenares que foram
sendo transmitidos de geração para geração. Sem que ele percebesse, sua
mãe foi aprimorando os ensinamentos que ele havia aprendido com seu
mestre na cidade.

``Theo, tudo isso que estou te contando faz parte do seu treinamento. O
conhecimento dos antigos vai te ajudar a lidar melhor com os problemas e
com suas emoções. Quando você estiver com a mente tranquila, em
verdadeira paz, você poderá navegar praticamente como se fosse
invisível, porque qualquer turbulência será eliminada de sua mente.''

``Theo, suas necessidades individuais são pouco importantes frente às
necessidades da comunidade.''

``De onde você tirou a ideia que o sucesso depende da opinião de outras
pessoas? O sucesso é um conceito pessoal.''

A cada dia ele ia aprendendo um pouco mais sobre navegar e sobre o
``lugar de comunhão'', como sua mãe chamava o mare nostrum.

Um dia, ele acordou e logo sentiu um clima diferente no ar. Frutas
frescas na mesa do café, Osmarina cantarolando toda alegre na cozinha
enquanto preparava os alimentos que seriam servidos no almoço.

Antes do anoitecer, Kokayjho mandou chamar Theo para que ele jantasse
com ela, e ao chegar ele viu sua mãe sentada em um tronco preparando a
tinta para pintura corporal. Sem falar nada, ela fez sinal para que ele
sentasse e começou a pintar sua pele com traços finos e regulares. O
padrão era conhecido de Theo após o período em que esteve na comunidade,
mas em seu peito, Kokayjho fez um desenho cujo significado ele não entendeu.

``O que é isso?''

``Theo, você está pronto. Hoje você vai entender o meu papel para nosso
povo.''

Ao terminar o jantar, eles foram para a Casa do Guerreiro, e Theo se
juntou aos índios. Eles começaram a dançar exatamente da mesma forma que
o ritual anterior, a dança progressivamente aumentando até que sua
mãe se aproximou dele, segurou suas mãos e olhou em seus olhos. Sem
qualquer movimento de seus lábios, Theo entendeu o que ela quis dizer.

``Perceba como as ondas cerebrais vão entrando em consonância, eles estão
prontos.''

Apesar de fisicamente estar parada na frente dele, Theo sentiu sua mãe,
através de sua mente, se comunicando com os participantes da cerimônia.
Diferentemente do primeiro ritual em que havia participado, dessa vez
ele conseguiu prestar mais atenção no que estava acontecendo e percebeu
o gesto que os índios estavam fazendo. Eles iam se curvando levemente,
como se estivessem se reverenciando uns aos outros, o que funcionava
como uma senha que derrubava a barreira entre as mentes unia a todos.

Theo percebeu que havia chegado sua vez quando um paciente seu o olhou
nos olhos e se curvou. Theo pôde sentir sua sincera gratidão, um
sentimento que o encheu de alegria. Até que olhou para o lado e viu sua
mãe sorrindo em sua direção, sinalizando o que ele deveria fazer. Theo
fez um enorme esforço para dobrar suas costas, mas havia uma enorme
força dentro dele que não o deixava, era como se a dor por tudo o que
ele havia passado até hoje o impedisse.

De repente, todos os índios começaram a dar as mãos e fizeram uma grande
roda em seu redor. Eles então dobraram os joelhos e se curvaram até o
chão. Sentindo uma força inexplicável, Theo lentamente caiu sobre seus
joelhos e curvou as costas lentamente até sua testa tocar o chão em um
movimento suave e indolor, mas poderoso. De olhos fechados, honrando sua
mãe, vieram a sua mente imagens dela e de seu pai, assim como de seus
amigos e pessoas que ele nem reconhecia, mas que sabia que faziam parte de
sua vida de alguma forma. Não havia mais julgamentos ou rancor, tudo o
que ele sentia era um profundo sentimento de gratidão e amor.

``Theo, você conseguiu. O segredo da nossa união é a gratidão, e você
precisava me perdoar para reconhecer que eu te dei tudo o que eu pude,
perdoar o mundo pelas curvas que sua vida fez e principalmente se
perdoar pelas suas imperfeições e aceitar quem você realmente é'', ouviu
em sua mente. ``Nada mais te separa de mim e do nosso povo, agora somos
todos um.''

Claramente, ela tinha razão: ele já estava unido a todos os índios e
tomado pela energia do ritual. Percebendo que a força de
todos extrapolava em muito a força de cada um somada, sentia que aquela união
potencializava a energia de todos. Olhando para o lado, ele levou um
choque porque percebeu que cada um dos índios era grato de alguma forma
a quem estava ao seu lado, seja por ter ganhado um peixe de seu vizinho
ou simplesmente por dividirem a mesma terra.

\asterisc

%\emph{48 - Theo vai embora}

Theo quis aproveitar ao máximo seu último dia na aldeia. Apesar do
cansaço devido ao ritual da noite anterior, ele estava cheio de energia
e foi logo para o consultório dar as últimas instruções aos pacientes
com quem não havia conseguido conversar, ensinando como continuar o
tratamento em sua ausência.

Fernanda continuaria mais um tempo na aldeia e prometeu cuidar de Picún
e sua mãe, o que deixou Theo mais tranquilo. Nos últimos dias a relação
deles estava mais próxima do que nunca, havia um certo desespero de
viver o máximo que podiam já que sabiam que logo estariam distantes.

Na última noite que passaram juntos, Theo começou a ensaiar um discurso
de despedida, mas foi logo interrompido.

``Não estraga tudo sendo machista, Theo. Vivemos o momento de forma
plena, você não me deve nada e nem eu a você, quem disse que eu ia
querer namorar você se você morasse aqui? Mais uns dias e eu esqueço
você. Seu coxinha metido!''

Disse isto de um jeito tão carinhoso que encerrou o assunto. Jamais se
esqueceriam.

As malas de Theo já estavam prontas quando ele voltava de seu último
mergulho no rio. Sorte ou não, estava vazio quando foi se banhar e pôde
pensar com calma em tudo o que aconteceu na viagem, contando apenas com
o barulho das águas batendo nas pedras e dos passarinhos cantando
escondidos nas árvores.

Colocou sua mochila nas costas e já estava indo para a casa de sua mãe
para se despedir quando uma criança veio o chamar. A pequena indiazinha
o pegou pela mão e ele deixou"-se conduzir até a entrada da aldeia.

Ao chegar perto do barco, Theo encontrou todos os índios da tribo o
esperando, pintados como se estivessem esperando por uma grande festa.
Eles o abraçaram e lhe deram muitos presentes. Picún saiu do lado de sua
mãe e pulou no colo dele quando o viu, ficando grudado durante toda a
despedida. Após muitos abraços e algumas lágrimas emocionadas, Theo
finalmente foi se despedir de sua mãe. Sabendo que não seria fácil, eles
deram um longo abraço e Theo agradeceu por tudo o que ela fez por ele.
Sabiam que ainda se veriam muitas vezes, mas não tinham ideia de
quando.

Theo sentiu falta de Fernanda na despedida, e entrou no barco um pouco
decepcionado. O prático ligou o motor e eles estavam prontos para ir
embora quando ela apareceu correndo com um enorme buquê de plantas na
mão, toda suada e feliz por ter conseguido chegar a tempo.

``É um de cada espécie, são plantas que só existem na Amazônia'', disse,
tentando respirar. ``Eu acabei de colher, vão te dar sorte.''

Ele a agradeceu feliz, eles se abraçaram e o barco já estava saindo
quando a ouviu gritar da aldeia.

``Não esquece de pegar sua mala de rodinhas!''

Eles riram e o barco avançou sobre o rio. A viagem de volta foi muito
mais agrádavel que a de ida. O calor já não o incomodava e Theo
apreciava a natureza já com saudades do que não iria encontrar em São
Paulo. Lembrou"-se de que também não encontraria seu mestre e sentiu uma
dor de saudade, até que veio a sua memória a imagem dos índios no ritual
do silêncio, dançando felizes diante da perda. As palavras de sua mãe
ecoaram em sua mente e trouxeram"-lhe paz. Ele sabia que tudo que
vivera e aprendera estaria vivo dentro dele e desejou apenas ser capaz
de utilizar bem toda essa sabedoria.

\asterisc

%\begin{itemize}
%\item
%\end{itemize}

%\emph{49 - Theo navega com o pai}

A casa estava muito bem cuidada, como Theo tinha imaginado. Afinal, a Léa
trabalhava para sua família havia muitos anos e ele sabia que poderia
deixá"-la cuidando de tudo. Theo entrou bem devagar no quarto e
seu pai estava deitado na mesma posição de sempre.

``Doutor Theo? Não sabia que o senhor voltaria hoje, a Léa não nos disse
nada.''

Francisca, uma das cuidadoras contratadas para ficar com seu pai, estava
surpresa por vê"-lo ali.

``Onde está a Léa?''

``Ela teve que levar a mãe para fazer exame, vai chegar mais tarde.''

``Francisca, hoje você pode ir embora mais cedo, pode deixar que eu fico
com meu pai.''

Theo queria ficar sozinho com seu pai. Sentou ao pé de sua cama e não
sabia por onde começar, havia tanto a dizer. Ele então fechou os olhos
para navegar e em pouco tempo estava em seu igarapé. Dessa vez, no
entanto, a sensação era diferente porque ele sabia todos os significados
daquele lugar. Ele viu algo se mexer em uma árvore e avistou o conhecido
passarinho azul. Sabendo que era seu pai, também se transformou em um
pássaro e voou atrás dele.

Os dois subiram bem alto em direção ao céu e sobrevoaram juntos a
Amazônia. Theo observou a rapidez e a agilidade de seu pai e se lembrou
de tudo o que sua mãe havia contado sobre ele e seus alunos, o que o
encheu de orgulho e lhe deu mais força para voar. Eles passaram pelos
conhecidos pontos da aldeia de Kokayjho, brincaram e caçaram junto. Theo
sabia que eram apenas imagens que existiam na memória dos dois, pois
eles não conseguiriam se conectar com os índios que ali viviam devido ao
seu isolamento, mas isso não importava. Ele estava voando com o seu pai
e só de poder fazer coisas ao seu lado bastava.

Theo então acordou, olhou para o lado e viu seu pai deitado na cama,
parecendo sorrir. Ele sorriu também e o abraçou, chorando muito. O voo
que fizeram juntos veio para Theo como uma redenção, ele finalmente
percebeu o quanto havia sofrido com a sua paralisia e conseguiu
perdoá"-lo por sua doença. Durante o abraço dos dois, agradeceu"-o por
tudo o que tinha feito por ele.

Theo ouviu um barulho na cozinha, levantou e, enxugando as lágrimas, foi
correndo encontrar com Léa.

``Ô menino, você já voltou? Nem falou nada?''

Ele a abraçou forte e começou a agradecê"-la, lembrando pequenas coisas
do passado.

``Que é isso? Não precisa agradecer, não, esse é o meu trabalho, seu
Theo.''

``Independente disso, Léa, você faz tudo com muito amor e isso é o que
importa, eu tenho muita gratidão a você.''

``Imagina, seu Theo, a coisa mais valiosa que a gente tem na vida são os
amigos. Vocês significam muito pra mim.''

Aquela frase veio como um murro na cara de Theo. Onde estavam seus
amigos, será que eles ainda o receberiam?

Pegou o celular e ligou para Mauro.

``Theo? É você?''

\asterisc

%\emph{50 - Theo encontra os amigos}

Theo chegou ao Bar do Xerife, mas não entrou diretamente. Ficou do outro
lado da rua olhando os amigos à mesa, com medo de como seria recebido.

Checou as mensagens algumas vezes, mesmo sabendo que não encontraria
nada de novo, e então desligou o celular, criou coragem e andou até eles.
Carlos foi o primeiro a olhá"-lo, mas não o reconheceu.

``Carlos!''

``Theo? É você?''

A cor da pele, a barba e o cabelo comprido o deixaram confuso.

``Pô, não é possível. Se você sempre fosse assim a gente até fundava um
partido juntos. O \versal{PBSR}, Partido dos Barbudos e Sem Rumo.''

Eles riram e Theo sentiu pela primeira vez que Carlos não estava sendo
irônico.

Cumprimentou um por um da velha turma com abraços apertados e piadas,
até Dani -- para quem ele preferiu não fazer piada.

Depois de alguns minutos juntos, quando o silêncio veio, ele percebeu
que não seria tão fácil falar, mas sabia que eles esperavam alguma coisa
e era responsabilidade dele corresponder.

``Eu sinto muito a falta de vocês, tenho muita gratidão pela nossa
amizade. Muita coisa aconteceu, tenho que atualizar vocês, muitas vezes
eu nem conseguia perceber o quanto nossa conexão é importante para mim,
mas é, é muito importante.''

``Ooohn.''

Mayara achou a frase fofa e foi abraçar Theo, Carlos falou que ele
estava muito gay, Mauro deu uns tapinhas camaradas nas suas costas e
Dani disfarçou, porque por dentro estava achando tudo aquilo muito
bonitinho.

``Atualizar? Vai nos contar sobre a viagem pra Amazônia?''

``Não, vou contar a vocês sobre o mare nostrum, sobre como eu me perdi e
me achei, e também sobre como vou mostrar quem vocês podem ser.''

Eles o olharam sem entender nada.

``Alguém aí se lembra o que é uma gaiola de Faraday?''

Apesar de estar com medo, Theo estava pronto para ensiná"-los a navegar.

\chapter{Carta do autor}

Para fazer justiça às várias pessoas envolvidas -- algumas até citadas --, assim como à maravilhosa e riquíssima diversidade cultural indígena, algumas ressalvas devem ser feitas.

Muitas das situações descritas durante a transformação do Theo em sua experiência em área indígena realmente aconteceram.

Nos últimos 10 anos tenho atuado como cirurgião na \versal{OSCIP} Expedicionários da Saúde, e, nas idas e vindas em área indígena, algumas destas situações foram vividas por mim, outras por colegas expedicionários, e algumas delas foram inventadas mesmo.

Admito que realmente cogitei limpar a areia quando estive em Surucucu, na fronteira com a Venezuela. Não me conformava em ver crianças com risco de perder os pés por causa de uma infestação com insetos em pleno século XXI. Não tentamos ensinar os cachorros a usar jornal na floresta, mas fizemos curativos com silver tape e Crocs.

Por fim, eu e meus grandes amigos Marcelo Averbach e Fabio ``Pinguim'' Paganini, após limpar centenas de bichos"-de"-pé, compreendemos que a doença era social, e merecia uma abordagem que nossa expertise de cirurgiões não resolveria. Era necessária uma compreensão mais profunda e global da doença e do doente. Aliás, como sempre deveria ser a medicina.

Graças a Deus nunca tivemos uma situação tão trágica como a fratura exposta narrada no livro, mas toda contextualização bem poderia ter acontecido\ldots{}

Os costumes e tradições também merecem uma nota.

Os fatos narrados neste livro são pura ficção, não há na rica diversidade cultural do nosso país uma comunidade como a descrita. Ela foi criada como uma colcha de retalhos baseada em minhas experiências em áreas indígenas, muita conversa ao redor de fogueiras e muitos ``causos'' contados.

Para dar andamento à saga do Theo, criei esta comunidade como uma grande licença poética autoconcedida.

No entanto, as transformações experimentadas pela personagem durante o livro de fato ocorrem.

Posso assegurar que muitas vezes percebi estas mudanças em meus ``irmãos de armas'' médicos, enfermeiros e voluntários dos Expedicionarios da Saúde, durante nossas idas e vindas pelo Brasil. Não quero dizer com isso que precisamos ir para o meio da floresta para encontrar nosso verdadeiro eu. Cada um tem o seu caminho próprio, mas afirmo sem medo de errar que sair de seu cotidiano, abrir mão do conforto e ter um perspectiva externa de si mesmo faz milagres para o crescimento pessoal e para estabelecer as prioridades na vida. Foi o que aconteceu com o Theo, comigo e com muitos outros Expedicionários. Eu aconselho.
